%We have some reason to suspect leverage already, the constraints drive returns toward BH and some reason to doubt lottery Rachev.
Already, leverage constraints appear to be a promising explaination given how returns to both the SV and AV managed portfolios converge to the buy and hold as investment is limited. Also there is some reason to doubt the influence of lottery preferences as the Rachev ratio for both the SV and AV manged portfolios are better than for the buy and hold. Each of the managed portfolios offeres higher extreme returns for extreme losses. Indeed, the fundamental argument that investors prefer the buy and hold market over the average variance or volatility managed investment because the market is more lottery-like is unclear.

\citet{bali_maxing_2011} show that the maximum daily return over the past one month, MAX, is a good measure of the lottery-like payoffs of a stock and a significant indicator of lower future returns robust to size, book-to-market, momentum, short-term reversals, liquidity, and skewness. This means that for lottery seeking investors to prefer the buy and hold market its MAX measures must be significantly different from the average variance and volatility managed portfolios. Yet, as seen in table \ref{tab:tab_max_stats} the mean and median values of the highest one day returns, MAX1, and the average of the five highest daily returns within the month, MAX5, are highest for the average variance managed portfolio. Using daily return values scaled by the prior months portfolio volatility, as in \citet{asness_betting_2018}, the volatility managed portfolio is the most lottery like with the highest mean and median values of scaled MAX1 and MAX5. Notably, the average variance managed portfolio still has higher mean SMAX1 and SMAX5 values than the buy and hold portfolio. The buy and hold investment is the most lottery like in no measure. This does not mean that the buy and hold portfolio is not viewed as a lottery and investors do not take some additional utility from holding it, however it seems very unlikely that this is greater than the lottery utility provided by the average variance or volatility managed portfolio let alone large enough to compensate for the difference in return or CER gain.

It remains possible that on some other yet unknown measure the buy and hold strategy is more lottery like. To form a more direct test of the lottery preference explaination I test the effect of lottery preferences on the returns to the AV managed portfolio using measures of the market capitalization of gaming industry firms, MAX5 and MAX5 scaled for the market portfolio. On the assumption that high lottery preferences across the public will manifest in higher market capitalization for gambling and gaming firms, I sum the market capitalization for 27 firms whose main business is gambling, e.g., casinos and horse tracks. Table \ref{tab:tab_gambling} GMCAP is the gaming industry market capitalization calculated monthly. Total gaming capitalization is likely tied closely to total market capitalization which some argue is related to future returns. To seperate the effect of total market capitalization, I calculate GMCAP$_{scaled}$ as the ratio of the market capitalization of gaming firms to the total market capitalization of the 500 largest firms.

Table \ref{tab:tab_lottery} presents the results of running regressions using the proxies for lottery preference levels in the economy. \citet{fama_dissecting_2016} argue that measurement of low-risk anomoly excess returns requires controlling for the known Fama-French risk factors and although that is in a cross-sectional specification I include them here to control for other sources of aggregated mispricing. The GMCAP and GMCAP$_{scaled}$ series start in December 1985.  

The final performance ratio investigated is the Rachev. Developed in \citet{biglova_different_2004}, the Rachev ratio is the expected tail return in the best n\% of the return distribution over the expected tail loss in the worst n\%. I set n = 5 and measure the top 5\% expected return over the expected loss in the worst 5\%. The general Rachev ratio formula is, again against a target return level of 0:
\begin{equation}
\frac {ETL_{\alpha }\left({{r_{f}}-x'r}\right)}{ETL_{\beta}\left({x'r-{r_{f}}}\right)}
\end{equation}
where $ET{L_{\alpha }}={\frac {1}{\alpha }}\int _{0}^{\alpha }{VaR_{q}\left(X\right)dq}$.