We have some reason to suspect leverage already, the constraints drive returns toward BH and some reason to doubt lottery Rachev.

However, the fundamental argument that investors prefer the buy and hold market over the average variance or volatility managed investment because the market is more lottery-like. This itself is unclear.

\citet{bali_maxing_2011} show that the maximum daily return over the past one month, MAX, is a good measure of the lottery-like payoffs of a stock and a significant indicator of lower future returns robust to size, book-to-market, momentum, short-term reversals, liquidity, and skewness. This means that for lottery seeking investors to prefer the buy and hold market its MAX measures must be significantly different from the average variance and volatility managed portfolios. Yet, as seen in table \ref{tab_max} the mean and median values of the highest one day returns, MAX1, and the average of the five highest daily returns within the month, MAX5, are higher for the average variance managed portfolio than either volatility or the buy and hold market portfolio. Using daily return values scaled by the prior months portfolio volatility, as in \citet{asness_betting_2018}, the volatility managed portfolio is the most lottery like with the highest mean and median values of scaled MAX1, SMAX1, and scaled MAX5, SMAX5. Notably, the average variance managed portfolio still has higher mean SMAX1 and SMAX5 values than the buy and hold portfolio. This does not mean that the buy and hold portfolio is not viewed as a lottery and investors do not take some additional utility from holding it, however it seems very unlikely that this is even greater than the lottery utility provided by the average variance or volatility managed portfolio let alone large enough to compensate for the difference in return or CER gain.

It remains possible that on some other yet unknown measure the buy and hold strategy is more lottery like. To form a more direct test of the lottery preference explaination 