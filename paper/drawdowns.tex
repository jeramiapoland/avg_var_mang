Drawdowns, the peak-to-trough decline in the value of a portfolio, may be the most natural measure of real market risk. \citep{magdon-ismail_maximum_2006} A drawdown occurs when a portfolio loses value for two or more consecutive months. The drawdown continues until the portfolio regains the value it had at the beginning of the first month of the drawdown even if the portfolio gains back some value in between. Maximum drawdown, the largest peak-to-trough decline in portfolio value, in particular is often used in place of return variance as a portfolio risk measure. \citep{johansen_large_2000,articlev1,noauthor_sornette_nodate} Drawdowns play a significant role in the lives of fund managers as deep losses not only rob the fund of capital but motivate investors to withdraw funds making drawdowns a significant determinant of fund survival. \citep{baba_hedge_nodate,papaioannou_procyclical_2013,lang_2006} %To compare SV and AV managed portfolios against the market buy and hold, I consider drawdowns longer than one month so two consecutive months of negative returns will start a drawdown.  

\bigskip
\centerline{\bf [Place Table~\ref{tab:tab_drawdowns} about here]}
\bigskip

Table \ref{tab:tab_drawdowns} panel (a) presents the drawdown statistics for the buy and hold, SV managed, and AV managed portfolios. The AV managed portfolio has more discrete drawdown events, 87, than either the buy and hold or SV managed portfolio. However, the drawdowns are much less severe. The buy and hold strategy losses a maximum of 84.8\% of its value at the deepest point of its maximum drawdown, Max DD. AV and SV lose only 60.3\% and 63.6\% at the bottom of their worst drawdowns. The SV managed portfolio has the worst average loss during a drawdown, Avg DD, at 11.2\% of the portfolio value while AV's average loss is only 9\%. SV also stays "underwater" the longest both on average, 15 months, and during its longest drawdown, 246 months. Figure \ref{fig:fig_drawdowns} shows the time series of drawdowns for the buy and hold, SV, and AV managed returns. However, the deepest and longest sustained losses of value for the SV managed portfolio start in the 1960s and SV does not recover until 1989. The notion that there is a drawdown so severe that it causes the collapse of the fund, or at least a management turnover, is known as the "knockout" drawdown. \citep{pav_notes_nodate} Given a knockout drawdown value, it is possible to estimate the likelihood of the knockout occurring, the fund or manager not surviving, by fitting a binomial distribution to the drawdown observations using the knockout drawdown level as a cutoff to create binary values indicating a drawdown exceeding that level, 1, or not, 0. \citep{pav_notes_nodate} Setting the knockout drawdown at 40\%, as suggested by \citet{pav_notes_nodate}, in any given month the SV managed and AV managed portfolios have probabilities of 1.49\% and .85\% of incurring a knockout drawdown in the next 12 months. The SV managed portfolio if far more likely to fail and 75.7\% more expensive, in theory, to insure using the max drawdown insurance of \citet{carr_maximum_2011}.\footnote{Calculation of actual insurance costs require prices on the zero coupon bond, however given this common price the digital call option on the knockout value of the SV managed portfolio is 1.757 times the price of the AV managed portfolio.} This holds if the knockout drawdown is taken at either 35\% or 45\% where SV is 55.2\% and 91.7\% more expensive. Certainly, US equity investors are better served by the AV managed portfolio than the SV. However clear this conclusion, it raises the question of generalization.

\bigskip
\centerline{\bf [Place Figure~\ref{fig:fig_drawdowns} about here]}
\bigskip

%Table \ref{tab:tab_drawdowns} panel (b) shows the drawdown statistics for each portfolio across countries. Unlike for the prior return measures, it is not always the case that SV management is an improvement over the buy and hold strategy. In Japan, India, and the UK, SV managed drawdowns are deeper, longer, and take longer to recover from on average. In every country but Australia, the AV managed portfolio has a shallower average drawdown. The AV managed portfolio has shorter average drawdowns in every country tested. In every country but China, the AV managed portfolio has a shorter average recovery time. Indeed, in every country but Japan, the AV management strategy is able to recover from a drawdown in less than 10 months on average. SV management of the world index makes the average drawdown statistics worse across the board allowing AV management to draw some sharp distinctions. The AV management strategy has an average drawdown depth nearly 30\% shallower than the SV strategy, a 6.982\% versus 9.776\% loss and the average drawdown length for the AV management strategy is more than 20\% shorter than for SV, 9.9 versus 12.5 months.
