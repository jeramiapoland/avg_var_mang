%The effectiveness of either SV or AV as an investment management signal will be driven by their relationship with future risk and return. It is the trade-off which is key to the leverage management strategy. Assuming that investors hold a portfolio with whose risk-return ratio they are indifferent, when risk increases but expected returns do not, the risk-return ratio become more unattractive and any risk-averse investor would like to decrease there position. Conversely, when the risk goes down without a decrease in return there is an opportunity to leverage into the position and return to the previous level of risk but now with a magnified return.

%\citet{pollet_average_2010} argue that average correlation is a better measure of systematic market risk. Increases in average correlation are related to a higher covariance between labor income growth and the stock market, and the stock market and bond returns. Also, they show that in the context of Roll’s Critique, average correlation should be a better indicator of risk for the true market portfolio rather than stock market variance. \citep{roll_critique_1977} The quarterly regressions in \citet{pollet_average_2010} show AC is related to higher returns in the following quarter.

As detailed in section \ref{sec:theory} managing by AV increases investments in periods of higher systematic risk and return covariance across the economy. Investors should benefit from investments not just in the equity market but from other assets as well. AV managed returns should depend systematically on the relationship of the observed index and aggregate wealth and there should be evidence of a dynamic information relationship between AV, AC, and future returns supporting the portfolio performance in section \ref{sec:port_performance}. I should find empirical evidence to support the theoretical foundation in \cite{pollet_average_2010}.