%As \citet{pollet_average_2010} show AC is positively related to the correlation of market returns and aggregate wealth, including the unobserved component of the larger market for which, as \citet{roll_critique_1977} points out, the equity market is only a proxy. 
%The ability of AC to signal changes in systemic risk depends on both the representativeness of the daily returns used to calculate AC for the daily returns of the unobserved market and the correlation of the unobserved market with aggregate wealth. 
%If the daily returns are not a good proxy for market returns and the market is not a significant component of aggregate wealth it is unlikely AC will serve as a signal of systematic risk or changes in the economy. This is related to difference in results shown by \citet{goyal_idiosyncratic_2003} and \citet{bali_does_nodate} when the latter removes a significant number of daily returns and the forecasting ability of idiosyncratic volatility disappears.

The CRSP daily return data covers only returns for assets traded on the New York Stock Exchange (NYSE) prior to 1962. This makes the pre-1962 data very different from the post-1962 data. The earlier data is much shallower having months with fewer than 400 assets total that meet the data requirements. Given the value of the assets traded outside the NYSE, as much as 13\% of traded securities by market capitalization are missing from the CRSP data as of the 1950s. \citep{nyse_history,staff_american_2003} Twice as many firms covering twice as many industries are available in July 1962 versus June. As shown in \citet{taylor_2014} the NYSE market was not a significant part of marginal wealth in the US following the Great Depression and before the late 1950s. And, as documented in \citet{jones_century_2002} the pre-1962 period is significantly and persistently more illiquid. Merrill Lynch an the NYSE began its first monthly investment plan and "Own Your Share" advertising push in 1954 with the goal of linking Wall Street and Main Street; the program included commission discounts and automatic dividend reinvestment and the push more than doubled the number of investors in the US from 1954 to 1968. \citep{noauthor_investment_1964,traflet_own_2003}  Thus, AC calculated from daily returns prior to 1962 is not likely to be a good proxy for systemic risk. SV and AV will either be measures of the same risk or SV may outperform AV. This means regressions on the relationship of AC and future returns in the 1926 to 1962 sub-sample can provide us evidence suggesting that AV management works when systemic risk is better proxied.

%Table \ref{tab:tab_in_sample_full} presents the results for the in-sample regressions across the whole CRSP data set, from 1926 to 2016. Across the whole data set, AV is an even better predictor of next month’s SV than current SV. The panel A results here are even better than in table \ref{tab:tab_in_sample_1962}. The results in panel B show the same. The prior results are not only supported but appear better with AV being the best predictor of next month’s AV and SV providing no help and becoming insignificant with both variables are included. As before, across the whole data set stock market variance is more highly related to next month’s average correlation than is average variance. Panel C in table \ref{tab:tab_in_sample_full} supports panel C in table \ref{tab:tab_in_sample_1962}. Panel D is the first place the results appear different in a meaningful way. There is no relationship between AV and next month’s return in panel D of table \ref{tab:tab_in_sample_full}. The coefficient on AV is insignificantly positive with an adjusted $R^{2}$ of -0.1\%. Over the full sample, SV is a better predictor of next month's log excess return. A one standard deviation increase in this month's stock market variance indicates a .056 standard deviation, .3 percentage points, lower return. However, with an adjusted $R^{2}$ of .2\%, there are better return predictors. While it is no longer obvious AV is better
%than SV at both risk and return anticipation across the whole data set, the results still suggest that investors are likely to be better off using AV than SV as it is a better risk anticipation measure and at least unrelated to future returns. AV should capture more returns for the same level of risk, if
%not avoid negative returns.

Table \ref{tab:tab_in_sample_full} presents return prediction regressions for the 1926 to 1962 CRSP sample, pre-1962 , and the sample after 1962 which encompasses the \citet{pollet_average_2010} sample. Confirming the prior quarterly results, in the post 1962 sample AC is positively and significantly related to next month's return. Both AV and SV are both significantly related to future SV with nearly the same coefficient, .550 and .556, and similar $R^{2}$ values, 29.6\% and 30.3\%. As in the full sample AV is a better predictor than SV for next month's AV. As in the full sample, investors are served as well by AV as SV in timing risk and as AC is related to higher future returns, investor have the opportunity to stay invested in good times using AV management. In contrast, in the prior 1962 period, AC is not significantly related to future returns. The coefficient on AC is actually negative but insignificant. The relationships between AV, SV and future risk remain the same but as suspected AC no longer signals compensated risk. This supports the understanding that AC is a signal of systemic risk when calculated from a proxy that is a large portion of the market and sufficiently related to aggregate wealth.