%from PW: indicates that the positive forecasting power of average variance, shown in Goyal and Santa-Clara, disappears if Nasdaq and Amex stocks are excluded from market returns or if value-weighted average variance or median variance is used in place of equal-weighted average variance.

As \citet{pollet_average_2010} show AC is positively related to the correlation of market returns and aggregate wealth, including the unobserved component of the larger market for which, as \citet{roll_critique_1977} points out, the equity market is only a proxy. The ability of AC to signal changes in systemic risk depends on both the representativeness of the daily returns used to calculate AC for the daily returns of the unobserved market and the correlation of the unobserved market with aggregate wealth. In short, if the daily returns are not a good proxy for market returns and the market is not a significant component of aggregate wealth it is unlikely AC will serve as a signal of systemic risk or changes in the economy. This is related to difference in results shown by \citet{goyal_idiosyncratic_2003} and \citet{bali_does_nodate} when the latter removes a significant number of daily returns and the forecasting ability of idiosyncratic volatility disappears.

The CRSP daily return data contains only returns for assets traded on the New York Stock Exchange (NYSE) prior to 1962. This difference in coverage makes the pre-1962 data very different from the post-1962 data. The earlier data is much shallower having months with fewer than 400 assets total that meet the data requirements. Given the value of the assets traded outside the NYSE, as much as 13\% of traded securities, by market capitalization, are missing from the CRSP data as of the 1950s. \citep{nyse_history,staff_american_2003} Twice as many firms covering twice as many industries are available at the end of 1962 as compared to the end of 1961. As show in \citet{taylor_2014} the NYSE market was not a significant part of marginal wealth in the US following the Great Depression before the late 1950s. And, as documented in \citet{jones_century_2002} the pre-1962 period is significantly and persistently more illiquid. Thus, AC calculated from daily returns prior to 1962 is not likely to be a good proxy for systemic risk and AV management is unlikely to outperform SV. SV and AV will either be measures of the same risk or SV may outperform AV by taking larger values resulting in more extreme investment changes. This means regressions on the relationship of AC and future returns and the portfolio performance of AV and SV in the 1926 to 1962 sub-sample can provide us evidence suggesting that AV management works when systemic risk is better proxied.