%To calculate stock market variance, average asset variance and average asset correlation, I use daily return data from CRSP and calculate the variance of daily returns monthly. 
%%Where t is the number of trading days in the period month the realized stock market variance is
%%\begin{equation}
%%	SV_{t} = \frac{1}{t-1}\sum_{\tau = 1}^{T} \left(R_{m,\tau} - \frac{\sum_{\tau = 1}^{T} R_{m,\tau}}{t}\right)^{2}
%%\end{equation}
%%where $R_{m,t}$ is the return on the CRSP market portfolio at time t.
%To simplify the analysis of individual assets I require that the asset be traded on each in the month. This mitigates any liquidity affects and insures consistent variance, covariance and correlation calculations. This makes the calculation of asset variance:
To calculate stock market variance, average asset variance, and average asset correlation, I use daily return data from CRSP and calculate the variance of daily returns monthly. To simplify the analysis of individual assets, I require that the asset be traded on each in the month which mitigates any liquidity effects and ensures consistent variance, covariance and correlation calculations. These conditions make the calculation of asset variance:
\begin{equation}
	\sigma^{2}_{m,t} = \frac{1}{T-1}\sum_{\tau = 1}^{T} \left(R_{m,\tau} - \frac{\sum_{\tau = 1}^{T} R_{m,\tau}}{T}\right)^{2}.
\end{equation}
where $R_{m,\tau}$ is the daily return, including dividends, on an asset for day $\tau$ in month t. When the asset is the market portfolio, so $R_{m,\tau}$ = $R_{s,\tau}$, the result is the variation of market returns, SV. The standard Pearson’s correlation where the correlation of assets m and n for month t is:
\begin{equation}
	\rho_{m,n,t} = \frac{\sum^{T}_{\tau = 1}\left(R_{m,\tau} - \frac{\sum_{\tau = 1}^{T} R_{m,\tau}}{t}\right)\left(R_{n,\tau} - \frac{\sum_{\tau = 1}^{T} R_{n,\tau}}{t}\right)}{\sqrt{\left(R_{m,\tau} - \frac{\sum_{\tau = 1}^{T} R_{m,\tau}}{t}\right)^{2}\sum_{\tau=1}^{T}\left(R_{n,\tau} - \frac{\sum_{\tau = 1}^{T} R_{n,\tau}}{t}\right)^{2} }}.
\end{equation}
Unfortunately, for samples as small as the monthly series of daily returns Pearson’s correlation is not an unbiased estimator of the true correlation, even if the returns are normal. \cite{hotelling_1953} The average month in my sample has 22 trading days however the number commonly drops into the teens.\footnote{The shortest trading month in the sample is September 2001 with 15 trading days while 17 is a common number in the months with holidays.} For samples of these sizes, the bias causes an underestimation of the correlation which is worse the lower the true correlation. I employ an approximate correction from \cite{olkin_1958} such that the monthly correlation between two assets m and n is:
\begin{equation}
	\rho_{m,n,t} = \widehat{\rho_{m,n,t}}\left(1 + \frac{1+\widehat{\rho^{m,n}}^{2}}{2(t-3)}\right)
\end{equation}
where $\widehat{\rho_{m,n,t}}$ is the Pearson correlation between a and b.\footnote{The exact correction suggested in \cite{olkin_1958} is too computationally taxing for the equiptment to which I have access.} Average variance and average correlation are value-weighted so each month I calculate market capitalization for all of the assets available in CRSP. The capitalization used in month t for asset m is the product of the end of month price (PRC) and common shares outstanding (SHROUT) values for asset m in month t-1. 
\begin{equation}
MCAP_{m,t} = PRC_{m,t-1}\times SHROUT_{m,t-1}
\end{equation}
To make the analysis more computationally trackable I use only, at most, the top 500 assets in CRSP by market capitalization for a given month.\footnote{The least number of assets which trade every day in a given month is 392 in August of 1932. There are regularly 500 qualifying assets by the end of the 1930s.} Given this restriction, an assets market capitalization weight is defined by:
\begin{equation}
w_{m,t} = \frac{MCAP_{m,t}}{\sum_{j=1}^{J}MCAP_{n,t}}
\end{equation}
with j $\leq$ 500. Thus, the other series of interest, market variance, SV, average variance, AV, and average correlation, AC, are defined by:
\begin{align}
	SV_{t} &= \frac{1}{t-1}\sum_{\tau = 1}^{T} \left(R_{s,\tau} - \frac{\sum_{\tau = 1}^{T} R_{s,\tau}}{t}\right)^{2}\\
	AV_{t} &= \sum_{m=1}^{M} w_{m,t}\sigma^{2}_{m,t}\\
	AC_{t} &= \sum_{m=1}^{M}\sum_{m \neq n}^{N}w_{m,t}w_{n,t}\rho_{m,n,t}
\end{align}
Figure \ref{fig:time_series} shows the time series behavior of market and average variance, in percent, as well as
average correlation. With the easily noticeable exception of October 1987, spikes in both market and average variance concentrate around NBER defined recessions.\\
\bigskip
\centerline{\bf [Place Figure~\ref{fig:time_series} about here]}
\bigskip
Table \ref{tab_summary_stats} shows the summary statistics for the calculated variables. Despite the use of the actual number of trading days, the restriction to assets that trade every trading day, and the adjustment to the calculation of correlation, the quarterly calculated values are almost identical to those in \citet{pollet_average_2010} over the same sample. Over the expanded the period, the annualized monthly average variance has a mean value of .88\%. The annualized stock market variance mean is much lower at .25\% monthly. Average correlation is relatively consistent at .23 quarterly in the \cite{pollet_average_2010} sample, .261 monthly in the same sample and .276 over the full time period. Average variance is more volatile than the stock market variance, more than twice as much. In each sample average variance has the highest autocorrelation. While average correlation is also persistent, the stock market variance is only strongly persistent at the monthly frequency with autocorrelation of .61. All three time series are stationary rejecting the unit root null in the tests of \citet{dickey_distribution_1979}, \citet{Ng2001}, and \citet{ers1996}.\\
\bigskip
\centerline{\bf [Place Table~\ref{tab_summary_stats} about here]}
\bigskip
As my primary interest is in the use of average variance versus market variance in the management of leverage in the CRSP market portfolio, I test AV and SV against CRSP log excess returns.
Specifically, I take the difference between the natural log of one plus the CRSP market return and the natural log of one plus the risk-free rate using:
\begin{equation}
	r_{t} = \log R_{m,t} - \log R_{f,t} 
\end{equation}
where $R_{f,t}$ the risk-free return for which I use the 1-month treasury bill rate from Ken French's website\footnote{http://mba.tuck.dartmouth.edu/pages/faculty/ken.french/data\_library.html\#Research}. As shown in table \ref{tab_correlations}, over the full data period, each variance and correlation time series are contemporaneously correlated to lower log excess returns. Average variance, AV, is significantly correlated with next month's market variance AV, SV, and AC. Surprisingly, over the full data set, this month's AV is even nominally more correlated with next month's SV than this month's SV is, 0.625 versus .612. Over the basis period, AV is time series most negatively correlated with next months log excess return at -.129, while it is entirely unrelated to next month's return over the whole data set.

As in the prior literature, July 1963 serves as the start of the basis data period for the regression analysis.\footnote{CRSP has, as of 2005, backfilled NYSE daily returns to 1926, however the pre-1962 data is very different from the post-1962 data. The earlier data is much shallower having months with fewer than 400 assets total that meet the data requirements. Twice as many firms covering twice as many industries are available at the end of 1962 as compared to the end of 1961. And, as documented in \citet{jones_century_2002} the pre-1962 period is significantly and persistantly more illiquid.} For in and out-of-sample tests, I regress market and average variance, and average correlation against these excess log return values. Out-of-sample regressions require an in-sample training period which, I set this at 15\% of the available time series for consistent calculation of robust out of sample statistics later in the analysis. This training window means that out of sample regressions, analysis begins at the end of July 1970 in the basis sample and December 1939 in the
full sample.

Proxies for lottery preference and leverage constraints are needed to test which has a measured influence on the relationship of buy and hold market returns and volatility, or average variance managed portfolio returns. \citet{bali_maxing_2011} define MAX as the maximum daily return for an asset in a given month. High values for MAX are shown to be good indicators for future conditions associated with higher lottery preference behavior. High values of MAX are a significant indicator of lower future returns. I calculate $MAX1$ and $MAX5$ as the highest daily return and the average of the highest five daily returns for each portfolio in a given month. As another indicator of higher lottery preferences in the economy, I capture the market capitalization for all gaming stocks in CRSP. $GMCAP$ is the total market capitalization of all assets with SIC code 7999 in the CRSP dataset. Gaming industry firms should do better in times of high lottery preference driving higher levels of $GMCAP$. However, the total valuation may move with the overall market capitalization. To generate a measure independent of the size of the market, I calculate $GMCAP_{scaled}$ as the ratio of gaming market capitalization to total market capitalization.

To generate proxies for the lending constraints operating in the market, I look to measures of financial intermediary leverage, investors leverage, borrowing levels and interest rates. I use data
on broker call money\footnote{Call money is the money loaned by a bank or other institution which is repayable on demand.} rates from 1988 to 2014 from Bloomberg, $Broker_{Call}$, average bank call money rates for 1984-2005 from Datastream, $Bank_{Call}$, and bank prime lending rates for 1984-2014 from Datastream, $Bank_{Prime}$. I use a seasonally adjusted version year-on-year nominal bank credit growth, $Credit_{CHG}$, as another proxy of borrowing conditions from 1984 to 2014, similar to the variable of \citet{Gandhi2016}.\footnote{Seaonally adjusted nominal monthly bank credit is available in statistical release H.8 (Assets and Liabilities of Commercial Banks in the U.S.) of the Board of Governors of the Federal Reserve System.} Typically investors
use margin, borrowed from brokers, to take leverage investment positions. As a proxy for the capital available to financial intermediaries, like brokers, I use the intermediary leverage factor, $LF_{AEM}$, of \citet{adrian2014financial}. $LF_{AEM}$ measures shocks to a seasonally-adjusted broker-dealer leverage ratio, the ratio of total broker-dealer financial assets and that value minus debt.
\begin{align}
	LF_{AEM} &= \Delta Ln(\frac{FinAsst}{FinAsst - BankDbt})
\end{align}
This variable is available from Asaf Manela's website\footnote{http://apps.olin.wustl.edu/faculty/manela/}. The total amount of margin borrowing is available from the NYSE website.\footnote{http://www.nyxdata.com/nysedata/asp/factbook/viewer\_edition.asp?mode=table\&key=3153\&category=8} I calculate $MD_{1984}$ as the change in the level of margin debt, month to month, in real 1984 dollars.

I use the proxies for lottery preferences and lending constraints in in-sample regressions along with the Fama-French five-factors, which are taken from Ken French's website. In contrast, the fundamental relationship between AV and future returns is established through in-sample regressions needing only the time series of average variance and log excess monthly returns.