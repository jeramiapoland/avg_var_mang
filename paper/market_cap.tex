The importance of the relationship of the stock market and aggregate wealth to the performance of the AV managed strategy should show up in returns systematically. As the sub-sample results above imply where the market proxy is less representative or significant to investor wealth AV management will under-perform relative to where the proxy is better. As GDP is related to aggregate income, demand, and the consumption of aggregate wealth. It often appears as a proxy in cross country studies of wealth and income effects. \citep{barro_cross-country_1989,levine_what_1993,baird_aggregate_2010} Thus, market capitalization to GDP will serve as a useful, though imperfect, a proxy for
the proportion of aggregate wealth represented by the stock market. Starting in January 2005, I sort countries annually based on the market capitalization to GDP ratio observed in the prior December. If the relationship of the proxy returns and aggregate wealth is systematically important to AV returns in the direction expected, the AV managed strategy should do better in countries with above median market to GDP ratio. Thus, a strategy long the high ratio countries and short the low ratio countries should have a positive and significant alpha. 

The long/short AV strategy performs just as expected. Investors capture average annualized
returns over 11\% with a Sharpe ratio above .77. Moreover,
the portfolio has a positive and significant Fama-French five-factor and five-factor plus Momentum
alphas meaning the relationship of the stock market to aggregate wealth is systematically important
 to AV management as seen in the pre-1962 results above. This does not hold for SV management
which has insignificant alphas, lower returns and a smaller Sharpe ratio.