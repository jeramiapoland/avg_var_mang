The importance of the relationship of the stock market and aggregate wealth to the performance of the AV managed strategy should show up in the AV managed returns systematically. By the arguments made in \citet{pollet_average_2010} AV management will under-perform where the stock market return is a smaller portion and less representative of returns to aggregate wealth.

GDP per capita  often appears as a control in cross country studies of wealth and income effects. \citep{barro_cross-country_1989,levine_what_1993,baird_aggregate_2010} Thus, the market capitalization to GDP per capita ratio should serve as a useful, though imperfect, proxy for the proportion of aggregate wealth represented by the stock market. From January 1975 to December 2015, I use periods when the ratio is above the median against when it is below to test the importance of the ratio in the time series of AV managed US equity returns. As seen in table \ref{tab:tab_sub_performance} panel (a), AV management produces positive returns in both sub-samples. However, returns and the Sharpe ratio are better when the ratio is higher as expected from equation (6) from \citet{pollet_average_2010}. Moreover, high ratio periods have positive and significant Fama-French three, five-factor, and five-factor plus Momentum alphas meaning the relationship of the stock market to aggregate wealth is systematically important as expected from equation (8) from \citet{pollet_average_2010}. %These results support the regression results above.

To test this hypothesis across countries, I use Credit Suisse's annual reports on global wealth.\footnote{The reports are available for 2011 through 2017, covering data from 2000 to 2017, at https://www.credit-suisse.com/corporate/en/research/research-institute/global-wealth-report.html} This should be a better proxy than market capitalization to GDP per capita, however this will still not be perfect as it depends on how well Credit Suisse's measure of wealth captures all of aggregate wealth. Each year, starting in 2005, I rank countries on the ratio of annual market return to return on annual wealth. I form portfolios long in the above median ratio countries, short in the below and a long minus short using both. For robustness, I also use market capitalization to GDP per capita. If the relationship of the index returns and aggregate wealth is systematically important to AV returns in the direction expected, the AV managed strategy should do better in countries with above median market to GDP ratio. Thus, a strategy long the high ratio countries and short the low ratio countries should have a positive and significant alpha. 

As seen in table \ref{tab:tab_sub_performance} panel (b), AV management produces positive returns across all countries so the performance of the long-short strategy will depend on the long side producing significantly better performance than the short. The long-short AV strategy performs just as expected. Using the Credit Suisse wealth numbers, investors capture average annualized returns over 12.6\% with a Sharpe ratio above 0.74 on the long side. %Unfortunately in this case, performance is also strong on the short side where investors earn 7.5\%. 
The long-short portfolio nets investors better than 5\% annualized return with a Sharpe ratio better than the US buy and hold equity return. Moreover, the portfolio has positive and significant Fama-French three, five, and five-factor plus Momentum alphas meaning the relationship of the stock market to aggregate wealth is systematically important to AV management as expected from the arguments in \citet{pollet_average_2010}. These results hold for the market capitalization to GDP per capita sorting strategy as well. The risk-adjusted performance of the strategy depends, robustly, on the quality of the signal of underlying relationship as theoretically motivated.

%The full in-sample regression monthly results support the conclusions reached by \citet{pollet_average_2010} at the quarterly frequency. Using the intuition in the argument for AC as a signal of systematic risk which depends on the relationship of the market index and aggregate wealth, I demonstrated a placebo-like sub-sample with an expected lack of results, a difference in US equity performance, and a systematic difference in AV managed returns across countries. Each of these results suggest AV management works by allowing investors to time changes in the mix of systematic risk in market index variance. However, in-sample relationships are not sufficient to know that investors have access to the dynamics of this relationship in real-time. It is well documented that many in-sample strategies do not work out-of-sample.