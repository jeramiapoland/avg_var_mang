To better understand the channels through which $MC$ operates, we use forecast encompassing tests to compare the information content of $MC$ to that of other predictors. Forecast encompassing tests come from the literature on optimal
forecast combination \citep{Chong1986,Fair1990}. An optimal
forecast as a convex combination of two forecasts for month $t+1$
is
\begin{equation} \label{eq:forecast_comb}
\hat r^{*}_{t+1} = (1-\lambda)\hat r_{1,t+1} + \lambda \hat r_{2,t+1},
\end{equation}
where $\hat r_{1,t+1}$ is the forecast based on the first
variable, $\hat r_{2,t+1}$ is the forecast based on the second
variable, and  $\lambda$ such that $0 \leq \lambda \leq 1$ is
chosen to minimize $MSFE$ of $\hat r^{*}_{t+1}$. $\lambda=0$
suggests that the forecast $\hat r_{1,t+1}$ encompasses $\hat
r_{2,t+1}$. In other words, the second variable does not have any
information relevant to predict excess market returns beyond the
information contained in the first variable. However, if $ \lambda
> 0$, it suggests that that the forecast $\hat r^{*}_{t+1}$ does not
encompasses $\hat r_{2,t+1}$ and both variables 1 and 2 have
information useful to predict excess returns. We test the null
hypothesis that $H_{0}: \lambda = 0$ against the alternative that
it is greater than zero $H_{A}: \lambda > 0$. The statistical
significance is based on the \cite{Harvey1998} statistic.

Table~\ref{tab:lambda} shows the $\lambda$'s, the weight on $MD$,
$SII$, $S^{PLS}$, and one of the margin capacity variables in
combination with each other and other predictors, for monthly
horizon, $H=1$. Predictor 1, generating $\hat{r_{1,t+1}}$
with weight $1 - \lambda$ is listed in column 1. Predictor 2,
generating $\hat{r_{2,t+1}}$ with weight $\lambda$ are listed as
headings of the remaining columns. We find that the column under $MC$ has large
positive and statistically significant $\lambda$'s with values of
either 1 or very close to 1. In other words, $MC$ encompasses the
predictions based on all other variables. These include $DP$,
$MD$, $MCAP/GDP$, $SII$, $S^{PLS}$, and even other margin capacity
variables. Focusing on the last row, we find that none of the
predictions based on other variables have $\lambda$'s
significantly different from 0. $MC_{REAL}$, $MC_{NOM}$, and
$MC_{MCAP}$, while not being as informative as $MC$, still
encompass the information in all other non-margin capacity
variables. In unreported results, we find similar evidence for longer horizon
predictions. Thus, none of the other variables seem to provide
additional information over the information already contained in margin capacity.

Results in Sections \ref{sec:robustness} and
\ref{sec:asset_allocation} indicate that while $MC$ is a good
predictor both in expansions and recessions, it does better during
recessions. Table~\ref{tab:lambda_sub} investigates incremental
information in $MC$ during different subsamples. During contractions, no other predictor provides any information additional to $MC$ and during expansions $MC$ always has significant information over and above any other predictor.
For the optimal forecast combination, the weight on $MC$-based forecast is always
1 against all other forecasts during contractions and
the weight on any other predictor is insignificantly different from zero in expansions. We conclude from these tests that $MC$ encompasses the information in other predictors for return forecasts, but not vice-versa.