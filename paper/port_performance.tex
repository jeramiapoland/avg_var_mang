Before examining the constrained portfolio performance, I present the results for the SV and AV strategies targeting the buy and hold volatility without constraints in table \ref{tab:tab_performance1}. As in \citet{moreira_volatility-managed_2017} the portfolio performance is measured across the whole CRSP data set, however the relative performance is the same or better across the basis data set. 

\bigskip
\centerline{\bf [Place Table~\ref{tab:tab_performance1} about here]}
\bigskip

Table \ref{tab:tab_performance1} presents the performance ratios for the SV and AV managed portfolios targeting the buy and hold volatility without investment constraints. The buy and hold market strategy is included for reference. However since \citet{moreira_volatility-managed_2017} establish that the SV managed portfolio out performs the buy and hold, statistical significance results are only presented for the comparison of the SV and AV managed portfolios. The AV managed portfolio generates a statistically significant 1.09 percentage points higher average annualized log excess return. As shown in the bottom panel of figure \ref{fig:fig_returns} the AV strategy builds its performance advantage slowly but consistently starting from the early 1950s and from that time the SV managed portfolio is never a better investment. As both strategies are targeting the same volatility, the significant difference in return translates into a significant difference in Sharpe ratio. At .520 versus .462, for the SV managed portfolio, the AV managed Sharpe ratio is 12.6\% higher. There maybe some concern with the use of mean-variance symmetric performance measures like the Sharpe ratio, however AV management outperforms SV on asymmetric performance ratios as well. AV has a higher Sortino ratio. AV also generates significantly higher Kappa$_{3}$ and Kappa$_{4}$ ratios. The Fama-French three factor alpha is significantly higher for the AV versus SV managed portfolio. There is no difference in significance either way when the momentum factor is also included, but this is due to one data point as detailed below. So while the overall payment for downside risk, measured by Sortino ratio, may not be significantly higher, the payment for downside skewness and extreme downside return is. Measuring risk-adjusted returns using either the Fama-French five factor or five-factor with momentum model, AV management significantly outperforms AV. The differences are slightly higher than the difference in annualized average return.%Unfortunately the compensation for downside tail risk is not perfect. SV has a significantly higher Rachev ratio. The volatility managed market portfolio has higher expected tail return potential for each dollar of potential tail loss to the average variance managed portfolio. The maximum annualized return to the SV managed portfolio is 35.3\%, in August 1965. That month, the AV managed portfolio only returns 8.9\% and its maximum return is an annualized 22.11\%. The AV managed portfolio is a ticket with better winning chances and a higher expected return while the SV managed portfolio has a larger jackpot.

\bigskip
\centerline{\bf [Place Figure~\ref{fig:fig_returns} about here]}
\bigskip

Portfolio performance numbers always bring forward questions on performance in sub-samples. While divisions of the sample by date are largely arbitrary and do not automatically convey the importance of the sub-sample, divisions along business cycles call out specific periods of investor sensitivity. Panels (b) and (c) in table \ref{tab:tab_performance1} present the performance of the buy and hold, SV and AV managed portfolios. The AV managed portfolio is a significantly better performer across all measures. In business cycle expansion, the AV managed portfolio provides significantly more compensation and significantly more compensation for every measure of risk. The results for NBER contractions go in the other direction. The SV managed portfolio appears to be so much better that it might be a more desirable option given that we cannot know periods of extended contractions before they begin and investors may desire a portfolio that protects value through downturns more than one that maximizes returns during market upswings. However, as panel (d) shows, the better performance of the SV managed portfolio is due to one, albeit a rather important, data point. The significantly better performance of the SV managed portfolio through NBER contractions depends entirely on the 1929 to 1933 Great Depression. Excluding that time period, The AV managed portfolio, again, provides higher average and risk-adjusted returns but the difference in performance by any portfolio risk measure, while better for AV, is insignificant. As a result, its clear that without the 1929 to 1933 depression, the SV managed portfolio is unable to compete with AV management in measures of risk-return performance. Additionally, investors that are concerned about the loss of portfolio value, regardless of when it occurs, will value the AV managed portfolio more than the SV managed portfolio in drawdown and utility terms. Certainly, US equity investors are better served by the AV managed portfolio than the SV. However clear this conclusion, it raises the questions of generalization. 
%Does the AV manged strategy work in equity markets outside of the US and does it still outperform the SV managed strategy? Testing the performance of AV management outside of US equities is even more important as confirmation of the argument that AV management works because returns to the AV managed portfolio depend more on AC and thus are more aligned with systematic risk. If AV management aligns better with systematic risk it should perform across assets.
%To test the performance of AV management across international equity markets, I collect daily returns for the Australian (AUS), Brazilian (BRA), Chinese (CHN), German (DEU), French (FRA), Indian (IND), Italian (ITA), Japanese (JPN), and English (UK) markets from Compustat - Capital IQ. Table \ref{tab:tab_int_summary} shows the names of the indices used, the data time frames and the number of assets used for the calculation of market capitalization weights, AV and AC. In all cases the data series are much shorter. The primary limitations are the availability of daily returns and dividend data. To test AV and SV management against a globablly diversified portfolio, I use the MSCI All Countries World Index (ACWI). To construct AV and SV leverage timing signals for the world index, I use a market capitalization weighted average of the country values. For each month the weight of the country is the index market capitalization in US dollard divided by the total market capitalization of the 10 country market indices, including the US.
%
%\bigskip
%\centerline{\bf [Place Table~\ref{tab:tab_int_summary} about here]}
%\bigskip

The AV management strategy is a better performer than SV across the world. Table \ref{tab:tab_intPerf1} shows the differences in performance and the AV managed portfolio generates higher annualized average returns and better Sharpe ratios in all countries except Italy. In the fastest growing markets, China and India, the AV managed portfolio increases annualized average return by 2.455\% and 2.637\%. Both AV and SV management are improvements over the buy and hold in all markets. Australia shows the best buy and hold Sharpe ratio at .614, but the AV management strategy is able to increase it to .981, a very attractive result for any investor. The results for the AV management strategy for the Chinese market are the most alluring, a 27.381\% annualized return with a .868 Sharpe ratio, but this also the shortest and the most volatile of the international return series. Both AV and SV management generate better returns and Sharpe ratios for the globally diversified world index, but again AV management is an improvement over SV.

%Drawdowns, the peak-to-trough decline in the value of a portfolio, may be the most natural measure of real market risk. \citep{magdon-ismail_maximum_2006} Maximum drawdown, the largest peak-to-trough decline in portfolio value, in particular is often used in place of return variance as a portfolio risk measure. \citep{johansen_large_2000,articlev1,articlev2,noauthor_sornette_nodate} Drawdowns play a significant role in the lives of fund managers as deep losses not only rob the fund of capital but motivate investors to withdraw funds making drawdowns a significant determinant of fund survival. \citep{baba_hedge_nodate,papaioannou_procyclical_2013,lang_2006} To compare SV and AV managed portfolios against the market buy and hold, I consider drawdowns longer than one month so two consecutive months of negative returns will start a drawdown. The drawdown continues until the portfolio regains the value it had at the beginning of the first month of the drawdown. 
%
%\bigskip
%\centerline{\bf [Place Table~\ref{tab:tab_drawdowns} about here]}
%\bigskip
%
%Table \ref{tab:tab_drawdowns} panel (a) presents the drawdown statistics for the buy and hold, SV managed, and AV managed portfolios. The AV managed portfolio has more discrete drawdown events, 87, than either the buy and hold or SV managed portfolio. However, the drawdowns are much less severe. The buy and hold strategy losses a maximum of 84.8\% of its value at the deepest point of its maximum drawdown, Max DD. AV and SV lose only 60.3\% and 63.6\% at the bottom of their worst drawdowns. The SV managed portfolio has the worst average loss during a drawdown, Avg DD, at 11.2\% of the portfolio value while AV's average loss is only 9\%. SV also stays "underwater" the longest both on average, 15 months, and during its longest drawdown, 246 months. From figure \ref{fig:fig_drawdowns}, the deepest losses for AV and the buy and hold occur during and after the Great Depression. However, the deepest and longest sustained losses of value for the SV managed portfolio start in the 1960s and SV does not recover until 1989. The notion that there is a drawdown so severe that it causes the collapse of the fund, or at least a management turnover, is known as the "knockout" drawdown. \citep{pav_notes_nodate} Given a knockout drawdown value, it is possible to estimate the likelihood of the knockout occurring, the fund or manager not surviving, by fitting a binomial distribution to the drawdown observations using the knockout drawdown level as a cutoff to create binary values indicating a drawdown exceeding that level, 1, or not, 0. \citep{pav_notes_nodate} Setting the knockout drawdown at 45\%, a loss of nearly half the current value, in any given month the SV managed and AV managed portfolios have probabilities of 1.06\% and .55\% of incurring a knockout drawdown in the next 12 months. The AV managed portfolio is far less risky in these terms as the SV managed portfolio would be nearly twice as likely to fail and 91.7\% more expensive, in theory, to insure using the max drawdown insurance of \citet{carr_maximum_2011}.\footnote{Calculation of actual insurance costs require prices on the zero coupon bond, however given this common price the digital call option on the knockout value of the SV managed portfolio is 1.9166 times the price of the AV managed portfolio.}
%
%\bigskip
%\centerline{\bf [Place Figure~\ref{fig:fig_drawdowns} about here]}
%\bigskip
%
%Table \ref{tab:tab_drawdowns} panel (b) shows the drawdown statistics for each portfolio across countries. Unlike for the prior return measures, it is not always the case that SV management is an improvement over the buy and hold strategy. In Japan, India, and the UK, SV managed drawdowns are deeper, longer, and take longer to recover from on average. In every country but Australia, the AV managed portfolio has a shallower average drawdown. The AV managed portfolio has shorter average drawdowns in every country tested. In every country but China, the AV managed portfolio has a shorter average recovery time. Indeed, in every country but Japan, the AV management strategy is able to recover from a drawdown in less than 10 months on average. SV management of the world index makes the average drawdown statistics worse across the board allowing AV management to draw some sharp distinctions. The AV management strategy has an average drawdown depth nearly 30\% shallower than the SV strategy, a 6.982\% versus 9.776\% loss and the average drawdown length for the AV management strategy is more than 20\% shorter than for SV, 9.9 versus 12.5 months.


Table \ref{tab:tab_costs} panel (b) highlights the differences in the trading costs of the strategies across country. This table differs from \ref{tab:tab_costs} panel (a) in that the break even trading costs are calculated in reference to reducing the annualized average returns of the AV and SV managed portfolios to the buy and hold and not reducing portfolio alphas to zero. The AV managed portfolio is able to tolerate higher trading costs in every country analyzed. In a majority of countries, 6 out of 9, the AV managed portfolio is able to tolerate trading costs twice as high as the SV managed portfolio. As for the individual country indices, the AV managed world index is far cheaper than the SV managed portfolio allowing an investor to keep more of the higher returns generated.
%
%\bigskip
%\centerline{\bf [Place Table~\ref{tab:tab_intPerf3} about here]}
%\bigskip

Across the globe, the AV managed portfolio is a better leverage management signal. It results in higher annualized average returns. It captures in better Sharpe ratios. It generates better drawdown statistics and is cheaper, in trading costs, to execute. Thus, AV management is the better equity investment strategy around the globe. However, if AV management better aligns returns with systematic risk it may be a better management strategy in more than just equities.