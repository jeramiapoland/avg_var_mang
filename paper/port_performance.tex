Before exaiming the constrained portfolio performance, I present the results for SV and AV strategies targeting the buy and hold volatility without investment constraints in table \ref{tab:tab_performance1}. As in \citet{moreira_volatility-managed_2017} the portfolio perforamnce is measured across the whole CRSP data set, however the relative performance is the same or better across the basis data set. 

\bigskip
\centerline{\bf [Place Table~\ref{tab:tab_performance1} about here]}
\bigskip

Table \ref{tab:tab_performance1} presents the performance ratios for the SV and AV managed portfolios targeting the buy and hold volatility without investment constraints. The buy and hold market strategy is included for reference. However since \citet{moreira_volatility-managed_2017} establish that the SV managed portfolio out performs the buy and hold, statistical significance results are only presented for the comparison of the SV and AV managed portfolios. The AV managed portfolio generates a statistically significant 1.08 percentage points higher average annualized log excess return. As shown in the bottom panel of figure \ref{fig:returns} the AV strategy builds its performance advantage slowly but consistently starting from the early 1950s and from that time the SV managed portfolio is never a better investment. As both strategies are targeting the same volatility, the significant difference in return translates into a significant difference in Sharpe ratio. At .520 versus .462, for the SV managed portfolio, the AV managed Sharpe ratio is 12.6\% higher. AV also generates significantly higher Kappa$_{3}$ and Kappa$_{4}$ ratios. So while the overall payment for downside risk, measured by Sortino ratio, may not be significantly higher, the payment for downside skewness and extreme downside return is higher. Unfortunately the compensation for downside tail risk is not perfect. SV has a significantly higher Rachev ratio. The volatility managed market portfolio has higher expected tail return potential for each dollar of potential tail loss to the average variance managed portfolio. The maximum annualized return to the SV managed portfolio is 35.3\%, in August 1965. That month, the AV managed portfolio only returns 8.9\% and its maximum return is an annualized 22.11\%. The AV managed portfolio is a ticket with better winning chances and a higher expected return while the SV managed portfolio has a larger jackpot.
\bigskip
\centerline{\bf [Place Table~\ref{tab:tab_performance2} about here]}
\bigskip

More practical analysis of portfolio performance requires the incorporation of limits on the level of investment taken in the market portfolio. Leverage of 50\%, a coefficient of 1.5 on the market, is a common constraint meant to mimic real market leverage constraints for the average investor based in part on the Reg. T margin requirement\footnote{Federal Reserve Board Regulation T (Reg T) establishes a baseline requirement that investors deposit 50\% of an investment position in their margin trading accounts, however a brokerage house may set a higher equity requirement.}. \citep{Rapach2016,moreira_volatility-managed_2017,deuskar_margin_2017} There are at least two exchange traded funds, ETFs, which three times the return of the SP500.\footnote{The Direxion Daily S\&P 500 Bull 3x Shares ETF, symbol SPXL, and ProShares Ultra Pro S\&P 500 ETF, symbol UPRO, are two such funds.} So, I take a market coefficient of three as the maximum feasible investment a typical investor can make in the market portfolio. Table \ref{tab:table_performance2} panel c presents the results from applying investment constraints after calculating the weights for AV and SV targeting the buy and hold volatility. While both portfolios still outperform the buy and hold, The seperations in average annualized excess return, 1.71\% and 2.07\%, are even greater when investment constraints are applied. Panels a and b in figure \ref{fig:returns} show the effects of the growing separation. While SV is barely able to clear the buy and hold strategy under typical brokerage constraints, returns to the AV managed portfolio remain clearly above. Investors that use a leveraged SP500 ETF to impliment the AV managed portfolio strategy are rewarded with returns significantly higher than the SV managed portfolio suggested weights. The investment restrictions pull the volatility of the AV managed portfolio too far from the SV returns to generate significant differences in performance ratios. However, investors using the leveraged ETFs are rewarded not only with higher returns but significantly better performance ratios across the board with the exception of the Rachev ratio which is at least no longer signficantly better for the SV managed portfolio. The ETF leverage constrained AV strategy even generates better Sharpe and Sortino ratios than the unconstrained strategy. The results in panels a and b demonstrate that better performance of AV is not a result of or contingent on looking to the volatility of the buy and hold strategy. As the targeted volatility is lowered the difference in performance between AV and SV becomes more significant. With the exception of the Rachev ratio, every performance measure is significantly better for the AV managed portfolio given lower volatility targets and investment constraints.

\bigskip
\centerline{\bf [Place Figure~\ref{fig:fig_returns} about here]}
\bigskip

Drawdowns, the peak-to-trough decline in value of a portfolio, may be the most natural measure of real market risk. \citep{magdon-ismail_maximum_2006} Maximum drawdown, the largest peak-to-trough decline in portfolio value, in particular is often used in place of return variance as a portfolio risk measure. \citep{johansen_large_2000,articlev1,articlev2,noauthor_sornette_nodate} Drawdowns play a significant role in the lives of fund managers as deep losses not only rob the fund of capital but motivate investors to withdraw funds making drawdowns a significant determinant of fund survival. \citep{baba_hedge_nodate,papaioannou_procyclical_2013,lang_2006} To compare SV and AV managed portfolios against the market buy and hold, I consider drawdowns longer than one month so two consecutive months of negative returns will start a drawdown. The drawdown continues until the portfolio regains the value it had at the beginning of the first month of the drawdown. 

\bigskip
\centerline{\bf [Place Table~\ref{tab:tab_drawdowns} about here]}
\bigskip

Table \ref{tab:tab_drawdowns} presents the drawdown statistics for the buy and hold, SV managed, and AV managed portfolios. The AV managed portfolio has more discrete drawdown events, 87, than either the buy and hold or SV managed portfolio. However, the drawdowns are much less severe. The buy and hold strategy losses a maximum of 84.8\% of its value at the deepest point of its maximum drawdown, Max DD. AV and SV lose only 60.3\% and 63.6\% at the bottom of their worst drawdowns. The SV managed portfolio has the worst average loss during a drawdown, Avg DD, at 11.2\% of the portfolio value while AV's average loss is only 9\%. SV also stays "underwater" the longest both on average, 15 months, and during its longest drawdown, 246 months. From figure \ref{fig:drawdowns}, the deepest losses for AV and the buy and hold occur during and after the Great Depression. However, the deepest and longest sustained losses of value for the SV managed portfolio start in the 1960s and SV does not recover until 1989. The notion that there is a drawdown so severe that it causes the collapse of the fund, or at least a management turnover, is known as the "knockout" drawdown. Given a knockout drawdown value, it is possible to estimate the likelihood of the knockout occuring, the fund or manager not surviving, by fitting a binomial distribution to the drawdown observations using the knockout drawdown level as a cutoff to create binary values indicating a drawdown exceeding that level, 1, or not, 0. \citep{pav_notes_nodate} Setting the knockout drawdown at 45\%, a loss of nearly half the current value, in any given month the SV managed and AV managed portfolios have probabilities of 1.06\% and .55\% of incurring a knockout drawdown in the next 12 months. The AV managed portfolio is far less risky in these terms as the SV managed portfolio would be nearly twice as likely to fail and 91.7\% more expensive, in theory, to insure using \citet{carr_maximum_2011}'s simplest insurance option.\footnote{Calculation of actual insruance costs require prices on the zero coupon bond, however given this common price the digital call option on the knockout value of the SV managed portfolio is 1.9166 times the price of the AV managed portfolio.}

\bigskip
\centerline{\bf [Place Figure~\ref{fig:fig_drawdowns} about here]}
\bigskip