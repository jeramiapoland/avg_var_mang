\documentclass[xcolor=table]{beamer}
\usetheme{Hannover}
\usecolortheme{seahorse}
\setbeamertemplate{caption}{\insertcaption}
\usepackage{graphics,graphicx,lmodern,amsmath,appendixnumberbeamer,enumerate}
\useoutertheme{infolines}
\setbeamertemplate{headline}{} % removes the headline that infolines inserts
\setbeamertemplate{footline}[frame number]
\usepackage{multimedia,etoolbox,xmpmulti,adjustbox,subfig,booktabs}
%\usepackage[table]{xcolor}
\usepackage{tikz}
\usetikzlibrary{snakes}
\usepackage{hyperref}
\makeatletter
\patchcmd{\insertverticalnavigation}%
{\ifx\beamer@nav@css\beamer@hidetext{\usebeamertemplate{section in sidebar}}\else{\usebeamertemplate{section in sidebar shaded}}\fi}%
{{\usebeamertemplate{section in sidebar}}}{}{}
\makeatother


\usepackage{graphicx}
\begin{document}
%	\setlist[itemize]{noitemsep, topsep=0pt}
\title[GCC Markets]{\textbf{Cross-Section of Returns in Frontier Markets:
		Evidence from the GCC Markets}}
\vspace{-24pt}
\author[J. Poland]{{Bader S. Alhashel}}
\date{\today}
\institute[ISB]{Jeramia Poland\\
	\includegraphics[scale=.5]{/home/jeramia/Pictures/Latex_Central/isb-logo.jpg}\\Indian School of Business}

\begin{frame}
\titlepage
\end{frame}

\section{FF3-GCC}
\begin{frame}{What the paper does}
   \begin{itemize}[<+->]
   	\item Arbitrage factors are tested in the GCC markets (combined and individually)
   	\item CAPM \& FF-3 are tested in the GCC markets
   	\item The results are in-line with prior research (developed, emerging, and frontier)
   \end{itemize}
\end{frame}

\begin{frame}{What the paper depends on}
 \begin{itemize}[<+->]
 	\item Interest in GCC
 	\item Factors have not been tested before in GCC
 	\item Return anomalies exit in GCC markets (prior lit)
 	\item Diversification investment (growing)
 	\item Oil markets
 	\item "Out-of-sample" factor tests
 \end{itemize}
\end{frame}

\begin{frame}{What the paper has going for it\\
		and what it doesn't}
\begin{itemize}[<+->]
	\item The execution is good
	\item The setting has potential
	\item The novelty of the emerging or frontier GCC probably not enough
	\item The results are not too surprising
\end{itemize}
\end{frame}

\section{Suggestions}

\begin{frame}{Comment}
\begin{itemize}[<+->]
	\item As is the paper really needs to convince the readers that the GCC is both different from prior markets and the same - tension between OOS testing and validation of prior results
	\item To argue the reader should be interested in GCC results in themselves there has to be some reason to expect unexpected results (different return generating factors)
	\item (lack of) Results for GCC and Kuwait (FF-3)
	\item "Out-of-sample": only yields meaningful results if the validation set has the same data generating process (same factors)
	\item The asset pricing model tests don't add much without and evaluation on the impact of prior work
\end{itemize}
\end{frame}

\begin{frame}{Suggestions}
	\begin{itemize}[<+->]
	\item Create some tension around the GCC - unexpected results
	\item Don't refer to the GCC - go bigger and more fundamental
	\item Position as test of asset pricing factors in commodity (petrol) heavy economy - MCap\_comm / MCap; return sensitivity
	\item \begin{itemize}
		\item Gorton and Rouwenhorst (2006); Buyuksahin, Haigh, and Robe (2008); Bhardwaj, Gorton, and Rouwenhorst (2015) document the lack of relationship between equity and commodity returns
		\item As cited its seen as a diversification location
		\item Discount rates could be set by risk factors outside the observed market
		\item Compare / contrast - US 1970s; Russia; South America w/ GCC and Kuwait
	\end{itemize} 
	\item Alternatively, institutional penetration or type
\end{itemize}
\end{frame}

%\begin{frame}{Why widen the paper}
%	\begin{itemize}[<+->]
%		\item Paper already contains some interesting observations which look unintuitive:
%		\begin{itemize}[<+->]
%			\item 2014 collapse in oil prices and the non-response by OPEC
%			\begin{itemize}[<+->]
%				\item We have the conclusion that profit is not sensitive to storage costs - What about in the face decreasing prices, as in 2014? Was OPEC acting optimally?
%				\item We have for fixed storage quantity the available quanity becomes less important as potential income decreases, What happens when extraction/production is fixed? Was there no real OPEC response one production because they had room to act optimally through storage?
%				\item General statement on fixed cost investment (storage) exposes members of an uncompetitive market to entry
%			\end{itemize}
%			\item In several places reserves levels become unimportant to the current management decision, typically as a result or affect
%			\begin{itemize}[<+->]
%				\item Making the reserve nullification a cause, could this have something to say about the emergence of alternative energy sources? 
%				\item General statement on the management of obsolete or expiring goods in less competitive markets, or in the face of uncertain pricing, demand or production?
%			\end{itemize} 
%		\end{itemize}
%	\end{itemize}
%\end{frame}

\begin{frame}{Summary}
\begin{itemize}[<+->]
	\item The papers content is well done
	\item A clearer stand needs to be taken on why GCC
	\item The motivation needs to claim bigger
	\item The data may provide the distinctions needed
\end{itemize}
\end{frame}


\end{document} 