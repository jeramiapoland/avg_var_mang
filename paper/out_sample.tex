Average variance is a good in-sample signal, however the out-of-sample performance remains in doubt. As \citet{Welch2008} definitively show, out-of-sample performance is not guaranteed by in-sample performance and is essential to any investment strategy which hopes to generate positive returns. To determine the out-of-sample relationships between market and average variance, average correlation and returns, I run regressions of the standard form
\begin{equation}
	\label{eq:oos_pred_reg}
	y_{t+1} = \alpha_{t} + \beta_{t}x_{t} + \epsilon_{t}
\end{equation}
where $\alpha_{t}$ and $\beta_{t}$ are estimated with from the data available only until time $t$. That is, I estimate
$\alpha_{t}$ and $ \beta_{t}$ by regressing $\{y_{s+1}\}_{s=1}^{t-1}$ on a constant and $\{x_{s}\}_{s=1}^{t-1}$. In all the reported results, I follow an expanding window approach so that for the next period $t+2$, $ y_{t+2}$ is estimated as $\alpha_{t+1} +  \beta_{t+1}x_{t+1}$, where $\alpha_{t+1}$ and $ \beta_{t+1}$ by regressing $\{y_{s+1}\}_{s=1}^{t}$ on a constant and $\{x_{s}\}_{s=1}^{t}$. I follow this process for all subsequent months. However, as part of a test on the robustness of the out-of-sample results, I demonstrate that the results do not depend on the use of an expanding window. Most critically, equation (\ref{eq:oos_pred_reg}) prevents any look-ahead bias. The out-of-sample prediction tests use the same set of variables as the in sample tests. Each out-of-sample test requires an in-sample training period in which parameters are estimated using all the data up to the time period before the first out-of-sample quarter or month.

For consistency, the first one-fourth of the data is used as the initial parameter estimation period with the remaining three-fourths of observations moved through recursively generating out-of-sample predictions. Three measures of out-of-sample performance are estimated.
%\footnote{Additionally, in unreported results, I find the same results if we use the \citet{Diebold1995} test.} I use the $R^{2}_{oos}$ statistic \citet{Campbell2008}, the mean squared error F-statistic of \cite{clark_tests_2001} to evaluate the accuracy of the out-of-sample predictions. $R^{2}_{oos}$ is defined as
%\begin{equation} \label{eq:oos_r2}
%R^{2}_{oos} = 1 - \frac{MSFE_{x}}{MSFE_{b}}
%\end{equation}
%where $MSFE_{x}$ is the mean squared forecast error when the variable $x$ is used to generate out-of-sample predictions. An $R^{2}_{OS} > 0$ suggests that $MSFE$ based on variable $x$ is less than that based on the benchmark, b. We evaluate the statistical significance of $R^{2}_{OS}$ using \cite{Clark2007} statistic. This statistic tests the null hypothesis that $H_{0}: R^{2}_{OS} \leq 0$ against the alternative $H_{A}: R^{2}_{OS} > 0$.
I use the \citet{Diebold1995} statistic and \citet{mccracken_asymptotics_2007} MSE-F as measures of the increased accuracy of AV based forecasts compared to forecasts from SV as a benchmark. The DM statistic is defined as:
\begin{equation}
	DM = \frac{\bar{d}}{\sqrt{\frac{2\pi f_{d}(0)}{T}}}
\end{equation}
where $\bar{d}$ is the mean difference in the loss differential. The loss differential is the function used to measure the difference between the forecasted and actual values. I use the squared forecast error, $(y_{t} - \hat{y}_{t})^{2}$. So, $\bar{d}$ is the mean value of the difference between the squared error using AV and the squared error using the benchmark forecast from SV.
\begin{equation}
	\bar{d} = \frac{1}{T}\sum_{T}^{\tau=1}((y_{t} - \hat{y}_{AV,t})^{2} - (y_{t} - \hat{y}_{SV,t})^{2})
\end{equation}
I use the same consistent estimator for the mean loss differential, $f_{d}(0)$ as in \citet{Diebold1995}. The statistic is normally distributed under the null hypothesis of no difference in accuracy between the benchmark and proposed model. The standard postive critial values from the normal Z-table serve as cutoffs to establish a significant improvement in forecast accuracy. $MSFE_{SV}$ is mean squared forecast error when a benchmark model is used to generate out-of-sample predictions. Mean squared forecast error is defined as
\begin{equation} \label{eq:msfe}
	MSFE_{x} = \frac{1}{T} \sum_{\tau=t}^{T} (y_{\tau} - \hat{y}^{x}_{\tau})^{2}
\end{equation}
where $\hat{y}^{x}_{t}$ is the out-of-sample prediction of $y_{t}$ generated from the a model using variable x, t is the first out-of-sample prediction time period, and T is the total number of out-of-sample time periods. The F-statistic in \citet{mccracken_asymptotics_2007} is calculated by
\begin{equation} \label{eq:mse-f}
MSE-F = T \frac{MSFE_{x}-MSFE_{b}}{MSFE_{b}}.
\end{equation}
The significance of the F-statistic is determined from bootstrapped values provided in \citet{mccracken_asymptotics_2007}.
Each of these two tests depends on the reduction of average squared error by the predictor x relative to a benchmark model. The final measure is a forecast encompassing statistic. 

Encompassing tests the more stringent requirement that the benchmark forecasts contain no useful information absent in the forecasts of variable x. Forecast encompassing tests come from the literature on optimal forecast combination. \citep{Chong1986,Fair1990} An optimal forecast as a convex combination of two forecasts for time period t +1 defined as
\begin{equation} \label{eq:forecast_comb}
\hat y^{*}_{t} = (1-\lambda)\hat y^{b}_{t} + \lambda \hat y^{x}_{t}
\end{equation}
where $\hat y^{x}_{t}$ are predicted values generated from the model using variable x and $\hat y^{b}_{t}$ are forecasts from the benchmark model. I use the forecast encompassing test of \citet{harvey_tests_1998}, ENC-HLN.
%, and the ENC-NEW statistic of \cite{clark_tests_2001}.
%The ENC-NEW statistic is defined as
%\begin{equation}
%ENC-NEW = T \frac{\sum_{\tau = t}^{T}{\left[(y_{\tau} - \hat{y}^{b}_{\tau})^{2}-(y_{\tau} - \hat{y}^{b}_{\tau})(y_{\tau} - \hat{y}^{x}_{\tau})\right]}}{\sum_{\tau = t}^{T}(y_{\tau} - \hat{y}^{x}_{\tau})^{2}}.
%\end{equation}
%The significance of the ENC-NEW is determined from bootstrapped values provided in the same paper.
The encompassing test of \citet{harvey_tests_1998} directly tests the value and significance of the forecast combination $\lambda$. The test procedure rests on the calculation of a modification to the \citet{Diebold1995} test statistic and the consistent estimation of the long-run covariance between the difference in forecast error between the benchmark model and a model based on a competing variable, x. As such there is no one line equation that sums up the statistic used to judge the significance of $\lambda$. However, intuitively $\lambda$ must be significantly different from zero for AV to have information above and beyond the forecasting information in SV and values close to one indicate that AV has all of the relevant information in SV and is optimal by itself.
\bigskip
\centerline{\bf [Place Table~\ref{tab:tab_out_sample} about here]}
\bigskip

The results in table \ref{tab:tab_out_sample} show that AV is a significantly better out-of-sample predictor of AV and SV. For both the variables, all three measures of out-of-sample performance show significant imporvement in both the out-of-sample period starting in 1970 and starting in 1939. The forecast encompassing tests also show that AV contains all the forecasting information in SV and is optimal on its own. This means that investors concerned about the variance in the returns on their investment in the market are better off using this month's level of average asset variance to hedge next month's stock market variance than using this month's stock market variance. AV is also a significantly better predictor of next month's log excess return for the out-of-sample period starting in 1970. The DM statistic, 1.278, is nearly significant at the 10\% level and both the MSE-F test and encompassing tests show significant imporvement for AV. Again the encompassing test shows that AV is optimal alone and no weight needs to be given to SV in the prediction of log excess returns. Over the period starting July 1970, investors attempting to predict next month's returns would have been better off using this month's average asset variance rather than total market variance and investors deleveraging based on high values of AV would have done better avoiding negative returns rather than deleveraging based on high values of SV. Supporting the results seen in the in-smaple tests, across the longer out-of-sample period, starting in 1939, AV is not a significant improvement over SV in the prediction of next month's log excess return. It will be important to look at the asset allocation performance of AV versus SV over the entire CRSP sample to see if portfolio management by AV does generate significantly higher returns than management by SV.

%Table \ref{tab_oos} shows the results of the out-of-sample tests. Panel A contains the results of running out-of-sample expanding window regression using AV as the predictor versus using only a constant. The use of only a constant in the rolling regression is effectively using the running historical mean as a benchmark in the tests of out-of-sample performance. It is clear that AV is a significant improvement in the prediction of next period market variance. It generates postive and significant $R^{2}_{oos}$ values in all samples. The values generated for MSE-F and ENC-NEW are very large and statistically significant; and the near 1 $\lambda$ values for the ENC-HLN tests, in all samples, indicate that AV provides all the information available in the historical mean. Its also clear that AV is not a predictor of future returns. $R^{2}_{oos}$ values are negative or insignificant indicating. Some positive and significant values appear in the MSE-F, ENC-NEW and ENC-HLN tests at the monthly level which is the result of how poor a predictor of monthly returns is the historical mean. Particularly around market downturns use of the historical mean generates massive prediction errors.
%
%Panel B shows the results of running the same test with forecasts generated by SV as the benchmark model. The results for the same but the magnitudes are very different. AV is a significantly better predictor of next period's market variance in all samples. However, the performance over and above SV is much more muted than in the historical mean in panel A. At the monthly frequency, the frequency used later for asset allocation, AV is a significant improvement over SV as indicated by the positive $R^{2}_{oos}$, MSE-F, and ENC-NEW values. However, the ENC-HLN $\lambda$ value is on .56. While this is statistically significant, it indicates that for the best out-of-sample prediction of monthly stock market variance both AV and SV should be included and at nearly equal weights. The results for the prediction of future excess log returns are even, slightly, better than those in panel A. This is again owning as much to the terrible performance of the benchmark, here SV, as to any positive performance of AV.