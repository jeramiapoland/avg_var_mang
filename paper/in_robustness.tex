Average variance is an auto-correlated time series; this opens the possibility that predictive regressions using AV have estimation bias as highlighted in \citet{stambaugh_predictive_1999}. \citet{campbell_no_1992} show that the Stambaugh bias in predictive regressions involving volatility measures and future returns can be particularly severe because of a ”volatility feedback” effect. To eliminate the Stambaugh bias in the estimated coefficients on AV in the regressions above, I follow the methodology in \citet{Amihud2004} and furthur make the p-values used for coefficient significance robust through wild-bootstrapping as detailed in \citet{mackinnon_bootstrap_2002}. Table \ref{tab:tab_in_sample_robust} shows that the relationships demonstrated above are unaffected by robust bias correction. Average variance is a predictor of higher average correlation and higher stock market variance across data sets. While AV is a significant predictor of lower returns in the 1962 forward period, it is unrelated to the next month’s log excess returns in the whole data set. So long as the relationships hold with the limited information that investors have available at the time they make investment decisions, AV is likely to be a better leverage management signal than SV.
%\bigskip
%\centerline{\bf [Place Table~\ref{tab:tab_in_sample_robust} about here]}
%\bigskip