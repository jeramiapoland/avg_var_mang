The average variance managed portfolio produces significantly higher returns than the buy and hold market portfolio for the same level of total variance. As long as investors have access to either portfolio, there must be another investor untility generating aspect to the holding the market portfolio in order for this risk-return relationship to not violate modern portfolio theory. Either investors are compensated more for the market portfolio or penalized more for the AV portfolio in utility terms. A reasonable explaination can be taken from the literature on the cross section of the low-risk anonomaly associated with "lottery-like" stocks. Lottery stocks are known to generate lower future returns and this literature posits a lottery preference for some investors. \citep{barberis_stocks_2008} This preference suplements the utility from assets returns with a lottery dividend the investors receieves from holding positions in assets which provide the possiblity of very high returns with low probability. These investors optimize their holdings in the presence of a reward for holding assets that provide the same type of satisfaction or excitment as holding a lottery ticket or placing a wager. Another reasonable explaination is that investors are either prevented from or excessively penalized through costs for taking the leverage needed for either the AV or SV suggested holdings. This lending constraint explaination has a long history originating soon after the idea of mean-variance portfolio optimization.

\subsection{Changes to the Capital Market Line}
In the single time period capital asset pricing model, CAPM, world, the market portfolio is the optimal risky portfolio and all investors hold some combination of the market and the risk-free rate. \cite{sharpe_capital_1964,lintner_valuation_1965} At the end of each month AV managed investors are taking a weight $\omega_{t}$ in the market portfolio proportional to the variance of the daily return of the market over the last month. Thus, in month t+1 the return to the AV management strategy is just:
\begin{equation} 
	r_{AV,t+1} = \omega_{t}r_{M,t+1}
\end{equation}
where $r_{M,t+1}$ is the return on the CRSP market portfolio. Both expected return and standard deviation scale directly with the investment weight so for each single time period the AV managed portfolio lines on the Capital Market Line, CML, defined by the market portfolio and risk-free rate. 
As a consequence, AV managed returns should perfectly fit the standard CAPM regression.
\begin{equation} \label{eq:eq_returns}
	r_{AV,t} = \alpha_{t} + \beta_{1}r_{M,t} + \epsilon_{t}
\end{equation}
However, when investors are subject to externatlities or frictions their optimal allocation changes and in equilbirium the market portfolio changes. When the externality is the utility from gambling through investment in lottery-like stocks the increased demand drives changes in prices and lower expected returns and as \citet{barberis_stocks_2008} show if shorting stocks is costly this mispricing cannot be arbitraged away, equilibrium holdings are heterogenous and the market portfolio return is lower. \citet{kumar_who_2009} and \citet{fong_risk_2013} also find lower returns in the presence of lottery preferences since investors are recieving compensation for the variance in returns through something other than expected return they are willing to accept the lower expected returns. When investment calls for high levels of leverage, demand for lending makes it costly. As \citet{black1972capital} first shows this decreases the excess return of portfolios leveraged into the market portfolio and flattens the CML. Indeed this affect on the CML is a method to identify which is affecting investor desicions lottery preferences or lending constraints. \citet{jylha_margin_2018} identifies the decrease in the slope of the CML given changes to the amount of equity investors are forced to deposit when taking loans to invest. So, both lottery preferences and lending constraints should flatten the CML as shown by \citet{barberis_stocks_2008}, \citet{kumar_who_2009}, \citet{fong_risk_2013}, \citet{bali_lottery-demand-based_2017}, \citet{black1972capital}, \citet{frazzini_betting_2014}, \citet{jylha_margin_2018}, however this effect occurs in equilibrium making it difficult to measure. %Following \citet{jylha_margin_2018} makes the argument that changes in equity requirements are exogenous and thus lending constraint effects on the CML are identified. This is not necessary. 
As \citet{bresnahan_oligopoly_1982} demonstrates, in a slightly different context, despite moving from one equilibrium to another the factor affecting returns can be measured with the interaction term of the market return and the proxy for the factor being tested. Figure \ref{fig:fig_cml} motivates the intuition graphically, as in \citet{bresnahan_oligopoly_1982}.
\bigskip
\centerline{\bf [Place Figure~\ref{fig:fig_cml} about here]}
\bigskip
While in any single month the relationship between the return on the market portfolio and the AV managed portfolio is fixed by the weight suggested by the prior average variance of the daily market returns, the interaction of the market return with, for example, a proxy for the lending constraints in the market will give us a measure of the effect the lending constraint has on the returns to AV. Using the regression specification:
\begin{equation}
		r_{AV,t} = \alpha_{t} + \beta_{1} r_{M,t} + \beta_{2} r_{M,t}x_{t} + \beta_{3}x_{t} + \boldsymbol{\beta}_{t}\boldsymbol{\chi}_{t} + \epsilon_{t}
\end{equation}
where x is a proxy for either lottery preference or lending constraint and $\chi$ is the vector of other controls, changes not only in the location capital market line but in any rotation around the market portfolio are captured. Changes in the rotation of the CML around the market change the returns to the AV managed portfolio; it only remains to specify what we expect to see from the regression. In the extreme, investors are subject to borrowing conditions which make them all indifferent, or worse, to borrowing. At this point the CML has rotated so that the portion beyond the market portfolio is completely flat; the cost of borrowing is the same as an additional return. In the extreme, there is sufficient lottery dividend that investors forego any excess return from the market portfolio and become indifferent between investing in the market with its low return and lottery dividend and the risk-free rate. Again, the CML has completely flattened. When this occurs the returns to the AV managed portfolio are the same as the returns to the market portfolio and from equation \ref{eq:eq_returns} when these two returns are the same $\beta_{1}$ will equal one. Hence, if $\beta_{1}$ is less than one $\beta_{2}$ should be positively significant and if $\beta_{1}$ is greater than one $\beta_{2}$ should be negatively significant.

\subsection{Lottery Preference}
%We have some reason to suspect leverage already, the constraints drive returns toward BH and some reason to doubt lottery Rachev.
Already, leverage constraints appear to be a promising explaination given how returns to both the SV and AV managed portfolios converge to the buy and hold as investment is limited. Also there is some reason to doubt the influence of lottery preferences as the Rachev ratio for both the SV and AV manged portfolios are better than for the buy and hold. Each of the managed portfolios offeres higher extreme returns for extreme losses. Indeed, the fundamental argument that investors prefer the buy and hold market over the average variance or volatility managed investment because the market is more lottery-like is unclear.

\citet{bali_maxing_2011} show that the maximum daily return over the past one month, MAX, is a good measure of the lottery-like payoffs of a stock and a significant indicator of lower future returns robust to size, book-to-market, momentum, short-term reversals, liquidity, and skewness. This means that for lottery seeking investors to prefer the buy and hold market its MAX measures must be significantly different from the average variance and volatility managed portfolios. Yet, as seen in table \ref{tab:tab_max_stats} the mean and median values of the highest one day returns, MAX1, and the average of the five highest daily returns within the month, MAX5, are highest for the average variance managed portfolio. Using daily return values scaled by the prior months portfolio volatility, as in \citet{asness_betting_2018}, the volatility managed portfolio is the most lottery like with the highest mean and median values of scaled MAX1 and MAX5. Notably, the average variance managed portfolio still has higher mean SMAX1 and SMAX5 values than the buy and hold portfolio. The buy and hold investment is the most lottery like in no measure. This does not mean that the buy and hold portfolio is not viewed as a lottery and investors do not take some additional utility from holding it, however it seems very unlikely that this is greater than the lottery utility provided by the average variance or volatility managed portfolio let alone large enough to compensate for the difference in return or CER gain.

It remains possible that on some other yet unknown measure the buy and hold strategy is more lottery like. To form a more direct test of the lottery preference explaination I test the effect of lottery preferences on the returns to the AV managed portfolio using measures of the market capitalization of gaming industry firms, MAX5 and MAX5 scaled for the market portfolio. On the assumption that high lottery preferences across the public will manifest in higher market capitalization for gambling and gaming firms, I sum the market capitalization for 27 firms whose main business is gambling, e.g., casinos and horse tracks. Table \ref{tab:tab_gambling} GMCAP is the gaming industry market capitalization calculated monthly. Total gaming capitalization is likely tied closely to total market capitalization which some argue is related to future returns. To seperate the effect of total market capitalization, I calculate GMCAP$_{scaled}$ as the ratio of the market capitalization of gaming firms to the total market capitalization of the 500 largest firms.

Table \ref{tab:tab_lottery} presents the results of running regressions using the proxies for lottery preference levels in the economy. \citet{fama_dissecting_2016} argue that measurement of low-risk anomoly excess returns requires controlling for the known Fama-French risk factors and although that is in a cross-sectional specification I include them here to control for other sources of aggregated mispricing. The GMCAP and GMCAP$_{scaled}$ series start in December 1985.  

The final performance ratio investigated is the Rachev. Developed in \citet{biglova_different_2004}, the Rachev ratio is the expected tail return in the best n\% of the return distribution over the expected tail loss in the worst n\%. I set n = 5 and measure the top 5\% expected return over the expected loss in the worst 5\%. The general Rachev ratio formula is, again against a target return level of 0:
\begin{equation}
\frac {ETL_{\alpha }\left({{r_{f}}-x'r}\right)}{ETL_{\beta}\left({x'r-{r_{f}}}\right)}
\end{equation}
where $ET{L_{\alpha }}={\frac {1}{\alpha }}\int _{0}^{\alpha }{VaR_{q}\left(X\right)dq}$.

\subsection{Lending Constraints}
High levels of bank credit growth are associated with the overextension of credit in the past, tighter current and future lending conditions, and lower future market returns.

Large positive shock, high value of $LF_{AEM}$, are associated with time of high intermediary funding illiquidity. Large bank debt constrains brokers abilities to aquire more funds to lend to investors and limits there willingness to lend to "risky" investors.