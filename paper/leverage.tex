More practical analysis of portfolio performance requires the incorporation of limits on the level of investment taken in the market portfolio. Leverage of 50\%, a coefficient of 1.5 on the market, is a common constraint meant to mimic real market leverage constraints for the average investor based in part on the Reg. T margin requirement\footnote{Federal Reserve Board Regulation T (Reg T) establishes a baseline requirement that investors deposit 50\% of an investment position in their margin trading accounts, however a brokerage house may set a higher equity requirement.}. \citep{Campbell2008,Rapach2010,Rapach2013,Huang2015,Rapach2016,moreira_volatility-managed_2017,deuskar_margin_2017} There are at least two exchange traded funds, ETFs, which three times the return of the SP500.\footnote{The Direxion Daily S\&P 500 Bull 3x Shares ETF, symbol SPXL, and ProShares Ultra Pro S\&P 500 ETF, symbol UPRO, are two such funds.} So, I take a market coefficient of three as the maximum feasible investment a typical investor can make in the market portfolio. 
\bigskip
\centerline{\bf [Place Table~\ref{tab:tab_performance2} about here]}
\bigskip
Table \ref{tab:tab_performance2} panel (c) presents the results from applying investment constraints after calculating the weights for AV and SV targeting the buy and hold volatility. While both portfolios still outperform the buy and hold, The seperations in average annualized excess return, 1.71\% and 2.07\%, are even greater when investment constraints are applied. Panels (a) and (b) in figure \ref{fig:fig_returns} show the effects of the growing separation. While SV is barely able to clear the buy and hold strategy under typical brokerage constraints, returns to the AV managed portfolio remain clearly above. Investors that use a leveraged SP500 ETF to impliment the AV managed portfolio strategy are rewarded with returns significantly higher than the SV managed portfolio suggested weights. The brokerage investment restrictions pull the volatility of the AV managed portfolio too far from the SV returns to generate significant differences in performance ratios. However, investors using the leveraged ETFs are rewarded not only with higher returns but significantly better performance ratios across the board with the exception of the Rachev ratio which is at least no longer signficantly better for the SV managed portfolio. The ETF leverage constrained AV strategy even generates better Sharpe and Sortino ratios than the unconstrained strategy. The results in Panels (a) and (b) demonstrate that better performance of AV is not a result of or contingent on looking to the volatility of the buy and hold strategy. As the targeted volatility is lowered the difference in performance between AV and SV becomes more significant. %With the exception of the Rachev ratio, every performance measure is significantly better for the AV managed portfolio given lower volatility targets and investment constraints.

The differences in investment weight profiles show in table \ref{tab:tab_weights} not only generate differences in returns but also in costs. As show in table \ref{tab:tab_costs} the AV managed portfolio generates less than half the turnover of the SV managed portfolio. The average monthly absolute change in investment weight is .752 for the SV managed portfolio and only .317 for AV. Table \ref{tab:tab_costs} also shows the Fama-French five factor and Fama-French five factor with momentum annualized alphas for the SV and AV managed portfolios. AV management results in higher annualized alphas, 3.036\% and 3.152\% vs 2.833\% and 2.769\%, than SV. To capture the effect of these difference in annualized alphas and turnover, I calculate the transaction costs needed to zero the strategy alphas so the portfolio only breaks even. \citep{frazzini_trading_2015,moreira_volatility-managed_2017} Seen in table \ref{tab:tab_costs}, the break even transaction costs are more than 2.5 time higher for the AV managed portfolio. The SV managed portfolio breaks even at 31.436 and 30.725 basis points while it takes costs of 79.992 and 83.061 basis points to zero out the AV managed annualized alphas. However, transaction costs are only the only expense incurred by the leveraged portfolios. To estimate the borrowing costs for each strategy, I assume that any month a strategy requires a position greater than one in the market the difference between the investment weight and one is borrowed. The average monthly cost of borrowing to invest for the AV managed strategy is nearly 25\% lower than for the SV managed portfolio. Using the broker call money lending rates available in Bloomberg from September 1988 to October 2016, SV incurs an average monthly cost of 15.107 basis points while AV costs only 11.411 basis points. This ignores the possibility of the investor using saved gains rather than borrowing and the necessity of borrowing when the investor has lost money, however, as the AV managed portfolio has greater returns and better drawdown statistics included gains and losses would only further the seperation.
\bigskip
\centerline{\bf [Place Table~\ref{tab:tab_costs} about here]}
\bigskip