More practical analysis of portfolio performance requires recognition of the borrowing conditions and leverage constraints faced by real-world investors. As table \ref{tab:tab_costs} shows, when targeting the volatility of the market portfolio, both portfolios are leveraged into the market on average with investment weights of 1.3, indicating 30\% leverage. Regardless of the volatility target the SV managed portfolio calls for extreme levels of leverage. Figure \ref{fig:weights_plot} shows that the SV strategy calls for investment weights above the maximum AV weight in several periods. More than 500\% leverage is needed at the end of the 1920s, throughout the 1960s, and in the 1990s. Given that these levels of leverage are unrealistic for most investors, it will be important to see if there is a difference in performance for the AV and SV strategies under real-world investment constraints and to investigate the associated costs generated by the trading needed for the SV managed portfolio.
\bigskip
\centerline{\bf [Place Figure~\ref{fig:weights_plot} about here]}
\bigskip

%Table \ref{tab:tab_performance3} shows the portfolio performance results in months with the highest and lowest 25\% borrowing costs measured by the risk-free rate, one month T-bill. There is a clear difference. The SV managed portfolio has better returns and performance rations in the period with the highest risk-free rate. AV is the better performer in the months in the bottom 25\% of the risk-free rate distribution. It is clear that lending constraints and borrowing conditions affect portfolio performance particularly for the SV managed portfolio which takes extremely large positions in the market in some months. 
\subsection{Investment Weights}
One method of measuring the impact of investment constraints on the AV and SV strategies is to externally impose limits on the weights generated by each. Leverage of 50\%, a coefficient of 1.5 on the market, is a common constraint meant to mimic real market leverage constraints for the average investor based in part on the Reg. T margin requirement\footnote{Federal Reserve Board Regulation T (Reg T) establishes a baseline requirement that investors deposit 50\% of an investment position in their margin trading accounts, however a brokerage house may set a higher equity requirement.}. \citep{Campbell2008,Rapach2010,Rapach2013,Huang2015,Rapach2016,moreira_volatility-managed_2017,deuskar_margin_2017} \citet{ang_hedge_2011} find mean hedge fund leverage to be around 36\%, an investment weight of 1.36, and at the extreme tops out at 400\%. There are at least two exchange traded funds, ETFs, which target three times the return of the SP500.\footnote{The Direxion Daily S\&P 500 Bull 3x Shares ETF, symbol SPXL, and ProShares Ultra Pro S\&P 500 ETF, symbol UPRO, are two such funds.} So, I take a market coefficient of three as the maximum feasible investment a typical investor can make in the market portfolio. This represents 200\% leverage is easily accessible for all investors.
\bigskip
\centerline{\bf [Place Table~\ref{tab:tab_performance2} about here]}
\bigskip

Table \ref{tab:tab_performance2} presents the results from applying investment constraints after calculating the weights for AV and SV targeting the buy and hold volatility. Panel (b) shows the ETF, 200\%, leverage constraint. AV management outperforms by 1.71 and 2.07 percentage points when targeting buy and hold volatility.  Investors using the leveraged ETFs are rewarded not only with higher returns but significantly better performance ratios across the board. SV management takes a performance hit first as leverage constraints are applied due to its more extreme leverage. The ETF leverage constrained AV strategy even generates better Sharpe and Sortino ratios than the unconstrained strategy. Panel (a) shows the results of applying the 50\% leverage limit. The brokerage investment restrictions pull the volatility of the AV managed portfolio too far from SV to generate significant differences in performance ratios, but the difference in AV and SV managed returns is always greater than one percentage point and close to two when targeting the buy and hold volatility. Panels (a) and (b) in figure \ref{fig:fig_returns} show the effects of the growing separation. While SV is barely able to clear the buy and hold strategy under typical brokerage constraints, returns to the AV managed portfolio remain clearly above.  %With the exception of the Rachev ratio, every performance measure is significantly better for the AV managed portfolio given lower volatility targets and investment constraints.
The results in Panels (a) and (b) of table \ref{tab:tab_performance2} demonstrate that better performance of AV is not a result of, or contingent on, looking to the target volatility. AV management is better, regardless of variance target, as leverage constraints are applied.

\subsection{Costs}

The differences in investment weight demanded by AV and SV management shown in table \ref{tab:tab_weights} not only generate differences in returns but also in costs. As show in table \ref{tab:tab_costs} panel (a), for the US market, the AV managed portfolio generates less than half the turnover of the SV managed portfolio. The average monthly absolute change in investment weight is 0.752 for the SV managed portfolio and only 0.317 for AV. Table \ref{tab:tab_costs} shows the trading costs needed to reduce the annualized average return of the SV and AV managed portfolios to the buy and hold or to reduce the factor alphas to zero. \citep{frazzini_trading_2015,moreira_volatility-managed_2017} Seen in table \ref{tab:tab_costs}, the break even transaction costs are more than twice that for the AV managed portfolio. The SV managed portfolio breaks even at 29.422, 60.694 and 35.472 basis points while it takes costs of 254.364, 176.467 and 83.176 basis points to reduce AV returns to the buy and hold or zero out the annualized alphas. However, transaction costs are not the only expense incurred by the leveraged portfolios. To estimate the borrowing costs for each strategy, I assume that any month a strategy requires a position greater than one in the market the difference between the investment weight and one is borrowed. The average monthly cost of borrowing to invest for the AV managed strategy is nearly 25\% lower than for the SV managed portfolio. Using the broker call money lending rates available in Bloomberg from September 1988 to October 2016, SV incurs an average monthly cost of 15.107 basis points while AV costs only 11.411 basis points. This ignores the possibility of the investor using saved gains rather than borrowing and the necessity of borrowing when the investor has lost money, however, as the AV managed portfolio has greater returns and better drawdown statistics included gains and losses would only further the separation.

Table \ref{tab:tab_costs} panel (b) highlights the differences in the trading costs of the strategies across countries. The break even trading costs are calculated in reference to reducing the annualized average returns of the AV and SV managed portfolios to the buy and hold only. The AV managed portfolio is able to tolerate higher trading costs in every country analyzed. In all countries, as in the US market, the AV managed portfolio is able to tolerate higher transaction costs. In a majority of countries, 6 out of 9, the AV managed portfolio is able to tolerate trading costs twice as high as the SV managed portfolio. As for the individual country indices, the AV managed world index is far cheaper than the SV managed portfolio allowing an investor to keep more of the higher returns generated.