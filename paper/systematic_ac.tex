So, the relationship of AC to future stock returns and the ability of AC to signal an increase in the correlation of returns across the economy depends on the portion that the observed proxy, i.e. stock index, returns make of the market and the significance of the market in aggregate wealth. If the daily returns are not a good proxy for market returns and the market is not a significant component of aggregate wealth it is unlikely AC will serve as a signal of systematic risk or changes in the economy. An example of a similar effect described occurs in the difference in results shown by \citet{goyal_idiosyncratic_2003} and \citet{bali_does_nodate} when the latter removes a significant number of daily returns from the market proxy and the forecasting ability of idiosyncratic volatility disappears. When the coverage of the proxy daily returns decreases the information on the mix of systemic and non-systemic risk disappears.


%AV management works by limiting the risk investors face to uncompensated risk, as much as possible. AV is non-systematic, uncompensated risk. AC is compensated systematic risk. Equation (6) from \citet{pollet_average_2010} gives the relationship of AC, AV and the risk premium.
%\begin{align*}
%	E_{t}[r_{s,t}] - r_{f,t} + \frac{\sigma^{2}_{s,t}}{2} &= \frac{\gamma}{\beta_{t}(1-\theta_{t})}\bar{\rho}\bar{\sigma}^{2}_{t} - \frac{\gamma}{\beta_{t}(1-\theta_{t})}\theta_{t}\bar{\sigma}^{2}_{t}
%\end{align*}
%As they explain, ceteris paribus a change in AV has both positive and negative effects on returns with similar magnitudes. Equation (8), in \cite{pollet_average_2010}, relates AC and AV to the correlation of stock returns and the unobserved component of aggregate wealth.
%\begin{align*}
%cov(r_{s,t},r_{u,t+1}) &= \pi_{0} + \frac{(1-w_{s,t}\beta(1-\theta))E[\bar{\sigma}^{2}_{t}]}{(1-w_{s,t})\beta(1-\theta)}\bar{\rho}_{t}- \frac{(1-w_{s,t}\beta(1-\theta))E[\bar{\rho}_{t}]-\theta}{(1-w_{s,t})\beta(1-\theta)}\bar{\sigma}^{2}_{t}
%\end{align*}
%Again, they explain the denominators for both coefficients of interest are positive if the $\beta$ of the stock market on aggregate wealth is positive and the proportion of the market which is observed, $w_{s}$ is greater than zero but less than one. Hence, the coefficient for AC is positive if 1-$w_{s}\beta \geq 0$ or equivalently if the covariance between the unobservable return and aggregate wealth is positive. For plausible parameter values 1-$w_{s}\beta(1-\theta)$ is small and $E[\bar{\rho}_{t}]-\theta$ is close to zero, as $\theta$ represents the portion of the shock to stock returns which is common among returns. Thus, the effect of AV will be negative but small. So, the relationship of AC to future stock returns and the ability of AC to signal an increase in the correlation of returns across the economy depends on the portion that the observed proxy, i.e. stock index, returns make of the market and the significance of the market in aggregate wealth. If the daily returns are not a good proxy for market returns and the market is not a significant component of aggregate wealth it is unlikely AC will serve as a signal of systematic risk or changes in the economy. An example of a similar effect described occurs in the difference in results shown by \citet{goyal_idiosyncratic_2003} and \citet{bali_does_nodate} when the latter removes a significant number of daily returns from the market proxy and the forecasting ability of idiosyncratic volatility disappears. When the coverage of the proxy daily returns decreases the information on the mix of systemic and non-systemic risk disappears.

