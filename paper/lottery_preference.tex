The average variance managed portfolio produces significantly higher returns than the buy and hold market portfolio for the same level of total variance. As long as investors have access to either portfolio, there must be another investor untility generating aspect to the holding the market portfolio in order for this risk-return relationship to not violate the modern portfolio theory. A reasonable explaination can be taken from the literature on the cross section of the low-risk anonomaly associated with "lottery-like" stocks. Lottery stocks are known to generate lower future returns and this literature posits a lottery preference for some investors. \citep{barberis_stocks_2008} This preference suplements the utility from assets returns with a lottery dividend the investors receieves from holding positions in assets which provide the possiblity of very high returns with low probability. These investors optimize their holdings in the presence of a reward for holding assets that provide the same type of satisfaction or excitment as holding a lottery ticket or placing a wager.

Suppose there are S securities in the market indexed by s. Each trades at price P$_{s,t}$ at time t, pays a random dividend $\delta_{s,t}$, and has a total number of shares outstanding $X_{s}$. The return to holding a given s at time t+1 is $R_{s,t+1}$ which is related to all the other returns by covariance matrix $\Omega_{t}$ each of which is independent of the risk-free return $R_{f}$. A total of I investors, indexed by i, have individual wealths of $W_{i,t}$ and portfolios $x_{i} = (x_{1,i},\dots,x_{s,i})$. Each i has a risk aversion, $\gamma_{i}$ and lottery preference, $\iota_{i}$. The lottery utility investor i recieves from security s is a function of the holding in s, $x_{s,i}$, the lottery preference of i, $\iota_{i}$, and a measure of the lottery nature of s. This measure of how lottery-like security s is has typically been taken to be a function of the variance or skewness of the return of s, as in \citet{kumar_who_2009}, or a measure of the potential for exteme positive tail of the return distribution of s, as in \citet{bali_maxing_2011}. Where $l_{s,t}$ is the lottery-likeness of security s in month t:
\begin{align}
	l_{s,t} = g(\mu_{n,s,t}) \text{ or } g(MAX(s,t))
\end{align}
where $\mu_{n,s}$ is the n-th momemt of the daily return distribution for security s in month t and $MAX_{s}$ is the highest daily return for s in month t. These measures are highly correlated as high daily return variance or skewness generates high maximum daily return values. The utility payoff to holding s is then a function of i's lottery preference and the lottery-likeness of the security. As \cite{fong_risk_2013} shows this payoff is monotonic while it cannot be estimated, if lottery preferences are working, $\iota_{i} \textgreater 0$ and $\frac{\partial g}{\partial \mu_{n}} \textgreater 0$. The expected utility of investors i holding protfolio $x_{i}$, using the maximum daily return as a lottery measure, is then:
\begin{align}
	U = x'_{i}E_{t}(R_{x_{i},t+1}) + (1 + R_{f})(W_{i,t} - x'_{i}P_{t})-\frac{\gamma}{2}x'_{i}\Omega_{t}x_{i} + x'_{i}\iota_{i}MAX(R_{x_{i},t})
\end{align}