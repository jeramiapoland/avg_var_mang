Certainly, US equity investors and fund managers are better served by the AV managed portfolio than the SV. However clear this conclusion, it raises the questions of generalizability. Does the AV manged strategy work in equity markets outside of the US and does it still outperform the SV managed strategy? Testing the performance of AV management outside of US equities is even more important as confirmation of the argument that AV management works because returns to the AV managed portfolio depend more on AC and thus are more aligned with systemic risk. If AV management aligns better with systemic risk it should perform across assets.

\bigskip
\centerline{\bf [Place Table~\ref{tab:tab_int_summary} about here]}
\bigskip

To test the performance of AV management across international equity markets, I collect daily returns for the Australian (AUS), Brazilian (BRA), Chinese (CHN), German (DEU), French (FRA), Indian (IND), Italian (ITA), Japanese (JPN), and English (UK) markets from Compustat - Capital IQ. Table \ref{tab:tab_int_summary} shows the names of the indices used, the data time frames and the number of assets used for the calculation of market capitalization weights, AV and AC. In all cases the data series are much shorter. The primary limitations are the availability of daily returns and dividend data.

\bigskip
\centerline{\bf [Place Table~\ref{tab:tab_intPerf1} about here]}
\bigskip

The AV managent strategy is a better performer than SV across the world. Table \ref{tab:tab_intPerf1} shows the differences in performance and the AV managed portfolio generates higher annualized average returns and better Sharpe ratios in all countries except Italy. In the fastest growing markets, China and India, the AV managed portfolio increases annualized average return by 2.455\% and 2.637\%. Both AV and SV management are improvements over the buy and hold in all markets. Its possible that some differences, e.g., the UK, are insignificant.\footnote{International differences are not run through the bootstraping significance tests due to computational tracability.} Australia shows the best buy and hold Sharpe ratio at .614, but the AV management strategy is able to increase it to .981, a very attractive result for any investor. The results for the AV mangement strategy for the Chinese market are the most alluring, a 27.381\% annualized return with a .868 Sharpe ratio, but this also the shortest data series available only from 2005. 

\bigskip
\centerline{\bf [Place Table~\ref{tab:tab_intPerf2} about here]}
\bigskip

Table \ref{tab:tab_intPerf2} shows the drawdown statistics for each portfolio across countries. Unlike for the prior return measures, it is not always the case that SV management is an improvement over the buy and hold strategy. In Japan, India, and the UK, SV managed drawdowns are deeper, longer, and take longer to recover from on average. In every country but Australia, the AV managed portfolio has a shallower average drawdown. The AV managed portfolio has shorter average drawdowns in every country tested. In every country but China, the AV managed portfolio has a shorter average recovery time. Indeed, in every country but Japan, the AV management strategy is able to recover from a drawdown in less than 10 months on average.

\bigskip
\centerline{\bf [Place Table~\ref{tab:tab_intPerf3} about here]}
\bigskip

Table \ref{tab:tab_intPerf3} highlights the differences in the trading costs of the strategies across country. This table differs from \ref{tab:tab_costs} in that the break even trading costs are calculated in reference to reducing the annualized average returns of the AV and SV managed portfolios to the buy and hold and not reducing portfolio alphas to zero. The AV managed portfolio is able to tolerate higher trading costs in every country analyzed. In a majority of countries, 6 out of 9, the AV managed portfolio is able to tolerate trading costs twice as high as the SV managed portfolio.

Across the globe, the AV managed portfolio is a better leverage management signal. It results in higher annualized average returns. It captures in better Sharpe ratios. It generates better drawdown statistics and is cheaper, in trading costs, to execute. Thus, AV management is the better equity investment strategy around the globe. However, if AV management better aligns returns with systemic risk it may be a better management strategy in more than just equities.