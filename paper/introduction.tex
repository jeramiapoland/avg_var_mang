\noindent Invented in 1867\footnote{Invented by Edward A. Calahan, an employee of the American Telegraph Company.} and in wide spread use starting in the 1870s, the stock ticker provided the first reliable means of conveying up to the minute stock prices over a long distances and market participants have been discusing the relationship between returns, risk, and portfolios for at least as long.\citep{rutterford_financial_2016} The fundamental principal that there is a "risk-premium" such that more risky investment must generate greater returns to attack capital existed from the begining. While the fundamental idea that higher risk, measured in some way, required higher return. Markowitz brought formality to the notion of risk, portfolio construction and optimization in the 1950s with modern portfolio theory and mean-variance analysis which stated that portfolios with higher variance should have higher returns.\citep{markowitz_portfolio_1952} However, several challanges have appeared to this foundational mean-variance claim. One of the most basic is commonly refered to as the "low-risk" or "low-volatilty" anomaly. \citet{haugen_1972} found there was little to no evidence for a "risk-premium" for increased portfolio volatility and it was indeed easy to construct portfolios with lower volatility which earned greater returns than those with higher volatilites over a span of 20 years. A more mordern and comprehensive treatment of the low-risk issue can be found in \citet{moreira_volatility-managed_2017} who show, across investment strategies and asset classes that simply managing leverage in a portfolio by its volatility produces greater expected returns and performance ratios than the underlying portfolio. As such, these results fundamentally challenge the mean-variance notion of investment risk premium. This risk-return trade off is central to modern financial theory, so naturally I want to address this problem by making it worse. By managing the market portfolio using the average variance of the individual assets in the prior quarter or month, I generate higher expected returns and significantly better performance ratios than by managing using the market portfolio volatility. Thankfully, at least in the equity market setting, this worse problem exposes evidence consistent with the leverage explaination of the low-risk anomaly and inconsistent with the lottery story, while shedding light on the risk-return trade-off dynamics in structural asset pricing theories.

%Building on this work, I use the decomposition of portfolio risk and demonstrate that even greater performance can be had by managing equity investment by the average variance of the portfolio assets rather than just the variance of the portfolio itself and in doing so I shed light on a possible explaination of the low-risk anomaly while finding evidence against another.

\citet{moreira_volatility-managed_2017} demonstrate that volatility managed portfolios which decrease leverage when volatility is high produce large alphas, increase Sharpe ratios, and produce large utility gains for mean-variance investors. These results hold across investment styles, e.g. value or momentum, and in different asset classes, e.g. equities and currency portfolios. The results even hold for positions which already seek to exploit the low-risk anomaly like the "betting-against-beta" strategy of \citet{frazzini_betting_2014}. Additionally, other researchers have applied similiar variance or volatility management to specific assets or trading styles.\footnote{See, for example, \citet{barroso_momentum_2015} and \citet{kim_time_2016} for discussions of volatility managment of the momentum portfolio.} Practioners call approaches like this rik-parity and as of 2016 at least \$150 and as much as \$400 billion dollars was invested in risk-parity funds.\citep{steward_truly_2010,cao_risk_2016} The generation of greater expected returns without a equivalent increase in portfolio volatility implies that a representative investor should time asset volatility, demanding less return when its higher meaning the investor's risk appetite must be higher in periods like recession and market downturns when its expected to be lower. In short, it seems risk does not equal reward. However, some volatility is more equal than other volatility.

\citet{pollet_average_2010} decompose market variance into the average correlation between pairs of assets and the average variance of the individual assets yeilds a series strongly related to both future volatility and higher returns, average correlation, and a series strongly related to future volatility but unrelated to returns, average variance. Scaling investment in the market portfolio by the inverse of previous month’s average variance should then improve return performance over and above those managing investment by the previous market variance since it avoids, somewhat, decreasing investment when average correlation is high and higher future returns are expected. Average variance is also a better candidate for portfolio management because it has a better chance to pickup economic information sooner as individual assets respond to informtion about its own risk and expected returns as it's made public while changes in aggregate materalize as information across assets is combined.\citep{campbell1997econometrics,campbell_have_2001} \citet{asness_betting_2018} take a related approach to decomposing the betting-against-beta strategy of \citet{frazzini_betting_2014} into betting-against-correlation, BAC, and betting-against-variance factors, BAV, finding the BAC factor has a significant Fama-French five factor alpha but unrelated to behavior explainations of the low-risk anomally.\citep{fama_dissecting_2016} From the decomposition of daily market returns I find similar results about the more general low-risk strategy of volatility management.

Using daily market returns from the Center for Research in Securities Prices, I follow \citet{pollet_average_2010} generating quarterly time series of stock market variance, SV, average correlation, AC, and average variance, AV. I then extend this by calculating the time series monthly. As expected, AV is strongly related to next period market variance and unrelated, at best insignificantly negatively, related to next period excess log returns. As is standard, excess log returns are the difference between the log of one plus the CRSP market return and the log of one plus the risk free return. The monthly risk free return is the return on the one month treasury bill while the quarterly risk free rate is the return on the three month treasury bill, each series is collected from the St. Louis Federal Reserve Economic Data (FRED) website. This relationship not only holds both in and out of sample but in both National Bureau of Economic Research (NBER) defined business cycle contractions and expansions. Out of sample forecast accuracy and encompassing tests show that AV is a superior predictor of both SV and log excess returns next period. Out of sample testing always raises questions about model specification, recursive expansion versus rolling window parameter estimation, and choices of training period and prediction window. Using \citet{rossi_out--sample_2012} robust statistics I show that the superior performance of AV, against either the running historical mean or the current time period SV in either a rolling or expanding regression specification regardless of window choice. 

The robust performance of AV translates into significant asset allocation gains. A strategy which weights investment in the market portfolio produces average annualized log excess returns of 7.8\% quarterly and 8.7\% monthly. The AV strategy produces quarterly and monthly Sharpe ratios, .48 and .59, each statistically significantly greater than those of the SV, volatility, management strategy. Additionally, AV generates higher values in more asymetric risk-return measures like the Sortino, Kappa, and upside potential ratios. Asset allocation performance is not all perfect, however, and the lower performance of AV in terms of Rachev ratio hints at possible explaination for the low-risk anomaly seen in volatility and average variance management. 

"There is no such thing as a free lunch," is not only a wonderfully prevasive adage, particullary loved by economists, but a provable restriction on optimization problems.\citep{wolpert_no_1997} Here, AV provides no free lunch. The improved performance measured by expected log excess returns, Sharpe, Sortino, Kappa and upside potential ratios is betrayed by worse performance in Rachev ratio. The Rachev ratio measures the right tail reward value at play relative to the left tail value at risk. In short it measures the ratio of expected lottery rewards and losses. Volatility managed market portfolio investment has higher lottery winning potential for each dollar of potenial lottery loss. Average variance management thins both tails of the return distribution but the upper tail slightly more trading the positive shift in the expected value of the log excess returns with a loss in the most extreme possible positive return values. This is inconsistent with the notion that the low-risk anomaly is the result of a behavioral preference of investors for lottery-like returns.\citep{barberis_stocks_2008,brunnermeier_optimal_2007} Rather, the generation of higher log excess returns by the AV, SV, and betting-against-beta strategies support the notion that the low-risk anomaly arises from leverage constraints first suggested in \citet{jensen1972capital}. \citet{boguth_leverage_2018} link capital constraints and the betting-against-beta strategy through mutual fund betas, and more generally \citet{malkhozov_international_2017} show that international illiquidity predicts betting-against-beta returns. 

Some change in the risk-return dynamic through average variance management was anticipated given the relationship between AV and next period SV and the lack of relationship between AV and next period log excess returns. Given that idosyncratic events are reflected in individual asset volatility first we should also be able to explain the better performance of AV in terms of timming and response. 
