\noindent Discussions on when to risk investment capital in the pursuit of maximum return took place over
telegraph lines nearly a century before Markowitz brought formality to the notion of risk, portfolio construction and optimization in the 1950s.\footnote{Invented by Edward A. Calahan, an employee of the American Telegraph Company in 1867 and in wide spread use starting in the 1870s, the stock ticker provided the first reliable means of conveying up to the minute stock prices over a long distances and market participants have been discusing the relationship between returns, risk, and portfolios for at least as long.\citep{rutterford_financial_2016}} %The fundamental principal that there is a "risk-premium" such that more risky investment must generate greater returns to attack capital existed from the begining. While the fundamental idea that higher risk required higher return came before, Markowitz brought formality to the notion of risk, portfolio construction and optimization in the 1950s with modern portfolio theory and mean-variance analysis. 
Modern portfolio theory states that portfolios with higher variance need to generate higher mean returns to attract rational investors.\citep{markowitz_portfolio_1952} However, several challenges have appeared to this foundational mean-variance claim. One of the most basic is commonly referred to as the ”low-risk” or ”low-volatility” anomaly. \citet{haugen_1972} found there was little to no evidence for a "risk-premium" for increased portfolio volatility. While Warren Buffet may not be an enthusiastic supporter, low volatility portfolios
with unusually high expected returns present an opportunity for leveraged investing.\footnote{\citep{noauthor_warren_nodate}} \citet{moreira_volatility-managed_2017} show across investment strategies and asset classes that merely managing leverage in a portfolio by that portfolio's volatility produces greater expected returns and performance ratios.
These results seem to challenge the mean-variance notion of investment risk premium fundamentally. This risk-return tradeoff is central to modern financial theory, so naturally, I address
this problem by making it worse. \citet{pollet_average_2010} show that the average correlation of portfolio assets is the component of portfolio volatility most related to systemic changes in the economy and is the risk component compensated with higher returns. By managing the market portfolio using the average variance of the individual asset returns in the prior period rather than the variance of the market portfolio, I generate higher expected returns and significantly better performance ratios. In addition to identifying a better strategy for investors, decoupling the idiosyncratic variance of individual returns from the that of the market portfolio returns sheds light on the risk-return trade-off dynamics of the market and allows leveraging investment into times when higher risk will be compensated and pulling out when it will not. Thankfully, this worse problem exposes evidence consistent with the explanation of low-risk anomalies which claims that investors are constrained from taking the leverage necessary in the low-risk portfolios and inconsistent with the story that investors prefer the high-risk portfolios because they enjoy lottery-like investments. As such both average variance and volatility managed portfolios represent a realization of a practical limitation of one of the assumptions of modern portfolio theory and not a fundamental problem with the concept of the risk-premium.
%Decoupling the idosyncratic variance of individual returns from the variance of the market portfolio returns also sheds light on the risk-return trade-off dynamics following risk shocks. ONE MORE SENTENCE HERE (transmission of shock to AC?)

%Building on this work, I use the decomposition of portfolio risk and demonstrate that even greater performance can be had by managing equity investment by the average variance of the portfolio assets rather than just the variance of the portfolio itself and in doing so I shed light on a possible explaination of the low-risk anomaly while finding evidence against another.

Since the identification of a low-risk anomaly, or the absence of a risk premium, a large number of researchers have sought to identify a positive relationship between return variance and expected returns. \citet{CAMPBELL1987373}, \cite{FRENCH19873}, \citet{glosten_1993} and many, many others have found very insufficient evidence identifying a positive relationship between return variance and future returns. On the other hand, \citet{moreira_volatility-managed_2017} demonstrate that volatility managed portfolios which decrease leverage when volatility is high produce large alphas, increase Sharpe ratios, and produce substantial utility gains for mean-variance investors. These results hold across investment styles, e.g., value or momentum, and in different asset classes, e.g., equities and currency portfolios. The results even hold for positions which already seek to exploit the low-risk anomaly like the "betting-against-beta" strategy of \citet{frazzini_betting_2014}. Additionally, other researchers have applied similar variance or volatility management to specific assets or trading styles.\footnote{See, for example, \citet{barroso_momentum_2015} and \citet{kim_time_2016} for discussions of volatility managment of the momentum portfolio.} Practitioners call approaches like this risk-parity, and as of 2016, at least \$150 and as much as \$400 billion sits in these funds.\citep{steward_truly_2010,cao_risk_2016} The generation of higher expected returns without a commensurate increase in portfolio volatility implies that a representative investor should time asset volatility, demanding less return when its higher meaning the investor’s risk appetite must be higher in periods like recession and market downturns when its expected to be lower. In short, it seems risk does not equal reward. However, as with pigs, some volatility is "more equal" than other volatility.\footnote{\citep{orwell1946animal}}

\citet{pollet_average_2010} decompose market variance into the average correlation between pairs of assets and the average variance of the individual assets. This decomposition yields a series strongly related to both future volatility and higher returns, the average correlation of market assets series, and a one strongly related to future volatility but unrelated to returns, market asset average variance. Scaling investment in the market portfolio by the inverse of the previous period average variance should then improve return performance over and above managing investment the previous market variance since it avoids, somewhat, decreasing investment when the average correlation is high and its expected compensation. Average variance is also a better candidate for portfolio management because it has a better chance to pick up economic information sooner as individual assets respond to information about their own risk and expected returns with public disclosure while changes in aggregate materialize as data across assets are combined.\citep{campbell1997econometrics,campbell_have_2001} 

%Prior literature on the low-risk anomaly proposes two explainations. Either investors are leverage constrained and are unable to form the positions which generate the abnormal returns or investors have a preference for the extreme right tail, "lottery", returns which are not possible when employing risk managed strategies. \citet{asness_betting_2018} take a related approach to decomposing the betting-against-beta strategy of \citet{frazzini_betting_2014} into betting-against-correlation, BAC, and betting-against-variance factors, BAV, finding the BAC factor has a significant Fama-French five factor alpha but unrelated to behavior explainations of the low-risk anomally.\citep{fama_dissecting_2016} From the decomposition of daily market returns I find similar results about the more general low-risk strategy of volatility management. Management by either market variance or average asset variance actually increases the lottery-like returns of the market portfolio. Both strategies increase the Rachev ratios of the returns.

Using daily market returns from the Center for Research in Securities Prices, I follow \citet{pollet_average_2010} generating quarterly time series of stock market variance, SV, average correlation, AC, and average variance, AV. I then extend this by calculating the time series monthly. As expected, AV is strongly related to next period market variance and unrelated, at best insignificantly negatively, related to next period excess log returns. 
%As is standard, excess log returns are the difference between the log of one plus the CRSP market return and the log of one plus the risk free return. The monthly risk free return is the return on the one month treasury bill while the quarterly risk free rate is the return on the three month treasury bill, each series is collected from the St. Louis Federal Reserve Economic Data (FRED) website. 
From June 1962 to the end of 2016, encompassing the sample of  \citet{pollet_average_2010}, asset average variance is a significant in-sample predictor of higher daily market return variance, average asset variance, pairwise correlation, and lower log excess market returns at the monthly frequency. A one standard deviation increase in annualized average variance, from .77 to 1.62, is related to an increase in next month’s annualized market return variance of .545 of a standard deviation or a .22 increase. This makes next months expected market variance more than double the mean. AV remains a significant predictor or next month's SV even when this month's SV is included. A one standard deviation increase in AV also anticipates a .13 standard deviation, or .58 percentage point,  lower log excess return. This the following month's expected return negative. When both AV and SV are used to predict next month's return, AV is significant but SV is not. These support results at the quarterly frequency in \citet{pollet_average_2010}. However, over the full, 1926 to 2016, sample average variance is a signficant predictor of higher daily market return variance, average asset variance, pairwise correlation, but not log excess market returns, as shown in table \ref{tab:tab_in_sample_full}. This evidence supports the use of average variance as a leverage management signal. Scaling investment in the market by the inverse of average asset variance in the current month will pull funds out when the following month will have high market variance without sacrificing higher expected returns. It may, in fact, avoid negative returns. These results support the intuition from the work on volatility management in \citet{moreira_volatility-managed_2017} and portfolio average variance and correlation in \citet{pollet_average_2010} but in-sample regression use all available information and do not necessarily identify tradeable strategies.\citep{Welch2008}
%This relationship not only holds both in and out of sample but in both National Bureau of Economic Research (NBER) defined business cycle contractions and expansions. Out of sample forecast accuracy and encompassing tests show that AV is a superior predictor of both SV and log excess returns next period.

Investors can only make decisions using the limited information available to them at a given time. For example in June of 2007 investors and investment models could only use historical information up to that month; the effects of November 2008 on the variable coefficients do not affect the predictions for July 2007. \citet{moreira_volatility-managed_2017} demonstrate that market volatility is an effective market portfolio management technique across the CRSP data set from 1926 to 2015. To motivate average variance as a better market portfolio leverage signal, I run expanding window out-of-sample regressions using AV on market volatility, average variance, average correlation, and log excess returns. From June 1926 to December 2016 and using the predictions from SV as a benchmark, AV is a significantly better predictor or next month’s AV, AC and SV. It generates better \citet{Diebold1995} test statistics, significantly lower mean squared forecast errors judging by the MSE-F statistic from \citet{mccracken_asymptotics_2007} and the emcompassing test of \citet{harvey_tests_1998} show that average variance contains all of the predictive information in market variance. As with the in-sample results average variance serves investors at least as well as market variance and likely better in avoiding risk without giving up return. Out-of-sample testing always raises questions about model specification, recursive expansion versus rolling window parameter estimation, and choices of the training period and prediction window. Using the techniques in \citet{rossi_out--sample_2012}, the \citet{Diebold1995}, \citet{mccracken_asymptotics_2007}, and \citet{harvey_tests_1998} measures can be calculated robust to concerns on window selection for either an expanding or rolling specification.
% robust statistics I show that the superior performance of AV, against either the running historical mean or the current time period SV in either a rolling or expanding regression specification regardless of window choice.
The \citet{rossi_out--sample_2012}  robust statistics show that AV is a significantly better predictor than SV robust to the choice of window or regression specification. Thus, I expect managing leverage in the market portfolio by AV will produce substantially better return performance as compared to management by SV.

%The robust performance of AV translates into significant asset allocation gains. A strategy which weights investment in the market portfolio produces average annualized log excess returns of 7.8\% quarterly and 8.7\% monthly. The AV strategy produces quarterly and monthly Sharpe ratios, .48 and .59, each statistically significantly greater than those of the SV, volatility, management strategy. Additionally, AV generates higher values in more asymetric risk-return measures like the Sortino, Kappa, and upside potential ratios. Asset allocation performance is not all perfect, however, and the lower performance of AV in terms of Rachev ratio hints at possible explaination for the low-risk anomaly seen in volatility and average variance management. 
As promised by the out-of-sample results, AV is a substantially better than SV as a leverage management signal. Targeting the volatility of the buy and hold market portfolio return, as in \citet{moreira_volatility-managed_2017}, an investor without borrowing constraints earns an annualized average monthly return of 9.7\% from the average variance managed portfolio. This return is a statistically significant increase of more than 1\% over the SV managed portfolio; the difference in annualized average monthly returns grows to more than 2\% when practical leverage constraints are applied. With unconstrained borrowing, the AV managed portfolio has significantly better performance ratios like the symmetric Sharpe ratio, .52, and more asymmetric risk-return measures, e.g., Kappa 3 and Kappa 4 at .15 and .11 respectively. The advantage of managing with AV grows with risk aversion. The most risk-averse, $\gamma$ = 5, constrained investor sees a certainty equivalent return (CER) gain of more than 2\% annualized; this return represents a 27.4\% increase in utility nearly as substantial as the utility gains seen in return timing strategies. \citep{campbell1997econometrics} Targeting the volatility of the buy and hold return requires seeing into the future and knowing the buy and hold return volatility. However, this look-ahead does not affect performance ratios like the Sharpe ratio moreover, the significantly better performance of AV is robust to other choices of target volatility. The asset allocation gains are not all perfect, however, and the lower Rachev ratio performance of AV hints at a possible explanation for the low-risk anomaly seen in volatility and average variance management. 

”There is no such thing as a free lunch,” is not only a wonderfully pervasive adage, particularly loved by economists but a provable restriction on optimization problems. \citep{wolpert_no_1997} Here, AV provides no free lunch. The improved performance measured by expected log excess returns, Sharpe, Sortino, and Kappa ratios is betrayed by worse performance in Rachev ratio. The Rachev ratio measures the right tail reward value at play relative to the left tail value at risk. In short, it measures the ratio of expected lottery rewards and losses. The volatility managed market portfolio has higher lottery winning potential for each dollar of potential lottery loss compared to the average variance managed portfolio. Both SV and AV generate better Rachev ratios than the buy and hold return. Prior literature on the low-risk strategies proposes two explanations. Either investors are leverage constrained and unable to form the positions which generate the abnormal returns or investors have a preference for the extreme right tail, ”lottery,” returns which are not possible when employing risk-managed strategies. As AV and SV managed portfolios have better Rachev ratios than the buy and hold strategy, it would seem they are better lotteries to play. \citet{asness_betting_2018} take a related approach to decomposing the betting-against-beta strategy of \citet{frazzini_betting_2014} into betting-against-correlation, BAC, and betting-against-variance factors, BAV, finding the BAC factor has a significant Fama-French five-factor alpha but unrelated to behavior explanations of the low-risk anomaly. \citep{fama_dissecting_2016} From the decomposition of daily market returns, I find similar results about the more general low-risk strategy of volatility management. Management by either market variance or average asset variance increases the lottery-like returns of the market portfolio. Both strategies increase the Rachev ratios of the returns. This change is inconsistent with the notion that the low-risk anomaly is the result of a behavioral preference of investors for lottery-like returns. \citep{barberis_stocks_2008,brunnermeier_optimal_2007} Instead, the generation of higher log excess returns by the AV, SV, and betting-against-beta strategies support the notion that the low-risk anomaly arises from leverage constraints first suggested in \citet{black1972capital}. \citet{boguth_leverage_2018} link capital constraints and the betting-against-beta strategy through mutual fund betas, and more generally \citet{malkhozov_international_2017} show that international illiquidity predicts betting-against-beta returns. Testing directly for changes in the capital market line, first shown to flatten when leverage is costly by \citet{jylha_margin_2018}, I find a significant effect of many proxies for credit constraints on the returns to AV management but no significant effect for proxies of lottery preference. Financial intermediary leverage, bank credit growth, growth in margin investing and lending rates all affect the return on the AV managed portfolio and flatten the capital market line while proxies for high investor lottery preference like market capitalization of gaming industry stocks and extreme values of market daily returns are not. The evidence is consistent with leverage constraints affecting the returns to low-risk strategies, not market-wide lottery preferences. 

%Some change in the risk-return dynamic through average variance management was anticipated given the relationship between AV and next period SV and the lack of relationship between AV and next period log excess returns. Given that idosyncratic events are reflected in individual asset volatility first we should also be able to explain the better performance of AV in terms of timing and response. Using impulse-response functions from vector auto-regression (VAR) analysis its clear that AV responds more quickly and too a larger degree to shocks than SV. A one standard deviation positive shock to AV remains significant for at least five months. Consistent with the lack of relationship show in the regression analysis the response of excess log returns to the shock in AV is negative but insignificant in the first month turning and staying positive, but in significant there after. In response, AV driven investment weight drops the first month by more than 11\%. It recovers sharply in the second month but then slowly drifts back to its initial level over the course of the next 24 months. Similar to AV, a one standard deviation shock to SV remains significant for at least five months. Excess log return reaction is also similiar to the reaction to AV but less pronounced, again consistent with the return regression results. Investment weight reaction is different however, the SV strategy sheds up to 22\% of its investment in the market before recovering. This leaves the SV strategy further out of the market during the recovery to the shock than the AV strategy which results in lost returns.

By using the decomposition of market variance, I identify a better portfolio leverage management signal. Weighting investment leverage by the inverse of the average of individual asset variance, AV, rather than overall return variance, SV, is a new addition to the portfolio management literature letting investors capture better performance as measured by expected annualized returns. Investors also capture better investment ratios except for the Rachev ratio. The Rachev exposes the trade-off which investors must accept when managing risk by manipulating leverage conditional on average variance. The change in ratio contributes evidence against the lottery explanation of the low-risk anomaly in mean-variance analysis and supports the leverage constraints explanation. Furthur evidence shows that the returns and capital market line responds to proxies for tight lending conditions but not high lottery preference. This finding contrasts with conclusions reached in the study of cross-sectional low-risk anomalies which have been explained through the behavioral, lottery, channel.%By scarificing some potential extremely positive returns, investors using AV have access that better responds to risk shocks and participates more in the subsequent recovery. This supports prior literature which argues that individual assets should signal changes due to economic events before aggregate signals. All of this, of course, depends on the data and the construction of the individual and aggreagate signals.

The formation and analysis of the AV signal are relatively straightforward. It requires a few publically available datasets and a few considerations for the calculations at the monthly frequency. Most of the work is in the calculations required to show significant regression and portfolio performance. 