\noindent %Discussions on investment risk and the pursuit of maximum return took place over telegraph lines nearly a century before Markowitz brought formality to the notion of risk, portfolio construction and optimization in the 1950s.\footnote{Invented by Edward A. Calahan, an employee of the American Telegraph Company in 1867 and in wide spread use starting in the 1870s, the stock ticker provided the first reliable means of conveying up to the minute stock prices over a long distances and market participants have been discusing the relationship between returns, risk, and portfolios for at least as long.\citep{rutterford_financial_2016}} %The fundamental principal that there is a "risk-premium" such that more risky investment must generate greater returns to attack capital existed from the begining. While the fundamental idea that higher risk required higher return came before, Markowitz brought formality to the notion of risk, portfolio construction and optimization in the 1950s with modern portfolio theory and mean-variance analysis. 
%Modern portfolio theory states that portfolios with higher variance need to generate higher mean returns to attract rational investors.\citep{markowitz_portfolio_1952} However, several challenges have appeared to this foundational mean-variance claim. One of the most basic is commonly referred to as the ”low-risk” or ”low-volatility” anomaly. \citet{haugen_1972} found there was little to no evidence for a "risk-premium" for increased portfolio volatility. 
Between 400 billion and half a trillion dollars sits in funds utilizing a risk management, targeting, or risk-parity strategy. \citep{cao_risk_2016,verma_volatility-targeting_2018} Many of these funds employ cross-sectional trading strategies but as \citet{moreira_volatility-managed_2017} show\footnote{This holds even for low-risk anomaly strategies like the \citet{frazzini_betting_2014} betting against beta portfolio.} merely managing weight in a portfolio by its return volatility in time series produces greater expected returns and performance ratios over long periods. So, while Warren Buffet may not wholly support taking leverage to invest\footnote{"When leverage works, it magnifies your gains. Your spouse thinks you're clever, and your neighbors get envious,...but leverage is addictive. Once having profited from its wonders, very few people retreat to more conservative practices. And as we all learned in third grade — and some relearned in 2008 — any series of positive numbers, however impressive the numbers may be, evaporates when multiplied by a single zero. History tells us that leverage all too often produces zeroes, even when it is employed by very smart people." \citep{noauthor_warren_nodate}}, there appears to be a disconnect in the dynamics of the risk-return relationship allowing investors to move out of the market index when risk will not accompany return and leverage in when it will. In short, it seems risk does not equal reward. %However, as with pigs, some volatility is "more equal" than other volatility.\footnote{\citep{orwell1946animal}}
However, the central premise of modern portfolio theory is that the systematic risk demands compensation but, even for diversified portfolios, not all variance is systematic. \citep{markowitz_portfolio_1952} Zeroing in on this unsystematic uncompensated part better manages dynamic volatility, takes on less extreme leverage, generates higher returns at lower risk, costs less for the investor, benefits fund managers and informs us about changes in the systematic risk across the economy. 

%One of the central tenants of modern portfolio theory is that the variance of a diversified portfolio is systematic risk. \citep{markowitz_portfolio_1952} 
Portfolio variance is the standard measure of investment risk yet, empirically, the link between the variance of the market index and future returns is weak. A large number of researchers have sought to identify a positive relationship between variance and expected returns. \citet{haugen_1972,FRENCH19873,glosten_1993,ang_cross-section_2006} and many, many others have found insufficient evidence supporting a positive relationship between variance and future returns. 
%Some investors seek to exploit aspects of this disconnection through risk management and risk-parity funds.
%Portfolio variance is a function of asset variance and the pairwise covariance between the assets in the portfolio. 
%Only the component composed of the pairwise relationship between assets is systemic. Managing exposure to the individual asset variance is what drives the performance gains observed in the prior literature. Zeroing in on this uncompensated part shows a better way of dynamic volatility management which takes on less extreme leverage, generates higher returns at lower risk, costs less for the investor, benefits fund managers and informs us about changes in systematic risk across the economy.
%low volatility portfolios with unusually high expected returns present an opportunity for leveraged investing. 
%even at fixed levels of portfolio volatility timing investing can increase returns.
%\citet{moreira_volatility-managed_2017} show across investment strategies merely managing leverage in a portfolio by that portfolio's volatility produces greater expected returns and performance ratios. As \citet{rutterford_financial_2016} document, the discussion of the risk-return trade-off is at least a century old
%\footnote{Invented by Edward A. Calahan, an employee of the American Telegraph Company in 1867 and in wide spread use starting in the 1870s, the stock ticker provided the first reliable means of conveying up to the minute stock prices over a long distances and market participants have been discusing the relationship between returns, risk, and portfolios for at least as long.\citep{rutterford_financial_2016}} 
%and central to modern financial theory. Investment management that generates higher overall returns with no increase in total risk is of interest not only to investors but anyone seeking an understanding of the risk-return dynamic. 
%Unfortunately, the volatility managed portfolio requires extreme levels of leverage. I show that managing investment in the market portfolio by the inverse of the average variance of individual daily asset returns is not only a better performer, but cheaper, more practically implementable, successful globally and across asset classes.
%These improvements arise because returns to the average variance managed portfolio depend on the level of average pairwise correlation between individual asset daily returns which exposes a  systematic risk pricing dynamic which is obscured by the use of total market volatility.
%Portfolio variance is the standard measure of investment risk at the foundation of modern portfolio theory and optimization. \citep{markowitz_portfolio_1952} Portfolio variance is a function of asset variance and the pairwise covariance between the assets in the portfolio. 
%Investment management that generates higher overall returns with no increase in total risk is of interest not only to investors but anyone seeking an understanding of the risk-return dynamic. The \citet{moreira_volatility-managed_2017} result even holds for positions which already seek to exploit the low-risk anomaly like the "betting-against-beta" strategy of \citet{frazzini_betting_2014}. Additionally, other researchers have applied similar variance or volatility management to specific assets or trading styles.\footnote{See, for example, \citet{barroso_momentum_2015} and \citet{kim_time_2016} for discussions of volatility management of the momentum portfolio.}  The signal from the average variance of the individual assets is a better investment timing signal and it better manages exposure to un-systematic, uncompensated risk.
%These results seem to challenge the mean-variance notion of investment risk premium fundamentally. 
%The notion of a risk-return tradeoff is at least a centurty old\footnote{Invented by Edward A. Calahan, an employee of the American Telegraph Company in 1867 and in wide spread use starting in the 1870s, the stock ticker provided the first reliable means of conveying up to the minute stock prices over a long distances and market participants have been discusing the relationship between returns, risk, and portfolios for at least as long.\citep{rutterford_financial_2016}} and central to modern financial theory, so naturally, I address the problem of volatility management generating excess returns by making it worse. Thankfully, this worse problem exposes evidence consistent with the explanation of low-risk anomalies which claims that investors are constrained from taking the leverage necessary in the low-risk portfolios and inconsistent with the story that investors prefer the high-risk portfolios because they enjoy lottery-like investments.
%Since the early identification of a low-risk anomaly, the absence of a need to take on more risk for higher returns as in \citet{haugen_1972}, a large number of researchers have sought to identify a positive relationship between return variance and expected returns. \citet{CAMPBELL1987373,FRENCH19873,glosten_1993,ang_cross-section_2006} and many, many others have found insufficient evidence supporting a positive relationship between return variance and future returns. Some investors seek to exploit aspects of this disconnection through risk management and risk-parity funds. As of 2016, at least \$150 and as much as \$400 billion sat in these funds. \citep{steward_truly_2010,cao_risk_2016} 
%The generation of higher expected returns without a commensurate increase in portfolio volatility implies that the representative investor, holding the market, requires less equity premium in time of high volatility. %should time asset volatility. %and appears to demand less return when it is higher. meaning the investor’s risk appetite must be higher in periods like recession and market downturns when its expected to be lower. 
%In short, it seems risk does not equal reward. %However, as with pigs, some volatility is "more equal" than other volatility.\footnote{\citep{orwell1946animal}}
%However, the central premise of modern portfolio theory is that systematic risk is rewarded as investors are not able to avoid it and not all portfolio variance is systematic.
The total variance of stock market index returns (SV) is a function of both the variance of individual asset returns and the covariance between assets in the index. \citet{pollet_average_2010} show that the average pair-wise correlation (AC) of assets is the component most related to systematic changes in the economy and is the risk component compensated with higher returns. By responding to total variance, the prior literature only indirectly manages exposure to the individual asset variance (AV). 
%Zeroing in on this uncompensated part shows a better way of dynamic volatility management which takes on less extreme leverage, generates higher returns at lower risk, costs less for the investor, benefits fund managers and informs us about changes in systematic risk across the economy. 
%By managing monthly investment in the market index by the other component of total volatility, the average variance of the individual asset returns (AV), I generate higher average annualized returns and significantly better Sharpe, Sortino, Kappa ratios, and factor alphas. 

%In addition to identifying a better strategy for investors, decoupling AV from the SV sheds light on the risk-return trade-off dynamics of the market and allows timing investment across markets and asset classes when higher volatility will be compensated and pulling out when it will not.   %I then extend this by calculating the time series monthly. As expected, AV is strongly related to next period market variance and unrelated, at best insignificantly negatively, related to next period excess log returns. %As such both average variance and volatility managed portfolios represent a realization of a practical limitation of the free borrowing assumption of modern portfolio theory and not a fundamental problem with the concept of the risk-premium.
%Decoupling the idosyncratic variance of individual returns from the variance of the market portfolio returns also sheds light on the risk-return trade-off dynamics following risk shocks. ONE MORE SENTENCE HERE (transmission of shock to AC?)
%Building on this work, I use the decomposition of portfolio risk and demonstrate that even greater performance can be had by managing equity investment by the average variance of the portfolio assets rather than just the variance of the portfolio itself and in doing so I shed light on a possible explaination of the low-risk anomaly while finding evidence against another.
%Since the identification of a low-risk anomaly, or the absence of a required trade of more risk for higher returns, a large number of researchers have sought to identify a positive relationship between return variance and expected returns. \citet{CAMPBELL1987373}, \cite{FRENCH19873}, \citet{glosten_1993} and many, many others have found insufficient evidence identifying a positive relationship between return variance and future returns. On the other hand, \citet{moreira_volatility-managed_2017} demonstrate that volatility managed portfolios which decrease leverage when volatility is high produce large alphas, increase Sharpe ratios, and produce substantial utility gains for mean-variance investors. These results hold across investment styles, e.g., value or momentum, and in different asset classes, e.g., equities and currency portfolios. The results even hold for positions which already seek to exploit the low-risk anomaly like the "betting-against-beta" strategy of \citet{frazzini_betting_2014}. Additionally, other researchers have applied similar variance or volatility management to specific assets or trading styles.\footnote{See, for example, \citet{barroso_momentum_2015} and \citet{kim_time_2016} for discussions of volatility managment of the momentum portfolio.} As of 2016, at least \$150 and as much as \$400 billion sat in risk-parity funds attempting to exploit this disconnect between risk and reward.\citep{steward_truly_2010,cao_risk_2016} The generation of higher expected returns without a commensurate increase in portfolio volatility implies that a representative investor should time asset volatility, demanding less return when its higher meaning the investor’s risk appetite must be higher in periods like recession and market downturns when its expected to be lower. In short, it seems risk does not equal reward. However, as with pigs, some volatility is "more equal" than other volatility.\footnote{\citep{orwell1946animal}}
%\citet{pollet_average_2010} decompose market variance into the average correlation between pairs of assets and the average variance of the individual assets. This decomposition yields a series strongly related to both future volatility and higher returns, the average correlation of market assets series, and a one strongly related to future volatility but unrelated to returns, market asset average variance. Scaling investment in the market portfolio by the inverse of the previous period average variance should then improve return performance over and above managing investment the previous market variance since it avoids, somewhat, decreasing investment when the average correlation is high and its expected compensation. Average variance is also a better candidate for portfolio management because it has a better chance to pick up economic information sooner as individual assets respond to information about their own risk and expected returns with public disclosure while changes in aggregate materialize as data across assets are combined.\citep{campbell1997econometrics,campbell_have_2001} 
%Prior literature on the low-risk anomaly proposes two explainations. Either investors are leverage constrained and are unable to form the positions which generate the abnormal returns or investors have a preference for the extreme right tail, "lottery", returns which are not possible when employing risk managed strategies. \citet{asness_betting_2018} take a related approach to decomposing the betting-against-beta strategy of \citet{frazzini_betting_2014} into betting-against-correlation, BAC, and betting-against-variance factors, BAV, finding the BAC factor has a significant Fama-French five factor alpha but unrelated to behavior explainations of the low-risk anomally.\citep{fama_dissecting_2016} From the decomposition of daily market returns I find similar results about the more general low-risk strategy of volatility management. Management by either market variance or average asset variance actually increases the lottery-like returns of the market portfolio. Both strategies increase the Rachev ratios of the returns.
%Using daily market returns from the Center for Research in Securities Prices, I follow \citet{pollet_average_2010} generating quarterly time series of stock market variance, SV, average correlation, AC, and average variance, AV. I then extend this by calculating the time series monthly. As expected, AV is strongly related to next period market variance and unrelated, at best insignificantly negatively, related to next period excess log returns. 
%As is standard, excess log returns are the difference between the log of one plus the CRSP market return and the log of one plus the risk free return. The monthly risk free return is the return on the one month treasury bill while the quarterly risk free rate is the return on the three month treasury bill, each series is collected from the St. Louis Federal Reserve Economic Data (FRED) website. 
%AV is substantially better than SV as a investment management signal. 
I find that directly managing AV is better for investors. Keeping the volatility of the AV managed portfolio the same as the buy and hold market portfolio, as in \citet{moreira_volatility-managed_2017}, an investor without borrowing constraints earns an annualized average return of 9.7\%. This return is a statistically significant increase of more than one percentage point over the SV managed portfolio; the difference in annualized average monthly returns grows to more than two percentage points when practical leverage constraints are applied. With unconstrained borrowing, the AV managed portfolio has significantly better performance ratios like the symmetric Sharpe ratio, .52, and more asymmetric risk-return measures, e.g., Kappa$_{3}$ and Kappa$_{4}$ at .15 and .11 respectively. The advantage of managing with AV grows with risk aversion. The most risk-averse, $\gamma$ = 5, constrained mean-variance investor sees a certainty equivalent return (CER) gain of more than two percentage points annualized; this return represents a 27.4\% increase in utility nearly as substantial as the utility gains seen in return forecasting strategies. \citep{campbell1997econometrics} Stochastic dominance tests show the preference for AV management extends beyond mean-variance investors. Investors with a broad class of utility functions, including those with higher order risk and prospect theory preferences, prefer the AV management strategy to SV. 
%Targeting the volatility of the buy and hold return requires seeing into the future and knowing the buy and hold return volatility. However, this look-ahead does not affect performance ratios like the Sharpe ratio, moreover 
The significantly better performance of AV is also robust to other choices of target volatility; AV outperforms SV management when both portfolios seek to have 10\% or 12\% annual return volatility. 
%The asset allocation gains are not all perfect, however, and the lower Rachev ratio performance of AV hints at a possible explanation for the low-risk anomaly seen in volatility and average variance management. 

The AV and SV portfolio management strategies each generate a weight in the market index as a function of the daily returns of the market the prior month. Some of the investments demanded by the SV management strategy are unrealistic. The SV managed portfolio takes far more extreme leverage than AV and far more often. In addition to higher average monthly borrowing costs, 15.107 versus 11.411 basis points, the SV managed portfolio has higher turnover than AV. The AV managed portfolio generates lower transactions costs with a break-even transaction cost more than twice the SV managed portfolio. This makes the AV managed portfolio cheaper for investors in both borrowing and transaction costs while generating higher returns. The AV managed portfolio is also cheaper to insure. Drawdowns are shallower and shorter for the AV managed portfolio at 9\% versus 11.2\% and 10.5 versus 15 months on average. The drawdown profiles for the AV and SV managed portfolios also reveal that the AV managed portfolio benefits fund managers in addition to fund investors. The SV managed portfolio exposes a fund manager to twice the risk of a drawdown severe enough to threaten their job, or possibly shutter the fund, and would be nearly 91.7\% more expensive to insure against such a loss.

AV equity management is also better globally. The AV managed portfolio is a better performer in returns, ratios, drawdowns, and costs in 8 of 9 non-US markets tested. AV management generates significantly better performance in both developed and emerging markets including Brazil, Germany, France, India, Japan, and Great Britain. In China, AV management more than quintuples index returns over the buy and hold strategy. In Japan, AV management transforms negative buy and hold returns into a significant positive 1.375\% annualized from 1993 to 2015. AV management of investment in the Germany HDAX index increases annualized returns by almost 3.5 percentage points over SV management and results in a significant improvement in Sharpe ratio. AV index management works not only in the countries individually but for an equity portfolio diversified across the globe. AV management of the MSCI All Countries World Index (ACWI) produces higher annualized returns, 8.6\%, with a better Sharpe ratio, .551, than either SV management or the buy and hold strategy. AV managed MSCI ACWI also has shallower drawdowns and is robust to much higher trading costs. 

The identification of a better portfolio management signal is valuable, certainly for investors and fund managers. However, discovering the reason that AV outperforms SV makes a more significant contribution to our understanding of the market, the risk represented by portfolio variance, and the dynamics of the risk-return relationship. I run several regression specifications to examine the relationship of AV to future SV, AV, AC, and returns. The examination of regression results also allows for the correction of "volatility-feedback" effects, proposed by \citet{campbell_no_1992}, and makes it possible to judge the stability of the dynamic risk-return relationship mix revealed by AV and AC.

Using daily market returns from the Center for Research in Securities Prices (CRSP), I extend \citet{pollet_average_2010} generating monthly time series of SV, AC, and AV from August 1926 to December 2016. In results moved to the appendix for space, AV is a significant in-sample predictor of higher SV, AV, and AC.  A one standard deviation increase in annualized AV, from .77 to 1.62, is related to an increase in next month’s annualized SV of .627 of a standard deviation. The increase in SV makes it more than double its mean, .25 vs .56. AV remains a significant predictor of next month’s SV even when controlling for this month's SV. AV has a slightly higher $R^{2}$ value when predicting next month's SV and is better at predicting next month's AV. Controlling for this month's AV, SV is not a significant predictor of next month's AV. Neither SV nor AV is significantly related to future returns. However, as in the \citet{pollet_average_2010} results, AC is positively and significantly related to future returns. %These support results at the quarterly frequency in \citet{pollet_average_2010}. 
%However, over the full, 1926 to 2016, AV is a significant predictor of higher SV, SV, AC, but not log excess market returns.%, as shown in table \ref{tab:tab_in_sample_full}. 
This evidence explains the performance of AV as a leverage management signal. Scaling investment in the market with the inverse of AV in the current month will pull funds out when the following month will have high SV without sacrificing higher expected returns when AC is high. 
%These results support the intuition from the work on volatility management in \citet{moreira_volatility-managed_2017} and portfolio AV and AC in \citet{pollet_average_2010} but in-sample regression use all available information and do not necessarily identify tradeable strategies.\citep{Welch2008}
%This relationship not only holds both in and out of sample but in both National Bureau of Economic Research (NBER) defined business cycle contractions and expansions. Out of sample forecast accuracy and encompassing tests show that AV is a superior predictor of both SV and log excess returns next period.
These results support the existence of a long-run relationship between AC and future returns and support the conclusion that AV is not a compensated risk. However, when forecasting next month's excess log return over the full sample, the coefficient on AC is smaller and losses some significance compared to the post 1962 sample. 

It is, of course, difficult to test whether AV management better times investment to systematic risk. Still, I find several results that suggest AV is better than SV at signaling investors to times of better conditions across the economy. As always, asset pricing and portfolio tests are limited by the indices and proxies used. \citet{pollet_average_2010} show, theoretically, that AC is related to higher future returns and higher return correlation across the economy, but the relationship depends on the proportion of the market observed through the index proxy and the relationship of the stock market return and the aggregate wealth return. Thus, in a sub-sample where it is unlikely the daily returns used represent the whole market, and the stock market itself is known not to be an insignificant part of aggregate wealth, AC should not signal returns to systematic risk. This is the case for the CRSP data before 1962 which covers only the New York Stock Exchange (NYSE) and covers a period when the average investment was meager. \citep{taylor_2014} Indeed, in a placebo-like result, before 1962 AC does not predict next month's return even when controlling for AV.

The pre-1962 result supports the \citet{pollet_average_2010} argument that the information which AC, and thus AV, provides on the mix of compensated and uncompensated risk depends on the relationships of the index returns being used, the stock market, and aggregate wealth. This relationship should also be observable in portfolio performance. Using the ratios of market return to return on wealth and market capitalization to GDP per capita\footnote{I follow several papers sighted in section \ref{sec:market_cap} which use GDP per capita to control for cross-country wealth effects.} as proxies for the proportion of the stock market in aggregate wealth, AV works better in the cross-section of countries when the market is more related to aggregate wealth. Investors capture average annualized returns over 5\% with positive and significant factor alphas from a portfolio using AV managed weights long in countries with high and short in those with low ratios. The relationship of the stock market to aggregate wealth is systematically important to AV management supporting the pre-1962 results above.

The full in-sample results suggest AC positively forecasts next month's return and AV positively forecasts next month's risk without a positive relationship to returns. The appears to manifest in performance depending on relationships of AC and AV to the economy as a whole. However, as \citet{Welch2008} argue, forecasting relationships maybe unstable and quite sensitive to sample period choice. They may not respond dynamically with the limited information available to investors in real-time and may not explain or support the performance observed for any trading strategy at all. Furthermore, robust out-of-sample results would indicate that there is more information available on the real-time risk-return dynamics in AV than SV which would support the notion that AC is systematic and compensated while AV is un-systematic and uncompensated explaining the portfolio performance results.

Investors can only make decisions using the limited information available to them at a given time. For example in June of 2007 investors and investment models could only use historical information up to that month; the effects of November 2008 on the regression coefficients cannot affect the predictions for July 2007. %\citet{moreira_volatility-managed_2017} demonstrate that market volatility is a useful market portfolio management technique across the CRSP data set from 1926 to 2015. 
To explain why AV is a better real-time market portfolio leverage signal, I run expanding window out-of-sample regressions using AV on SV, AV, AC, and log excess returns. From July 1939 to December 2016 and using the predictions from SV as a benchmark, AV is a significantly better forecaster of next month’s AV, AC, and SV. It generates better \citet{Diebold1995} test statistics, significantly lower mean squared forecast errors judging by the MSE-F statistic from \citet{mccracken_asymptotics_2007} and the encompassing test of \citet{harvey_tests_1998} show that AV contains all of the forecasting information in SV. As with the in-sample results, AV serves investors at least as well as SV and likely better in avoiding risk without giving up return. Out-of-sample testing always raises questions about model specification, recursive expansion versus rolling window parameter estimation, choices of the training period and prediction window and the influence of specific periods. Using the techniques in \citet{rossi_out--sample_2012}, the \citet{Diebold1995} and \citet{harvey_tests_1998} measures can be calculated robust to concerns on window selection for either an expanding or rolling specification.
% robust statistics I show that the superior performance of AV, against either the running historical mean or the current time period SV in either a rolling or expanding regression specification regardless of window choice.
The \citet{rossi_out--sample_2012}  robust statistics show that AV is a significantly better predictor than SV robust to the choice of window or regression specification. %Thus, I expect managing leverage in the market portfolio by AV will produce substantially better return performance as compared to management by SV.
AV is a better investment timing signal than SV because it scales investment by future risk without giving up the future return. AV management does this because there is more information in AV on the the risk-return dynamic than in SV.

%The robust performance of AV translates into significant asset allocation gains. A strategy which weights investment in the market portfolio produces average annualized log excess returns of 7.8\% quarterly and 8.7\% monthly. The AV strategy produces quarterly and monthly Sharpe ratios, .48 and .59, each statistically significantly greater than those of the SV, volatility, management strategy. Additionally, AV generates higher values in more asymetric risk-return measures like the Sortino, Kappa, and upside potential ratios. Asset allocation performance is not all perfect, however, and the lower performance of AV in terms of Rachev ratio hints at possible explaination for the low-risk anomaly seen in volatility and average variance management. 
%As promised by the out-of-sample results, AV is substantially better than SV as a leverage management signal. Targeting the volatility of the buy and hold market portfolio return, as in \citet{moreira_volatility-managed_2017}, an investor without borrowing constraints earns an annualized average monthly return of 9.7\% from the average variance managed portfolio. This return is a statistically significant increase of more than 1\% over the SV managed portfolio; the difference in annualized average monthly returns grows to more than 2\% when practical leverage constraints are applied. With unconstrained borrowing, the AV managed portfolio has significantly better performance ratios like the symmetric Sharpe ratio, .52, and more asymmetric risk-return measures, e.g., Kappa 3 and Kappa 4 at .15 and .11 respectively. The advantage of managing with AV grows with risk aversion. The most risk-averse, $\gamma$ = 5, constrained investor sees a certainty equivalent return (CER) gain of more than 2\% annualized; this return represents a 27.4\% increase in utility nearly as substantial as the utility gains seen in return timing strategies. \citep{campbell1997econometrics} Targeting the volatility of the buy and hold return requires seeing into the future and knowing the buy and hold return volatility. However, this look-ahead does not affect performance ratios like the Sharpe ratio, moreover the significantly better performance of AV is robust to other choices of target volatility. 
%%The asset allocation gains are not all perfect, however, and the lower Rachev ratio performance of AV hints at a possible explanation for the low-risk anomaly seen in volatility and average variance management. 
%
%The AV and SV portfolio management strategies each generate a weight in the market portfolio as a function of the daily returns of the market the prior month. Some of the investments demanded by the AV mangaged portfolio maybe difficult to achieve, however the investments needed to make the SV management strategy work are simply problematic. The SV managed portfolio takes far more extreme leverage far more often. In addition to higher average monthly borrowing costs, 15.107 vs 11.411 basis points, the SV managed portfolio has higher turnover. The AV managed portfolio generates lower transactions costs with a break even transaction cost more than 2.5 times higher than the SV managed portfolio. This makes the AV managed portfolio cheaper for the investor in both borrowing and transaction costs while generating higher returns. The AV managed portfolio is also cheaper to insure. Drawdowns are shallower and shorter for the AV managed portfiolio at 9\% vs 11.2\% and 10.5 vs 15 months on average. The drawdown profiles for the AV and SV managed portfolios also reveal that the AV managed portfolio benefits fund managers in addition to fund investors. The SV managed portfolio exposes a fund manager to twice the risk of a drawdown sever enough to threaten their job, or possibly shutter the fund, and would be nearly 91.7\% more expensive to insure. 

%”There is no such thing as a free lunch,” is not only a wonderfully pervasive adage, particularly loved by economists but a provable restriction on optimization problems. \citep{wolpert_no_1997} Here, AV provides no free lunch. The improved performance measured by expected log excess returns, Sharpe, Sortino, and Kappa ratios is betrayed by worse performance in Rachev ratio. The Rachev ratio measures the right tail reward value at play relative to the left tail value at risk. In short, it measures the ratio of expected lottery rewards and losses. The volatility managed market portfolio has higher lottery winning potential for each dollar of potential lottery loss compared to the average variance managed portfolio. Both SV and AV generate better Rachev ratios than the buy and hold return. 

%%%%%%%%%%%%%%%%%%%%%%%%%%%%%%

%Prior literature on low-risk strategies, primarily cross-sectional, proposes two explanations. Either investors are leverage constrained and unable to form the positions which generate the abnormal returns or investors have a preference for the extreme right tail, ”lottery,” returns which are not possible when employing risk-managed strategies. 
%%As AV and SV managed portfolios have better Rachev ratios than the buy and hold strategy, it would seem they are better lotteries to play. 
%\citet{asness_betting_2018} take a related approach to decomposing the betting-against-beta strategy of \citet{frazzini_betting_2014} into betting-against-correlation, BAC, and betting-against-variance factors, BAV, finding the BAC factor has a significant Fama-French five-factor alpha but unrelated to behavior explanations of the low-risk anomaly. \citep{fama_dissecting_2016} From the decomposition of daily market returns, I find similar results about the more general low-risk strategy of volatility management. However, the lottery preference explaination, in this context, requires that the buy and hold market portfolio exhibit more lottery-like features than either AV or SV managed portfolios to compensate for the lost returns. Management by either market variance or average asset variance actually increases the lottery-like returns of the market portfolio. Each portfolio has larger maximum daily returns statistics, proxies used for the lottery attractiveness of an investment, than the buy and hold market. Both strategies increase the Rachev ratios of the returns. The numerator of the Rachev ratio is the expected value of extreme gains making this change is inconsistent with the notion that the lower market returns are the result of a behavioral preference for market's lottery. \citep{barberis_stocks_2008,brunnermeier_optimal_2007} Further, as \citet{barberis_stocks_2008} show that lottery preferences lead to mispricing which cannot be arbitraged away under resonable circumstances lowering the equilibrium market return. This flattens the capital market line, CML, and decreases rather than increases the seperation in market and managed returns. This effect shows up significantly in regressions testing the slop and intercept of the CML using maximum daily return statistics and gaming industry market capitalization as proxies for high lottery preferences. These CML tests come from \citet{jylha_margin_2018} who shows that lending constraints flatten the CML using margin equity requirement changes. This was first put forward by \citet{black1972capital} and appears in \citet{frazzini_betting_2014}. Again, tigher borrowing conditions flatten the CML and the returns for all market leverage portfolios are closer not further apart. I confirm the \citet{jylha_margin_2018} results using additional proxies for lending constraints like financial intermediary leverage, callable loan interest rates and margin borrowing levels. 
%
%Investors appear to be over investing in times of high market volatility and the explainations with cross-sectional bite, lottery preferences and lending constraints, do not work theorectially or empirically. Yet, what appears to a lower demand equity premium demanded for higher market volatility is actually the dynamics of the market return and the correlation of asset returns. 

%%%%%%%%%%%%%%%%%%%%%%%%%%%%%%%

%Instead, the generation of higher log excess returns by the AV, SV, and betting-against-beta strategies support the notion that the low-risk anomaly arises from leverage constraints first suggested in \citet{black1972capital}. \citet{boguth_leverage_2018} link capital constraints and the betting-against-beta strategy through mutual fund betas, and more generally \citet{malkhozov_international_2017} show that international illiquidity predicts betting-against-beta returns. Testing directly for changes in the capital market line, first shown to flatten when leverage is costly by \citet{jylha_margin_2018}, I find a significant effect of many proxies for credit constraints on the returns to AV management but no significant effect for proxies of lottery preference. Financial intermediary leverage, bank credit growth, growth in margin investing and lending rates all affect the return on the AV managed portfolio and flatten the capital market line while proxies for high investor lottery preference like market capitalization of gaming industry stocks and extreme values of market daily returns are not. The evidence is consistent with leverage constraints affecting the returns to low-risk strategies, not market-wide lottery preferences. 

%Some change in the risk-return dynamic through average variance management was anticipated given the relationship between AV and next period SV and the lack of relationship between AV and next period log excess returns. Given that idosyncratic events are reflected in individual asset volatility first we should also be able to explain the better performance of AV in terms of timing and response. Using impulse-response functions from vector auto-regression (VAR) analysis its clear that AV responds more quickly and too a larger degree to shocks than SV. A one standard deviation positive shock to AV remains significant for at least five months. Consistent with the lack of relationship show in the regression analysis the response of excess log returns to the shock in AV is negative but insignificant in the first month turning and staying positive, but in significant there after. In response, AV driven investment weight drops the first month by more than 11\%. It recovers sharply in the second month but then slowly drifts back to its initial level over the course of the next 24 months. Similar to AV, a one standard deviation shock to SV remains significant for at least five months. Excess log return reaction is also similiar to the reaction to AV but less pronounced, again consistent with the return regression results. Investment weight reaction is different however, the SV strategy sheds up to 22\% of its investment in the market before recovering. This leaves the SV strategy further out of the market during the recovery to the shock than the AV strategy which results in lost returns.

%By using the decomposition of market variance, I identify a better portfolio leverage management signal. Weighting investment leverage by the inverse of the average of individual asset variance, AV, rather than overall return variance, SV, is a new addition to the portfolio management literature letting investors capture better performance as measured by expected annualized returns. Investors also capture better investment ratios except for the Rachev ratio. The Rachev exposes the trade-off which investors must accept when managing risk by manipulating leverage conditional on average variance. The change in ratio contributes evidence against the lottery explanation of the low-risk anomaly in mean-variance analysis and supports the leverage constraints explanation. Furthur evidence shows that the returns and capital market line responds to proxies for tight lending conditions but not high lottery preference. This finding contrasts with conclusions reached in the study of cross-sectional low-risk anomalies which have been explained through the behavioral, lottery, channel.%By scarificing some potential extremely positive returns, investors using AV have access that better responds to risk shocks and participates more in the subsequent recovery. This supports prior literature which argues that individual assets should signal changes due to economic events before aggregate signals. All of this, of course, depends on the data and the construction of the individual and aggreagate signals.

%It is, of course, a difficult to test whether AV management better times investment to systematic risk. However, I find several results that suggest that AV is better than SV at signaling investors to times of better conditions across the economy. 
%\citet{pollet_average_2010} show that AC is related to higher future returns and higher return correlation across the economy, but the relationship depends on the proportion of the market observed and the relationship of the stock market return and the aggregate wealth return. Thus, in a sub-sample where it is unlikely the daily returns represent the whole market and the stock market itself is known not to be a significant part of aggregate wealth, AC should not significantly predict returns. This is the case for the CRSP data before 1962 which covers only the New York Stock Exchange (NYSE) and covers a period when the average investment was meager. \citep{taylor_2014}  
%It seems necessary to demonstrate that AV  management is not just a US market phenomenon. Indeed, the AV managed portfolio is a better performer in return, ratios, drawdowns, and costs in 8 of 9 non-US markets tested. AV management generates significantly better performance in both developed and emerging markets including Brazil, China, Germany, France, India, Italy, Japan, and Great Britain. For example, returns in China are more than quintupled over the buy and hold strategy by AV management. 
%Taking market capitalization to GDP as a proxy for the proportion of the stock market in aggregate wealth, AV works better in the cross-section of countries when the market is more related to aggregate wealth. Investors capture average annualized returns over 11\% with a Sharpe ratio above .77 from a portfolio using AV managed weights long in countries with high market capitalization to GDP ratio and short those with a low ratio. Moreover, the portfolio has a positive and significant Fama-French five-factor plus Momentum alpha meaning the relationship of the stock market to aggregate wealth is systematically important to AV management as seen in the pre-1962 results above. 
%AV equity portfolio management works not only in the countries individually but for an equity portfolio diversified across the globe. AV management of the MSCI All Countries World Index (ACWI) produces higher annualized returns, 8.6\%, with a better Sharpe ratio, .551, than either SV management or the buy and hold strategy. AV managed MSCI ACWI also has shallower drawdowns and is robust to much higher trading costs. Additionally, AV management works across asset classes.

A sharp test of AV management's ability to signal investors to changes in systematic risk would be its ability to manage returns from other asset classes. \citet{moreira_volatility-managed_2017} find that equity SV is not a useful signal for currency investment. However, equity AV works as a management signal not only for currency investments but real estate and bond market investment as well. Returns, ratios, drawdowns, and costs are all the better when managing investment in multiple currency, real estate, and bond market indices by the inverse of the average variance of equities compared to both the buy and hold, and equity SV management. I test this using four currency indices, the S\&P US Real Estate Investment Trust (REIT) index, and the Bloomberg US Universal Bond index. All series are from 2005 to 2015. The buy and hold and SV management strategies have annualized returns of -2.07\% and -.36\% for the Deutsche Bank Currency Carry Index. This carry trade index uses one of the oldest currency trade strategies based on the work on uncovered interest rate parity from \citet{noauthor_speculative_nodate,fama_forward_1984} and studied by \citet{lustig_cross_2007,brunnermeier_carry_nodate,burnside_carry_2011} among others. Management using equity AV, generates a 1.44\% annualized return from the carry trade index. SV management of the Carry Index results in a -.361\% return.  Equity AV is very successful at managing REIT investment over the sample period nearly doubling the SV managed returns, 26.7\% to 15\% and the AV management return of 3.95\% and Sharpe ratio of 1.17 are significant improvements over SV for the bond index. The average variance component of equity market variance is better globally and across asset classes supporting the argument that compensated systematic risk is higher when average variance is lower. 

Using the decomposition of market variance, I identify a better portfolio investment management signal. Weighting investment leverage by the inverse of AV, rather than SV, is a new addition to the portfolio management literature letting investors capture better performance as measured by expected annualized returns,  performance ratios, costs, and utility gains. My results complement recent work including the \citet{moreira_volatility-managed_2017} and \citet{hocquard_constant-volatility_2013} who use volatility timing and a constant target to manage portfolio tail risk. The returns to the AV managed portfolio improve our understanding of the risk-return dynamic by showing that the time variation in the mix between the systematic and unsystematic parts of the market index variance is an essential dimension of risk and optimal investment over time.  Recent work in the risk-return dynamic literature has also attempted to generate more systematic signals from components of portfolio volatility or across many portfolios. \citet{gonzalez-urteaga_cross-sectional_2016} examine the average volatility across many portfolios and the risk premium compensation. \citet{bollerslev_risk_2017} generate a better-realized volatility risk signal from the high-frequency data of many assets demonstrating better forecasting performance and utility gains for investors. 

The formation and analysis of the AV signal come from the fundamental understanding of risk, portfolio variance and returns. This work requires a few publicly available data sets and a few considerations for the calculations at the monthly frequency. Most of the work is in the calculations required to show significance in portfolio performance and in the regressions which establish the relationship of AV, risk, and future returns.