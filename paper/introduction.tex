\noindent Invented in 1867\footnote{Invented by Edward A. Calahan, an employee of the American Telegraph Company.} and in wide spread use starting in the 1870s, the stock ticker provided the first reliable means of conveying up to the minute stock prices over a long distances and market participants have been discusing the relationship between returns, risk, and portfolios for at least as long.\cite{rutterford_financial_2016} The fundamental principal that there is a "risk-premium" such that more risky investment must generate greater returns to attack capital existed from the begining. While the fundamental idea that higher risk, measured in some way, required higher return. Markowitz brought formality to the notion of risk, portfolio construction and optimization in the 1950s with modern portfolio theory and mean-variance analysis which stated that portfolios with higher variance should have higher returns.\cite{markowitz_portfolio_1952} However, several challanges have appeared to this foundational mean-variance claim. One of the most basic is commonly refered to as the "low-risk" or "low-volatilty" anomaly. \cite{haugen_1972} found there was little to no evidence for a "risk-premium" for increased portfolio volatility and it was indeed easy to construct portfolios with lower volatility which earned greater returns than those with higher volatilites over a span of 20 years. A more mordern and comprehensive treatment of the low-risk issue can be found in \cite{moreira_volatility-managed_2017} who show, across investment strategies and asset classes that simply managing leverage in a portfolio by its volatility produces greater expected returns and performance ratios than the underlying portfolio. As such, these results fundamentally challenge the mean-variance notion of investment risk premium. This risk-return trade off is central to modern financial theory, so naturally I want to address this problem by making it worse. By managing the market portfolio using the average variance of the individual assets in the prior quarter or month, I generate higher expected returns and significantly better performance ratios than by managing using the market portfolio volatility. Thankfully, at least in the equity market setting, this worse problem exposes evidence consistent with the leverage explaination of the low-risk anomaly and inconsistent with the lottery story, while shedding light on the risk-return trade-off dynamics in structural asset pricing theories.

%Building on this work, I use the decomposition of portfolio risk and demonstrate that even greater performance can be had by managing equity investment by the average variance of the portfolio assets rather than just the variance of the portfolio itself and in doing so I shed light on a possible explaination of the low-risk anomaly while finding evidence against another.

\cite{moreira_volatility-managed_2017} demonstrate that volatility managed portfolios which decrease leverage when volatility is high produce large alphas, increase Sharpe ratios, and produce large utility gains for mean-variance investors. These results hold across investment styles, e.g. value or momentum, and in different asset classes, e.g. equities and currency portfolios. The results even hold for positions which already seek to exploit the low-risk anomaly like the "betting-against-beta" strategy of \cite{frazzini_betting_2014}. Additionally, other researchers have applied similiar variance or volatility management to specific assets or trading styles.\footnote{See, for example, \cite{barroso_momentum_2015} and \cite{kim_time_2016} for discussions of volatility managment of the momentum portfolio.} Practioners call approaches like this rik-parity and as of 2016 at least \$150 and as much as \$400 billion dollars was invested in risk-parity funds.\cite{steward_truly_2010,cao_risk_2016} The generation of greater expected returns without a equivalent increase in portfolio volatility implies that a representative investor should time asset volatility, demanding less return when its higher meaning the investor's risk appetite must be higher in periods like recession and market downturns when its expected to be lower. In short, it seems risk does not equal reward. However, some volatility is more equal than other volatility.

\cite{pollet_average_2010} decompose market variance into the average correlation between pairs of assets and the average variance of the individual assets yeilds a series strongly related to both future volatility and higher returns, average correlation, and a series strongly related to future volatility but unrelated to returns, average variance. Scaling investment in the market portfolio by the inverse of previous month’s average variance should then improve return performance over and above those managing investment by the previous monthly SV since it avoids, somewhat, decreasing investment when average correlation is high and higer future returns are expected. Average variance is also a better candidate for portfolio management because it has a better chance to pickup economic information sooner as individual assets respond to informtion about its own risk and expected returns as it's made public while changes in aggregate materalize as information across assets is combined. \cite{asness_betting_2018} takes a related approach to decomposing the betting-against-beta strategy of \cite{frazzini_betting_2014} into betting-against-correlation, BAC, and betting-against-variance factors, BAV, finding the BAC factor has a significant Fama-French five factor alpha but unrelated to behavior explainations of the low-risk anomally.\cite{fama_dissecting_2016}

Using daily market returns from the Center for Research in Securities Prices, I follow \cite{pollet_average_2010} generating quarterly and monthly time series of stock market variance, market variance, average correlation, AC, and average variance, AV.