In the single time period capital asset pricing model, CAPM, world, the market portfolio is the optimal risky portfolio and all investors hold some combination of the market and the risk-free rate. \cite{sharpe_capital_1964,lintner_valuation_1965} At the end of each month AV managed investors are taking a weight $\omega_{t}$ in the market portfolio proportional to the variance of the daily return of the market over the last month. Thus, in month t+1 the return to the AV management strategy is just:
\begin{equation} 
	r_{AV,t+1} = \omega_{t}r_{M,t+1}
\end{equation}
where $r_{M,t+1}$ is the return on the CRSP market portfolio. Both expected return and standard deviation scale directly with the investment weight so for each single time period the AV managed portfolio lines on the Capital Market Line, CML, defined by the market portfolio and risk-free rate. 
As a consequence, AV managed returns should perfectly fit the standard CAPM regression.
\begin{equation} \label{eq:eq_returns}
	r_{AV,t} = \alpha_{t} + \beta_{1}r_{M,t} + \epsilon_{t}
\end{equation}
However, when investors are subject to externatlities or frictions their optimal allocation changes and in equilbirium the market portfolio changes. When the externality is the utility from gambling through investment in lottery-like stocks the increased demand drives changes in prices and lower expected returns and as \citet{barberis_stocks_2008} show if shorting stocks is costly this mispricing cannot be arbitraged away, equilibrium holdings are heterogenous and the market portfolio return is lower. \citet{kumar_who_2009} and \citet{fong_risk_2013} also find lower returns in the presence of lottery preferences since investors are recieving compensation for the variance in returns through something other than expected return they are willing to accept the lower expected returns. When investment calls for high levels of leverage, demand for lending makes it costly. As \citet{black1972capital} first shows this decreases the excess return of portfolios leveraged into the market portfolio and flattens the CML. Indeed this affect on the CML is a method to identify which is affecting investor desicions lottery preferences or lending constraints. \citet{jylha_margin_2018} identifies the decrease in the slope of the CML given changes to the amount of equity investors are forced to deposit when taking loans to invest. So, both lottery preferences and lending constraints should flatten the CML as shown by \citet{barberis_stocks_2008}, \citet{kumar_who_2009}, \citet{fong_risk_2013}, \citet{bali_lottery-demand-based_2017}, \citet{black1972capital}, \citet{frazzini_betting_2014}, \citet{jylha_margin_2018}, however this effect occurs in equilibrium making it difficult to measure. %Following \citet{jylha_margin_2018} makes the argument that changes in equity requirements are exogenous and thus lending constraint effects on the CML are identified. This is not necessary. 
As \citet{bresnahan_oligopoly_1982} demonstrates, in a slightly different context, despite moving from one equilibrium to another the factor affecting returns can be measured with the interaction term of the market return and the proxy for the factor being tested. Figure \ref{fig:fig_cml} motivates the intuition graphically, as in \citet{bresnahan_oligopoly_1982}.
\bigskip
\centerline{\bf [Place Figure~\ref{fig:fig_cml} about here]}
\bigskip
While in any single month the relationship between the return on the market portfolio and the AV managed portfolio is fixed by the weight suggested by the prior average variance of the daily market returns, the interaction of the market return with, for example, a proxy for the lending constraints in the market will give us a measure of the effect the lending constraint has on the returns to AV. Using the regression specification:
\begin{equation}
		r_{AV,t} = \alpha_{t} + \beta_{1} r_{M,t} + \beta_{2} r_{M,t}x_{t} + \beta_{3}x_{t} + \boldsymbol{\beta}_{t}\boldsymbol{\chi}_{t} + \epsilon_{t}
\end{equation}
where x is a proxy for either lottery preference or lending constraint and $\chi$ is the vector of other controls, changes not only in the location capital market line but in any rotation around the market portfolio are captured. Changes in the rotation of the CML around the market change the returns to the AV managed portfolio; it only remains to specify what we expect to see from the regression. In the extreme, investors are subject to borrowing conditions which make them all indifferent, or worse, to borrowing. At this point the CML has rotated so that the portion beyond the market portfolio is completely flat; the cost of borrowing is the same as an additional return. In the extreme, there is sufficient lottery dividend that investors forego any excess return from the market portfolio and become indifferent between investing in the market with its low return and lottery dividend and the risk-free rate. Again, the CML has completely flattened. When this occurs the returns to the AV managed portfolio are the same as the returns to the market portfolio and from equation \ref{eq:eq_returns} when these two returns are the same $\beta_{1}$ will equal one. Hence, if $\beta_{1}$ is less than one $\beta_{2}$ should be positively significant and if $\beta_{1}$ is greater than one $\beta_{2}$ should be negatively significant.