\centerline{\bf ABSTRACT}

\begin{doublespace}  % Double-space the abstract and don't indent it
%	Managed portfolios that take less risk when volatility is high produce large alphas,
%
%increase Sharpe ratios, and produce large utility gains for mean-variance investors.
%
%We document this for the market, value, momentum, profitability, return on equity,
%
%investment, and betting-against-beta factors, as well as the currency carry trade.
%
%Volatility timing increases Sharpe ratios because changes in volatility are not offset
%
%by proportional changes in expected returns. Our strategy is contrary to conventional
%
%wisdom because it takes relatively less risk in recessions. This rules out typical risk-
%
%based explanations and is a challenge to structural models of time-varying expected
%
%returns.
  \noindent While taking leverage to invest may be seen as risky, timing leverage can produce higher returns and timing by the right signal can produce significantly higher returns. Compared to a portfolio that manages investment in the market by the variance of daily market returns, a portfolio that manages by the average variance (AV) of the individual assets in the market portfolio produces utility gains by generating significantly higher average returns with significantly better Sharpe, Sortino, and Kappa ratios. The average variance managed portfolio is also cheaper and more practically implementable than the variance managed portfolio. This performance gain arises because AV is uncompensated, non-systemic risk using it to time leveraged investment reduces investor exposure when risk does not accompany reward and increases investment when it does. The AV managed strategy works across countries %and asset classes 
  supporting the argument that reducing AV exposure better aligns investment with systemic risk.
\end{doublespace}