%The regression and equity portfolio evidence is consistent with the explanation that AV management outperforms SV management because AV better signals investors to changes in the mix of systematic and unsystematic risk found in SV. There remains one additional test based results from \citet{moreira_volatility-managed_2017} which will show that AV management limits exposure to unsystematic risk. If investment management by the average variance of equity market returns better times leverage with changes in systematic risk then it should be able to generate better results for investors across asset classes.
\citet{moreira_volatility-managed_2017} show that equity SV does not perform as a currency management signal. If equity AV serves as a signal for other assets, this will demonstrate a stark contrast and show AV signals changes in the mix of compensated and uncompensated risk. To test AV and SV management in other asset classes, I use the world market capitalization weighted AV and SV values constructed before and several asset class indices. These indices include the Bloomberg US Dollar Spot Index, the Deutsche Bank Currency Return Index, The Deutsche Bank Currency Carry Index, the Deutsche Bank Currency Momentum Index, the S\&P U.S. Real Estate Investment Trust (REIT), the Bloomberg Commodity Index, and the Bloomberg US Universal Bond index. All series start in July of 2005 and end in December 2015.

\bigskip
\centerline{\bf [Place Table~\ref{tab:tab_altPerf1} about here]}
\bigskip

Panel (a) of table \ref{tab:tab_altPerf1} shows the average annualized average returns and Sharpe ratios for each of the different asset class indices. The results for the currency indices support the conclusion in \citet{moreira_volatility-managed_2017}. Management of currency investments using SV does not work. In many cases returns and Sharpe ratios are worse, e.g. the Deutsche Bank Currency and Currency Momentum indices. In contrast, management by equity AV is better than both the buy and hold and SV management strategies for currencies and real estate investments. As shown in table \ref{tab:tab_correlations}, this is not driven by high correlation between world equity returns and currency returns. Negative returns on the Bloomberg Dollar Spot, Deutsche Bank Currency, and Deutsche Bank Currency Carry indices become positive when managed by AV. The AV managed real estate investment nearly quintuples returns to the S\&P REIT index, 26.7\% vs 5.3\% and the Sharpe ratio is more than four times as high, .995 vs .198. Equity AV management is able to generate significant improvements over SV management in the bond market. The annualized return of 3.951\% is a 2.5 percentage point improvement over SV and the 1.168 Sharpe ratio is more than twice that of the equity SV managed investment. Equity AV management does not appear to improve the return or Sharpe ratio to investment in commodities. \footnote{While AV does outperform SV management for the Bloomberg Commodity index, it generates a lower return and worse Sharpe ratio than the buy and hold strategy. This is certainly disappointing, however it may be expected. \citet{gorton_facts_2006,buyuksahin_commodities_2008,NBERw21243} document the lack of relationship between equity and commodity returns. This is likely the genesis for the advice suggesting commodities as a portfolio hedge. Additionally, as \citet{erb_conquering_nodate} show commodity returns are linked to income returns and not prices and link to wealth in a different manner than equities. Thus, it is possible that the systematic risk to investor wealth equity AV management times is different from the wealth risk related to the performance of commodity investments.}

Panel (b) of table \ref{tab:tab_altPerf1} show the result of using either the SV or AV calculated from the currency and bond indices to manage investment in the MSCI World index.\footnote{Constituent return data is not available, to me, for the other indices.} While SV is able to generate improvements over the buy and hold strategy, again for July 2005 to December 2015, the benefits for using currency or bond AV are greater. The average variance calculated from either currencies, equities, or bonds can be used to manage investment in any of the asset classes whereas the same cannot be said for SV. Equity SV does not improve, even over the buy and hold strategy for all assets and it is not better than AV for any of the assets.

%\bigskip
%\centerline{\bf [Place Table~\ref{tab:tab_altPerf2} about here]}
%\bigskip

Appendix table \ref{tab:tab_altPerf2} shows in all cases, even for commodities, AV management results in better drawdown statistics. The average depths are shallower, lengths and recoveries shorter. AV management cuts the average commodity loss by more than 50\%, reducing a 26.6\% loss to 10.1\%; the average length by nearly 70\%, 39.3 to 12.2 months; and the average recovery time by more than 50\%, 4.3 to 2.1 months. For all statistics SV management makes the drawdown statistics worse. Table \ref{tab:tab_altPerf3} also shows the AV managed investments have less turnover and volatility than SV. This means that they are able to tolerate higher trading costs, have higher break even costs, and lower borrowing costs.%So, while it may not generate higher returns or better Sharpe ratios, AV management of a commodity investment may allow better sleep at night.

%\bigskip
%\centerline{\bf [Place Table~\ref{tab:tab_altPerf2} about here]}
%\bigskip

%Finally, shown in appendix table \ref{tab:tab_altPerf3}, equity AV management of the other asset classes is cheaper than SV management. The equity AV and SV management signals carryover the same relationship from prior performance tables, the AV managed investments have less turnover and thus incur lower transaction costs than SV. This means that they are able to tolerate higher trading costs and have higher break even costs. Neither strategy improves commodity investment returns so neither strategy can tolerate even zero trading costs when managing the Bloomberg commodity index. However, the SV managed investments in the Deutsche Bank Currency and Currency Momentum indices cannot even tolerate zero trading costs while the break even costs for all of the AV managed investments are reasonable. 

Clearly, the use of equity AV to manage investments in other asset classes is better than the use of equity SV. The difference clearly indicate something qualitatively different in AV management compared to SV. Returns, Sharpe ratios, drawdown statistics and trading costs are better in currency, real estate, and bond investments. %There may be no benefit to AV management in commodities over the buy and hold, but SV management appears to be clearly worse than the buy and hold and AV management is a significant improvement over SV. This is likely due to commodity returns stemming from a risk unrelated to the risk identified by equity returns as detailed in the diversification and return generation arguments in \citet{gorton_facts_2006,buyuksahin_commodities_2008,NBERw21243,erb_conquering_nodate}. %The evidence from global equity and across asset class investments suggests that AV management better times investment positions to compensated systematic risk.
As the argument is AV signals dynamic change in the relationship of systematic risk and return, it should also be possible to use the AV of returns from other assets to manage equity returns. I test this using trade volume weighted AV for the 10 largest currency pairs and top 500 bond returns from Financial Industry Regulatory Authority's (FINRA's) Trade Reporting and Compliance Engine (TRACE) through WRDS.\footnote{A full history of currency data was not available. I used volume snapshots found on \citet{most_2013,most_2015,most_2016,most_2018}. For robustness, I test equally weighted AV from both currency and bond returns and the performance of equity returns is still better.} As table \ref{tab:tab_altPerf1} shows using either AV calculated from either currency or bond returns equity returns are higher than the buy and hold strategy. The average annualized returns nearly double from 4.914\% to 8.416\% and 8.582\%. The cross asset performance supports the notion that AV signals changes in the amount of systematic risk across the economy. Additionally, AV returns should display dependence on market $\beta$ and aggregate wealth as implied by \citet{pollet_average_2010}.