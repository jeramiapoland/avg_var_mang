The above evidence is consistent with the explanation that AV management outperforms SV management because AV better signals investors to changes in the mix of systematic and unsystematic risk found in SV. There remains one additional test based on a result from \citet{moreira_volatility-managed_2017} which will show that AV management limits exposure to unsystematic risk. If investment management by the average variance of equity market returns better aligns leverage with changes in systematic risk then it should be able to generate better results for investors across asset classes. This will demonstrate a stark contrast to \citet{moreira_volatility-managed_2017} whose results show that equity variance is not an investment timing signal for currency returns. To test AV and SV in other asset classes, I use the world market capitalization weighted AV and SV values constructed before and several asset class indices. These indices include the Bloomberg US Dollar Spot Index, the Deutsche Bank Currency Return Index, The Deutsche Bank Currency Carry Index, the Deutsche Bank Currency Momentum Index, the S\&P U.S. Real Estate Investment Trust (REIT), and the Bloomberg Commodity Index. All series start in July of 2005 and end in December 2015.

\bigskip
\centerline{\bf [Place Table~\ref{tab:tab_altPerf1} about here]}
\bigskip

Table \ref{tab:tab_altPerf1} shows the average annualized average returns and Sharpe ratios for each of the different asset class indices. The results for the currency indices support the conclusion in \citet{moreira_volatility-managed_2017}. Management of currency investments using the total volatility of equity returns does not work. In many cases returns and Sharpe ratios are worse, e.g. the Deutsche Bank Currency and Currency Momentum indices. In contrast, management by equity AV is better than both the buy and hold and SV management strategies for currencies and real estate investments. As shown in table \ref{tab:tab_correlations}, this is not driven by high correlation between world equity returns and currency returns. Negative returns on the Bloomberg Dollar Spot, Deutsche Bank Currency, and Deutsche Bank Currency Carry indices become positive when managed by AV. The AV managed real estate investment nearly quintuples returns to the S\&P REIT index, 26.7\% vs 5.3\% and the Sharpe ratio is more than four times as high, .995 vs .198.

Unfortunately, equity AV management does not appear to improve the return or Sharpe ratio to investment in commodities. While AV does outperform SV management for the Bloomberg Commodity index, it generates a lower return and worse Sharpe ratio than the buy and hold strategy. This is certainly disappointing, however it may be expected. \citet{gorton_facts_2006,buyuksahin_commodities_2008,NBERw21243} document the lack of relationship between equity and commodity returns. This is likely the genesis for the advice suggesting commodities as a portfolio hedge. Additionally, as \citet{erb_conquering_nodate} show commodity returns are linked to income returns and not prices and link to systematic wealth in a different manner than equities. Thus, it is possible that the systematic risk to investor wealth AV management times is different from the wealth risk related to the performance of commodity investments.

%\bigskip
%\centerline{\bf [Place Table~\ref{tab:tab_altPerf2} about here]}
%\bigskip

Appendix table \ref{tab:tab_altPerf2} shows the drawdown statistics for the equity AV and SV management strategies applied to the alternative asset class indices. In all cases, even for commodities, AV management results in better drawdown statistics. The average depths are shallower, lengths and recoveries shorter. Equity SV management is not as successful. For currencies, SV management is only successful in making the average drawdown depth less for the Bloomberg Dollar Spot Index, a 10.1\% versus 13.6\% loss, but still the AV managed average loss is less at 9.7\%. AV management cuts the average commodity loss by more than 50\%, reducing a 26.6\% loss to 10.1\%; the average length by nearly 70\%, 39.3 versus 12.2 months; and the average recovery time by more than 50\%, 4.3 versus 2.1 months. For all statistics SV management makes the drawdown statistics worse. So, while it may not generate higher returns or better Sharpe ratios, AV management of a commodity investment may allow better sleep at night.

%\bigskip
%\centerline{\bf [Place Table~\ref{tab:tab_altPerf2} about here]}
%\bigskip

Finally, shown in appendix table \ref{tab:tab_altPerf3}, equity AV management of the other asset classes is cheaper than SV management. The equity AV and SV management signals carryover the same relationship from prior performance tables, the AV managed investments have less turnover and thus incur lower transaction costs than SV. This means that they are able to tolerate higher trading costs and have higher break even costs. Neither strategy improves commodity investment returns so neither strategy can tolerate even zero trading costs when managing the Bloomberg commodity index. However, the SV managed investments in the Deutsche Bank Currency and Currency Momentum indices cannot even tolerate zero trading costs while the break even costs for all of the AV managed investments are reasonable.

Clearly, the use of equity AV to manage investments in other asset classes is better than the use of equity SV. Returns, Sharpe ratios, drawdown statistics and trading costs are better in currency and real estate investments. There may be no benefit to AV management in commodities over the buy and hold, but SV management appears to be clearly worse than the buy and hold and AV management is a significant improvement over SV. This is likely due to commodity returns stemming from a risk unrelated to the risk identified by equity returns. The evidence from global equity and across asset class investments suggests that AV management better aligns investment positions to compensated systematic risk.