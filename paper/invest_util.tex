One part still missing from the analysis of the difference in performance between AV and SV is a measure of the impact on different investors. Investors with different risk aversion will experience different utility effects to the constrained returns of AV and SV\footnote{As \citet{moreira_volatility-managed_2017} note, "With no leverage limit, percentage utility gains are the same regardless of risk-aversion because investors can freely adjust theiraverage risk exposure."}. I consider a mean-variance investor as in \cite{Kandel1996}, \cite{Campbell2008}, \cite{Ferreira2011}, \cite{Rapach2016}, and others. To measure changes to investor utility from switching between the SV and AV portfolios and due to leverage constraints, I use the difference in certainty equivalent return, CER. The CER change will measure the change in riskless return an investor demands, given their risk aversion, to use a given portfolio versus another given investment conditions. This can equivalently be though of as a measure of the management fee an investor, given a specific risk aversion, would be willing to pay to have access to an investment fund. CER change is calculated as the difference in risk adjusted return using:
\begin{equation}
\Delta \text{ CER} = \left(\hat \mu_{r_{x}} - \frac{1}{2}\gamma\hat \sigma^{2}_{r_{x}}\right) - \left(\hat \mu_{r_{y}} - \frac{1}{2}\gamma\hat \sigma^{2}_{r_{y}}\right)
\end{equation}
where $\hat \mu_{r_{x}}$, $\hat \mu_{r_{y}}$, $\hat \sigma^{2}_{r_{x}}$, and $\hat \sigma^{2}_{r_{y}}$ are the means and
variances of the returns to the x and y portfolios and $\gamma$ is the investor risk aversion coefficient. I multiply the gains by 12 to annualize them. All investor risk aversion coeffiecients from 1 to 5 are tested for investors subject to investment constraints from a limit of 1 to 3 on the market portfolio, no leverage to 200\% leverage.

\bigskip
\centerline{\bf [Place Figure~\ref{fig:cer_loss} about here]}
\bigskip

As shown in figure \ref{fig:cer_loss}, CER losses due to leverage constraints accumulate for the SV managed portfolio sooner than for the AV managed portfolio. This is due to the extreme leverage positions needed for the SV managed portfolio. Across risk aversion coefficients, leverage constraints bite sooner and cut deeper into the SV managed portfolio until both are driven together when no leverage is allowed. The CER of the AV starts higher at 7.95\%, 8.98\%, and 9.33\% vs 6.87\%, 7.90\%, and 8.25\% for investors with risk aversion coefficients of 1, 3, and 5. When those investors are subject to 200\% leverage constraint the CERs are 7.96\%, 8.99\%, 9.33\%, versus 6.22\%, 7.05\%, and 7.33\%. The AV manged portfolio provides the same investor utility while the benefit from the SV managed portfolio decreases by 9.46\%, 10.76\%, and 11.15\%. To incorporate the starting difference in utility I look at the gains to moving from the SV to AV managed portfolio. 

\bigskip
\centerline{\bf [Place Figure~\ref{fig:cer_gain} about here]}
\bigskip

As shown in figure \ref{fig:cer_gain}, CER gains for the market variance targeting AV managed portfolio are increasing in both risk aversion and leverage use for constrained risk averse mean-variance investors. An investor with a risk aversion coefficient of 2 would capture an annualized CER gain of 1.49\% using 50\% leverage and 1.91\% implementing the AV strategy through the 200\% leveraged ETFs. The most risk averse investors subject to a 20\% leverage limit see a CER gain of 1.35\% while the most risk tolerant see only 1.08\%. The most risk averse investor, using the highest feasible leverage, realizes a CER gain of over 2\% which translates to a utility increase of 27.4\%. This increase is in the neighborhood of those typically seen from return timing strategies. \citep{campbell1997econometrics,moreira_volatility-managed_2017} Risk averse, mean-variance investors see substaincial utility gains switching from the SV to AV managed portfolio and these gains increase with leverage usage.

%The return on the market portfolio is the return on the value-weighted representative average portfolio held by all investors. However, any investor subject to investment constraints or not, risk averse or tolerant benefits from switching from the SV managed portfolio to the AV managed portfolio. They benefit from switching from the market buy and hold to the AV managed portfolio even more. For the same level of variance risk they get far more return with better risk adjusted properties. How is it that this holds over nine decades of market history? Why is the average investor accepting the lessor return of the market portfolio?