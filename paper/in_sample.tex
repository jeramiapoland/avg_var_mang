In order to get an understanding of the relationship between stock market or average variance and returns, I begin with in sample regressions. In each of these regressions all of the information available in the sample is used to estimate the parameters. In general the regressions take this form
\begin{equation}
	y_{t+h} = \alpha + \beta x_{t} + \epsilon_{t}.
\end{equation}
Both monthly and quarterly analysis is done with non-overlaping time periods which effectily sets h equal to 1 in both. In the decomposition of current market variance, it will be the y variable and each of average correlation, average variance, and both AC and AV will serve as x. In the prediction of future market variance, $SV_{t+1}$ will be the y variable and the x variables tested will include current SV as well as SV + AV in addition to the x variables tested in the variance decomposition regressions. In the prediction of next period returns, $r_{t+1}$ will be the y variable tested against the same x variables as in the predition of future market variance.

Table \ref{tab_market_variance_replication} shows a replication of the in sample variance decomposition found in \citet{pollet_average_2010} table 2. The cotemporanious relationships between average correlation, average variance and stock market variance are the same, indeed the calculated values are almost identical. This confirms that average variance is a significant determinant of stock market variance and a significant measure of risk in the mean-variance sense. This verifies that stock market variance, SV, can be decomposed into average correlation, AC, and average variance, AV, but does not indicate any relationship to subsequent realized stock market variance, SV$_{t+1}$ or future returns, $r_{t+1}$. As such, the extension of this analysis to the full sample, either quarterly or monthly, is skipped.

The effectiveness of either market variance or average variance as an investment management signal will be driven primarily by their relationship with future risk and return. Its the trade-off which is key to the leverage management strategy. Assuming that investors hold a portfolio with whose risk-return ratio they are indifferent. When risk increases but expected returns do not, the risk-return ratio become more unattractive and any risk averse investor would like to decrease there position. In table \ref{tab_mv_next}, panel A is a replication of the relationships of current quarter stock market variance, average variance and average correlation and next quarter stock market variance. AV is a significant predictor of SV next period. This holds even when current period SV is included in the regression. In the full samples, average variance accounts for 38\% and 39\% of the variation in next periods market variance. This is even greater than the amount expalained by this periods market variance. Table \ref{tab_mv_next} shows that with full information average variance is the dominant predictor of future market variance. When predicting next period market variance, it generates higher $R^{2}$ and t-statistics in horse races and remains significant when current market variance is included in all samples.

Table \ref{tab_ret_in} panel A shows the results of my replication of table 3, panel A, from \cite{pollet_average_2010}. Confirming their results, AC is a strong predictor of next quarters excess log returns, but neither AV nor SV are predictive. When appearing alone,in the full quarterly sample the coefficient on AV is negative but insignificant so decreasing investment in the market portfolio does not mean a decrease in expected future returns if anything decreasing investment means avoiding approaching losses. However, when controlling for average correlation, the coefficient on AV is not only still negative but significant. This supports evidence in \citet{pollet_average_2010} suggesting that when controlling for average correlation an increase in average variance is a negative economic signal representing risk that investors are not compensated for taking. In the full monthly sample, AV is a significantly negative predictor of next month's excess log return. In the full quarterly and monthly samples SV behaves like AV with smaller coefficients and t-statics indicating a weaker relationship to future returns than AV. In either case investors are unlikely to be punished for pulling back on investment given higher values of AV or SV so long as the relationships hold with the limited information that investors have available at the time they make investment decisions.