
Average variance is an auto-correlated time series; this opens the possibility that predictive regressions using AV have estimation bias as highlighted in \citet{stambaugh_predictive_1999}. \citet{campbell_no_1992} show that the Stambaugh bias in predictive regressions involving volatility measures and future returns can be particularly severe because of a ”volatility feedback” effect. This may also be compromise the direct conclusion that the performance of AV management means AV and AC tell us about changes in risk. To eliminate the Stambaugh bias in the estimated coefficients on AV, I follow the methodology in \citet{Amihud2004} and further make the p-values used for coefficient significance robust through wild-bootstrapping as detailed in \citet{mackinnon_bootstrap_2002}. Robust coefficients and p-values are presented in all in-sample regression tables.
%Table \ref{tab:tab_in_sample_robust} shows that the relationships demonstrated above are unaffected by robust bias correction. Average variance is a predictor of higher average correlation and higher stock market variance across data sets. While AV is a significant predictor of lower returns in the 1962 forward period, it is unrelated to the next month’s log excess returns in the whole data set. So long as the relationships hold with the limited information that investors have available at the time they make investment decisions, AV is likely to be a better leverage management signal than SV.

%To get an understanding of the relationship between stock market or average variance and returns, I begin with in-sample regressions. 
%In each in-sample regression, all of the information available in the sample is used to estimate the parameters. In general, the regressions take this form: 
%\begin{equation}
%	y_{t+1} = \alpha + \beta x_{t} + \epsilon_{t}.
%\end{equation}
%Both monthly and quarterly analysis is done with non-overlaping time periods which effectily sets h equal to 1 in both. In the decomposition of current market variance, it will be the y variable and each of average correlation, average variance, and both AC and AV will serve as x. 
%When h = 0, as in the decomposition of market variance against average asset variance and correllation, the regression is cotemporaneous.  In the monthly predictive regressions, h = 1.
%In the prediction of future market variance, $SV_{t+1}$ will be the y variable and the x variables tested will include current SV as well as SV + AV in addition to the x variables tested in the variance decomposition regressions. In the prediction of next period returns, $RET_{t+1}$ will be the y variable tested against the same x variables as in the predition of future market variance.
The contemporaneous regressions decomposing market variance are left unreported. The results show the same relationships found in  \citet{pollet_average_2010} table 2. For all in-sample regressions, the series are standardized to a mean of zero and standard deviation of one.
%Panel A of Table \ref{table_in_sample} shows a replication of the in sample variance decomposition found in \citet{pollet_average_2010} table 2. The cotemporanious relationships between average correlation, average variance and stock market variance are the same, indeed the calculated values are almost identical. This confirms that average variance is a significant determinant of stock market variance and a significant measure of risk in the mean-variance sense. This verifies that stock market variance, SV, can be decomposed into average correlation, AC, and average variance, AV, but does not indicate any relationship to subsequent realized stock market variance, SV$_{t+1}$ or future returns, $r_{t+1}$. As such, the extension of this analysis to the full sample, either quarterly or monthly, is skipped.
%%\bigskip
%%\centerline{\bf [Place Table~\ref{tab:tab_in_sample_full} about here]}
%%\bigskip

Appendix table \ref{tab:tab_in_sample_full_app} contains the results of regressions run on the full sample from 1926 to 2016. The results largely support the quarterly regressions in \citet{pollet_average_2010}. AV is a significant predictor of next month’s SV in all specifications and slightly better in terms of $R^{2}$. This month’s AV even remains significant in the specification including this month’s SV and the inclusion of SV appears to be of little to no help as the adjusted $R^{2}$ increases only slightly. there is a definite advantage to using this month’s AV in the prediction of next month’s average variance. The adjusted $R^{2}$ of this month's AV is 51.5\% versus 36.7\%. When both AV and SV are included in the predictive regression, AV retains significance and SV does not. Investors are certainly no worse off using AV in the prediction of next month's SV and are better when predicting next months AV. Hence, investors have a signal at least as good for overall risk and better for uncompensated risk. AC is a significant predictor of higher returns in the next month. So, investors lose out on slightly higher returns by divesting when SV is high because AC is high. AV management avoids this. AV is not significantly related to returns in any specification. When included with AC, AV is insignificant. While no return specification is promising as a return timing strategy, the ability to manage risk without giving up return makes AV management a better strategy than SV.
%Panel (a) shows that AV is a significant predictor of next month’s SV in all specifications. A one standard deviation increase in AV means a .627 standard deviation increase in next month’s market variance. This change represents an increase from the mean of .2 to .46. The coefficient on AV is slightly higher than SV, .627 versus .615, and explains slightly more of the variation in SV$_{t+1}$ with an $R^{2}$ of 39\% versus 37.5\%. This month’s AV even remains significant in the specification including this month’s SV. Holding this month’s SV constant, a one standard deviation increase in AV still signals a .368 standard deviation increase in next month’s SV. Again, the inclusion of SV appears to be of little to no help as the adjusted $R^{2}$ only increases from 39\% to 41\%. Panel (b) shows predictive regressions of next month’s average variance. Here, unlike the prior relationship, there is a definite advantage to using this month’s AV in the prediction of next month’s average variance. The adjusted $R^{2}$ of this month's AV is 51.5\% versus 36.7\%. When both AV and SV are included in the predictive regression, AV retains significance and SV does not. Investors are certainly no worse off using AV in the prediction of next month's SV and are better when predicting next months AV. Hence, investors have a signal at least as good for overall risk and better for, what will be shown to be, uncompensated risk. Panel (d) presents the results of directly regressing next month’s log excess return on the market variance, average variance, and average correlation series. Indeed, AC is a significant predictor of higher returns in the next month and a better predictor than SV. So, investors lose out on slightly higher returns by divesting when SV is high because AC is high. AV management avoids this. AV is not significantly related to returns in any specification. When included with AC, AV is insignificant. While no return specification is promising from purely a return timing strategy, the ability to manage risk without giving up return makes AV a better strategy than SV.



%Thus, the results in panel C of table \ref{tab:tab_in_sample_full} make the argument that leverage management using AV is a better idea. While both AV and SV are positively related to next month’s AC, SV is more highly related. The adjusted $R^{2}$ value for the regression of next month’s AC on current SV is more than twice that of current AV. Additionally, each signal is related to higher levels of both AV and SV next month. Hence, the use of SV as a leverage signal means taking on less weight in the market portfolio when next month’s systematic risk is higher relative to idiosyncratic risk. Theoretically, this is avoiding times of higher compensated versus uncompensated risk. 

%The effectiveness of either market variance or average variance as an investment management signal will be driven primarily by their relationship with future risk and return. Its the trade-off which is key to the leverage management strategy. Assuming that investors hold a portfolio with whose risk-return ratio they are indifferent. When risk increases but expected returns do not, the risk-return ratio become more unattractive and any risk averse investor would like to decrease there position. 
%In table \ref{tab_mv_next}, panel A is a replication of the relationships of current quarter stock market variance, average variance and average correlation and next quarter stock market variance. AV is a significant predictor of SV next period. This holds even when current period SV is included in the regression. In the full samples, average variance accounts for 38\% and 39\% of the variation in next periods market variance. This is even greater than the amount expalained by this periods market variance. Table \ref{tab_mv_next} shows that with full information average variance is the dominant predictor of future market variance. When predicting next period market variance, it generates higher $R^{2}$ and t-statistics in horse races and remains significant when current market variance is included in all samples.

%Table \ref{tab_ret_in} panel A shows the results of my replication of table 3, panel A, from \cite{pollet_average_2010}. Confirming their results, AC is a strong predictor of next quarters excess log returns, but neither AV nor SV are predictive. When appearing alone,in the full quarterly sample the coefficient on AV is negative, -.371, but insignificant so decreasing investment in the market portfolio does not mean a decrease in expected future returns if anything decreasing investment means avoiding approaching losses. However, when controlling for market variance, the coefficient on AV is not only still negative but significant. This supports evidence in \citet{pollet_average_2010} suggesting that when controlling for average correlation an increase in average variance is a negative economic signal representing risk that investors are not compensated for taking. In the full monthly sample, AV is a significantly negative predictor of next month's excess log return with a coefficent of . In the full quarterly and monthly samples SV behaves like AV with smaller coefficients and t-statics indicating a weaker relationship to future returns than AV. In either case investors are unlikely to be punished for pulling back on investment given higher values of AV or SV so long as the relationships hold with the limited information that investors have available at the time they make investment decisions.
%\bigskip
%\centerline{\bf [Place Table~\ref{tab:tab_in_sample_full} about here]}
%\bigskip

%Table \ref{tab:tab_in_sample_full} presents the results for the in-sample regressions across the whole CRSP data set, from 1926 to 2016. Across the whole data set, AV is an even better predictor of next month’s SV than current SV. The panel A results here are even better than in table \ref{tab:tab_in_sample_1962}. The results in panel B show the same. The prior results are not only supported but appear better with AV being the best predictor of next month’s AV and SV providing no help and becoming insignificant with both variables are included. As before, across the whole data set stock market variance is more highly related to next month’s average correlation than is average variance. Panel C in table \ref{tab:tab_in_sample_full} supports panel C in table \ref{tab:tab_in_sample_1962}. Panel D is the first place the results appear different in a meaningful way. There is no relationship between AV and next month’s return in panel D of table \ref{tab:tab_in_sample_full}. The coefficient on AV is insignificantly positive with an adjusted $R^{2}$ of -0.1\%. Over the full sample, SV is a better predictor of next month's log excess return. A one standard deviation increase in this month's stock market variance indicates a .056 standard deviation, .3 percentage points, lower return. However, with an adjusted $R^{2}$ of .2\%, there are better return predictors. While it is no longer obvious AV is better
%than SV at both risk and return anticipation across the whole data set, the results still suggest that investors are likely to be better off using AV than SV as it is a better risk anticipation measure and at least unrelated to future returns. AV should capture more returns for the same level of risk, if
%not avoid negative returns.