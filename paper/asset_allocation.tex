Of course, the most direct and practically relevant measure of AV as a portfolio leverage management tool is whether or not it generates portfolio gains. And while superior out-of-sample predictability usually translates into better portfolio performance, here timing leverage to risk, it is important to make adjustments for the riskiness of the managed portfolio and compare performance across dimensions for a full investment picture. The AV managed portfolio may well generate higher annualized returns, but will it have a better Sharpe ratio than the SV managed portfolio?

To measure portfolio performance, in addition to annualized monthly log excess return, I will calculate each portfolios Sharpe ratio, Sortino ratio, two Kappa ratios and Rachev ratio. The classic Sharpe ratio is a symmetric measure of risk and is defined as the ratio of the expected excess portfolio return over the standard deviation of portfolio returns.
\begin{equation}
	\frac{\mathbb{E}[r_{x}]}{\sigma(r_{x})}
\end{equation}
While Sharpe measures each dollar of expected return for dollar of risk, the Sortino ratio attempts to more directly measure the risk most investors worry about. By using only downside deviation in the denominator, the Sortino quantifies each dollar of expected return for each dollar of loss. This downside is measured relative to a target return. \citep{sortino_performance_1994} As the log returns are already excess of the risk free rate, I set the Sortino target to 0 which makes the Sortino formula:
\begin{equation}
	\frac{\mathbb{E}[r_{x} - 0]}{\sqrt  {\int _{{-\infty }}^{0}(0-r_{x})^{2}f(r_{x})\,dr}}
\end{equation}
The Sortino is just a specific instance of a more general risk measurement ratio formula. The Kappa ratio keeps the expected return relative to a target in the numerator but allows any lower partial moment in the denominator. \citep{kaplan_kappa:_2004} With the target again set to 0 the general formula is of the form:
\begin{equation}
\frac{\mathbb{E}[r_{x} - 0]}{\sqrt[\leftroot{-2}\uproot{2}n]{LPM_{n}}} 
\end{equation}
The Sortino ratio is the Kappa$_{2}$ ratio. I calculate Kappa$_{3}$ and Kappa$_{4}$ also to see relative performance of AV and SV management adjusted for negative return skew and kurtosis. The final performance ratio investigated is the Rachev. Developed in \citet{biglova_different_2004}, the Rachev ratio is the expected tail return in the best n\% of the return distribution over the expected tail loss in the worst n\%. I set n = 5 and measure the top 5\% expected return over the expected loss in the worst 5\%. The general Rachev ratio formula is, again against a target return level of 0:
\begin{equation}
\frac {ET{L_{\alpha }}\left({{r_{f}}-x'r}\right)}{ET{L_{\beta }}\left({x'r-{r_{f}}}\right)}
\end{equation}
where $ET{L_{\alpha }}={\frac {1}{\alpha }}\int _{0}^{\alpha }{VaR_{q}\left(X\right)dq}$.

Measuring a difference in each of these measures for a pair of portfolios is not difficult; measuring a significant difference is.  When evaluating the difference in Sharpe ratios, the method in \citet{memmel_performance_2003} seems to be popular. However, when returns are not normally distributed or autocorrelated this method is not valid. AV and SV managed returns, like market returns, are weakly autocorrelated, slightly skewed, and have much fatter tails when compared to normally distributed returns. Moreover, current period returns to either the AV or SV managed portfolio depend on the prior period variance of the market return strongly questioning the i.i.d assumption made in most hyptothesis testing methodologies. Studentized time series bootstrap sampling preserves the time series properties of the AV and SV managed returns which is critical; for example, \citet{scherer_alternative_2004} demonstrates that methods which loose the time dependence in the calculation of differences in Sortino ratios fail to properly estimate the sampling distribution and critical values. Time seires bootstrap methods preserve the structure from the original data and allow for efficient robust hypothesis testing. \citep{politis_stationary_1994,davison1997bootstrap} \citet{ledoit_robust_2008} show that this method is even more efficient than using \citet{newey_simple_1987} or \citet{andrews_improved_1992} heteroskedasticity and autocorrelation corrected standard errors for testing the significance of differences between two portfolio Sharpe ratios. I follow the p-value estimation method in \citet{ledoit_robust_2008} to determine the significance of the difference between the portfolio performance ratio measures of the AV and SV mangement strategies. This uses circular block bootstrapping of the return time series, robust centered studentized statistics computed from the bootstrap samples and is proven to be the most efficient hypothesis testing method. \citet{politis_general_1992,ledoit_robust_2008}

As in \citet{moreira_volatility-managed_2017}, investment weight in the market portfolio is a function of the variance of daily market returns, SV, or the average daily return asset variance, AV, scaled by a constant, $c$. \citet{moreira_volatility-managed_2017} use a constant that scales the variance of the volatility managed portfolio equal to the buy and hold market portfolio. In the basic portfolio weighting specification, I use the same approach so that the returns of both the SV and AV managed portfolios have the same variance as the buy and hold strategy. This constant is denoted $c_{053}$ and it takes different values for SV and AV. This scaling requires knowing the full sample buy and hold return variance. While this induces distortions in the performance ratios, to insure robustness two other specifications for the scaling targets are used. Annual volatilities of 12\% and 10\% are common in academic literature and fund management so $c_{035}$ and $c_{029}$ approximately target those levels. \citet{barroso_momentum_2015,morrison_guarantees_nodate,verma_volatility-targeting_2018,fleming_economic_nodate,hocquard_constant-volatility_2013} Using each of these constant an investor reblances at the end of month t investing in the market portfolio with weight:
\begin{equation}
	w_{x,t} = \frac{c}{x(t)}
\end{equation} where $x(t)$ is either SV$_{t}$ or AV$_{t}$ and hold for month t + 1. Table \ref{tab:tab_weights} shows summary statistics for the resulting investment weights for AV and SV for the three volatility targets. When targeting the volatility of the market portfolio, both portfolios are leveraged into the market on averge with investment weights of 1.3 indicating 30\% leverage. Regardless of the volatility target the SV managed portfolio calls for extreme levels of leverage. Figure \ref{fig:weights_plot} shows that the SV strategy targeting the buy and hold volatility calls for investment weights above the maximum AV weight in several peiords. More than 500\% leverage is needed at the end of the 1920s, throughout the 1960s, and in the 1990s. Given that these levels of leverage are unrealistic for most investors, it will be important to see if there is a difference in performance for the AV and SV strategies under real-world investment constraints.

Before exaiming the constrained portfolio performance, I present the results for SV and AV strategies targeting the buy and hold volatility without investment constraints in table \ref{tab:tab_performance1}. As in \citet{moreira_volatility-managed_2017} the portfolio perforamnce is measured across the whole CRSP data set, however the relative performance is the same or better across the basis data set. Table \ref{tab:tab_performanc1} presents the performance ratios for the SV and AV managed portfolios targeting the buy and hold volatility without investment constraints. The buy and hold market strategy is included for reference. However since \citet{moreira_volatility-managed_2017} establish that the SV managed portfolio out performs the buy and hold, statistical significance results are only presented for the comparison of the SV and AV managed portfolios. The AV managed portfolio generates a statistically significant 1.08 percentage points higher average annualized log excess return. As shown in the bottom panel of figure \ref{fig:returns} the AV strategy builds its performance advantage slowly but consistently starting from the early 1950s and from that time the SV managed portfolio is never a better investment. As both strategies are targeting the same volatility, the significant difference in return translates into a significant difference in Sharpe ratio. At .520 versus .462, for the SV managed portfolio, the AV managed Sharpe ratio is 12.6\% higher. AV also generates significantly higher Kappa$_{3}$ and Kappa$_{4}$ ratios. So while the overall payment for downside risk, measured by Sortino ratio, may not be significantly higher, the payment for downside skewness and extreme downside return is higher. Unfortunately the compensation for downside tail risk is not perfect. SV has a significantly higher Rachev ratio. The volatility managed market portfolio has higher expected tail return potential for each dollar of potential tail loss to the average variance managed portfolio. The maximum annualized return to the SV managed portfolio is 35.3\%, in August 1965. That month, the AV managed portfolio only returns 8.9\% and its maximum return is an annualized 22.11\%. The AV managed portfolio is a ticket with better winning chances and a higher expected return while the SV managed portfolio has a larger jackpot.

More practical analysis of portfolio performance requires the incorporation of limits on the level of investment taken in the market portfolio. Leverage of 50\%, a coefficient of 1.5 on the market, is a common constraint meant to mimic real market leverage constraints for the average investor based in part on the Reg. T margin requirement\footnote{Federal Reserve Board Regulation T (Reg T) establishes a baseline requirement that investors deposit 50\% of an investment position in their margin trading accounts, however a brokerage house may set a higher equity requirement.}. \citep{Rapach2016,moreira_volatility-managed_2017,deuskar_margin_2017} There are at least two exchange traded funds, ETFs, which three times the return of the SP500.\footnote{The Direxion Daily S\&P 500 Bull 3x Shares ETF, symbol SPXL, and ProShares Ultra Pro S\&P 500 ETF, symbol UPRO, are two such funds.} So, I take a market coefficient of three as the maximum feasible investment a typical investor can make in the market portfolio. Table \ref{tab:table_performance2} panel c presents the results from applying investment constraints after calculating the weights for AV and SV targeting the buy and hold volatility. While both portfolios still outperform the buy and hold, The seperations in average annualized excess return, 1.71\% and 2.07\%, are even greater when investment constraints are applied. Panels a and b in figure \ref{fig:returns} show the effects of the growing separation. While SV is barely able to clear the buy and hold strategy under typical brokerage constraints, returns to the AV managed portfolio remain clearly above. Investors that use a leveraged SP500 ETF to impliment the AV managed portfolio strategy are rewarded with returns significantly higher than the SV managed portfolio suggested weights. The investment restrictions pull the volatility of the AV managed portfolio too far from the SV returns to generate significant differences in performance ratios. However, investors using the leveraged ETFs are rewarded not only with higher returns but significantly better performance ratios across the board with the exception of the Rachev ratio which is at least no longer signficantly better for the SV managed portfolio. The ETF leverage constrained AV strategy even generates better Sharpe and Sortino ratios than the unconstrained strategy. The results in panels a and b demonstrate that better performance of AV is not a result of or contingent on looking to the volatility of the buy and hold strategy. As the targeted volatility is lowered the difference in performance between AV and SV becomes more significant. With the exception of the Rachev ratio, every performance measure is significantly better for the AV managed portfolio given lower volatility targets and investment constraints.

One part still missing from the analysis of the difference in performance between AV and SV is a measure of the impact on different investors. Investors with different risk aversion will experience different utility effects to the constrained returns of AV and SV\footnote{As \citet{moreira_volatility-managed_2017} note, "With no leverage limit, percentage utility gains are the same regardless of risk-aversion because investors can freely adjust theiraverage risk exposure."}.   