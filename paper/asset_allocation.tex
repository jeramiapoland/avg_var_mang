%Of course, the most direct and practically relevant measure of AV as a portfolio leverage management tool is whether or not it generates portfolio gains. And while superior out-of-sample predictability usually translates into better portfolio performance, here timing leverage to risk, it is important to make adjustments for the riskiness of the managed portfolio and compare performance across dimensions for a full investment picture. The AV managed portfolio may well generate higher annualized returns, but will it have a better Sharpe ratio than the SV managed portfolio?
The most direct and practically relevant measure of AV as a portfolio leverage management tool is whether or not it generates portfolio gains. It is also important to make adjustments for the riskiness of the managed portfolio and compare performance across dimensions for a full investment picture. The AV managed portfolio may well generate higher annualized returns, but will it have a better Sharpe ratio than the SV managed portfolio?

To measure portfolio performance, in addition to annualized monthly log excess return, I will calculate each portfolio's Sharpe ratio, Sortino ratio, and two Kappa ratios and factor $\alpha$s. The classic Sharpe ratio is a symmetric measure of risk and is defined as the ratio of the expected excess portfolio return over the standard deviation of portfolio returns.
\begin{equation}
	\frac{\mathbb{E}[r_{x}]}{\sigma(r_{x})}
\end{equation}
While Sharpe measures each dollar of expected return for dollar of risk, the Sortino ratio attempts to more directly measure the risk most investors worry about. By using only downside deviation in the denominator, the Sortino quantifies each dollar of expected return for each dollar of loss. This downside is measured relative to a target return. \citep{sortino_performance_1994} As the log returns are already excess of the risk free rate, I set the Sortino target to 0 which makes the Sortino formula:
\begin{equation}
	\frac{\mathbb{E}[r_{x} - 0]}{\sqrt  {\int _{{-\infty }}^{0}(0-r_{x})^{2}f(r_{x})\,dr}}
\end{equation}
The Sortino is just a specific instance of a more general risk measurement ratio formula. The Kappa ratio keeps the expected return relative to a target in the numerator but allows any lower partial moment in the denominator. \citep{kaplan_kappa:_2004} With the target again set to 0 the general formula is of the form:
\begin{equation}
\frac{\mathbb{E}[r_{x} - 0]}{\sqrt[\leftroot{-2}\uproot{2}n]{LPM_{n}}} 
\end{equation}
The Sortino ratio is the Kappa$_{2}$ ratio. I calculate Kappa$_{3}$ and Kappa$_{4}$ also to see relative performance of AV and SV management adjusted for negative return skew and kurtosis. All of the measures are annualized, e.g., log returns are multiplied by 12 and ratios, like Sharpe, are multiplied by the square root of 12. Fama-French three and three-factor with momentum $\alpha$s are also calculated and annualized. The $\alpha$s come from the standard regression:
\begin{equation}
	r_{x,t} = \alpha + \beta\chi_{t} + \epsilon_{t}
\end{equation}
where $\chi$ contains the combinations of the small minus big, SMB, high minus low, HML, and market factors with and without the winners minus loser, WML or Mom, momentum factor for x equals AV or SV.

Measuring a difference in each of these measures for a pair of portfolios is not difficult; measuring a significant difference is.  When evaluating the difference in Sharpe ratios, the method in \citet{memmel_performance_2003} seems to be popular. However, when returns are not normally distributed or auto-correlated this method is not valid. AV and SV managed returns, like market returns, are weakly auto-correlated, slightly skewed, and have much fatter tails when compared to normally distributed returns. Moreover, current period returns to either the AV or SV managed portfolio depend on the prior period variance of the market return strongly questioning the i.i.d assumption made in most hypothesis testing methodologies. Studentized time series bootstrap sampling preserves the time series properties of the AV and SV managed returns which is critical; for example, \citet{scherer_alternative_2004} demonstrates that methods which loose the time dependence in the calculation of differences in Sortino ratios fail to properly estimate the sampling distribution and critical values. Time series bootstrap method preserves the original data structure and allows for efficient robust hypothesis testing. \citep{politis_stationary_1994,davison1997bootstrap} \citet{ledoit_robust_2008} show that this method is even more efficient than using \citet{newey_simple_1987} or \citet{andrews_improved_1992} heteroskedasticity and auto-correlation corrected standard errors for testing the significance of differences between portfolio Sharpe ratios. I follow the p-value estimation method in \citet{ledoit_robust_2008} to determine the significance of the difference between the ratios of the AV and SV management strategies. This uses circular block bootstrapping of the return time series, robust centered studentized statistics computed from the bootstrap samples and is proven to be the most efficient hypothesis testing method.\footnote{I randomize the block sampling size rather than calculate the optimal size. This results in wider standard errors and a more conservative test. \citet{lahiri_theoretical_1999}.} \citep{politis_general_1992,ledoit_robust_2008}

As in \citet{moreira_volatility-managed_2017}, investment weight in the market portfolio is a function of the inverse of the variance of daily market returns, SV, or the average daily return asset variance, AV, scaled by a constant, $c$. \citet{moreira_volatility-managed_2017} use a constant that scales the variance of the volatility managed portfolio equal to the buy and hold market portfolio. In the basic portfolio weighting specification, I use the same approach so that the returns of both the SV and AV managed portfolios have the same variance as the buy and hold strategy. This constant is denoted $c_{BH}$ and it takes different values for SV and AV. This scaling requires knowing the full sample buy and hold return variance. While this does not distort the performance ratios, to insure robustness two other specifications for the scaling targets are used. Annual volatilities of 12\% and 10\% are common in academic literature and fund management so $c_{12}$ and $c_{10}$ approximately target those levels. \citep{barroso_momentum_2015,morrison_guarantees_nodate,verma_volatility-targeting_2018,fleming_economic_nodate,hocquard_constant-volatility_2013} Using each of these constant an investor re-balances at the end of month t investing in the market portfolio with weight:
\begin{equation}
	w_{x,t} = \frac{c_{target}}{x(t)}
\end{equation} where $x(t)$ is either SV$_{t}$ or AV$_{t}$ and hold for month t + 1. 

\bigskip
\centerline{\bf [Place Table~\ref{tab:tab_weights} about here]}
\bigskip
Table \ref{tab:tab_weights} shows summary statistics for the resulting investment weights for AV and SV for the three volatility targets. The investment weight turnover, show later in \ref{tab:tab_costs}, mean, and standard deviation for the SV management strategy targeting the buy and hold are nearly identical to the values in \citet{moreira_volatility-managed_2017}. When targeting the volatility of the market portfolio, both portfolios are leveraged into the market on average with investment weights of 1.3, indicating 30\% leverage. Regardless of the volatility target the SV managed portfolio calls for extreme levels of leverage. Figure \ref{fig:weights_plot} shows that the SV strategy, regardless of target, calls for investment weights above the maximum AV weight in several periods. More than 500\% leverage is needed at the end of the 1920s, throughout the 1960s, and in the 1990s. Given that these levels of leverage are unrealistic for most investors, it will be important to see if there is a difference in performance for the AV and SV strategies under real-world investment constraints and to investigate the associated costs generated by the trading needed for the SV managed portfolio.
\bigskip
\centerline{\bf [Place Figure~\ref{fig:weights_plot} about here]}
\bigskip
