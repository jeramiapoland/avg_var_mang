Weighting investment by the inverse of the average asset variance, AV, rather than SV, increases market investment when total variance is expected to be low and decreases when higher total variance and lower returns are expected. The results are better Sharpe, Sortino and Kappa ratios with better Fama-French five factor and five factor plus momentum alphas. With better access to leverage timing on compensated risk, investors capture more utility with lower costs. AV portfolio investors, as well as fund managers, are more protected agains drawdowns. I analyze the relationship between AV, the average variance of individual asset returns, and future risk and future returns to reveal the mechanism that makes AV a better timing signal. AV is a significant predictor of future portfolio variance, controlling for current variance, both in and out of sample. In contrast, AV is not significantly related to future returns and thus serves as a leverage timing signal. By using the decomposition of market variance, I add a better portfolio leverage management signal to the literature, because AV management better aligns investment to times of compensated risk. These investment benefits manifest because the AV managed portfolio takes advantage of the correlation risk and return dynamics rather than the portfolio variance and return dynamics.

AV management outperforms across the globe. Returns, Sharpe ratios, drawdown statistics and costs are better in 9 of 10 countries studied. Performance is also better for a globally diversified equity investment. Not only is the average variance of equity returns better for the management of equity investments but it is better across asset classes. AV generates better performance in currency and real estate investment supporting the argument that AV management better aligns investment to times that compensated systematic risk is higher. SV management fails to perform, in many cases its worse than the buy and hold strategy, across asset classes. Hence, the AV managed portfolio adds another dimension to the risk-return literature.
%The change in ratio contributes evidence against the lottery explanation of the low-risk anomaly in mean-variance analysis and supports the leverage constraints explanation. Furthur evidence shows that the returns and capital market line responds to proxies for tight lending conditions but not high lottery preference. This finding contrasts with conclusions reached in the study of cross-sectional low-risk anomalies which have been explained through the behavioral, lottery, channel.%By scarificing some potential extremely positive returns, investors using AV have access that better responds to risk shocks and participates more in the subsequent recovery. This supports prior literature which argues that individual assets should signal changes due to economic events before aggregate signals. All of this, of course, depends on the data and the construction of the individual and aggreagate signals.

%We study how time-variation in margin capacity, defined as the aggregate excess debt capacity of
%levered investors buying securities on the margin, is related to the future economic conditions. We
%find that margin capacity strongly predicts many macroeconomic variables of interest. In particular,
%high margin capacity predicts lower aggregate stock returns, interest rates, intermediary equity
%capital, growth in aggregate earnings, dividends, employment and overall economic activity. It also
%forecasts higher VIX, and macro, financial, and policy uncertainty and tighter lending standards
%by banks. We argue that the investors accumulating margin capacity are likely to be informed and
%hence their cautious behavior is a pessimistic signal about future economic conditions.
%Our study contributes to three strands of literature in the macro-finance area. First, we con-
%tribute to the long literature on equity premium prediction. Margin capacity generates high out-
%of-sample R 2 and delivers high investment gains. Second, we establish that margin capacity is
%not driven by intermediary or borrowing constraints but, in fact, precedes them. Thus, our study
%introduces a predictor for the conditions which the intermediary asset pricing literature identifies
%as a key determinant of asset prices. Third, we show that margin investors reduce leverage ahead
%of higher risk and lower economic growth. This finding complements the recent literature which
%shows that an increase in the aggregate levels of debt in the economy precedes poor macroeconomic
%outcomes.